\documentclass{article}
\begin{document}
\begin{itemize}
\item
Il SW \`e diviso in vari task
\begin{enumerate}
\item Un task ``principale'' che legge i comandi da una coda condivisa
  di ``pacchetti interni'' nella coda scrivono i task che leggono da
  rete e quello che gestisce i pacchetti di controllo 
\item Una coppia di task per ogni ingresso
\begin{enumerate}
\item Un task ``stretto'' che legge i pacchetti che arrivano da rete e
  li scrive in un buffer condiviso ``primitivo''.  Questo task deve
  essere molto veloce per non perdere pacchetti.  Questo task pu\`o
  generare pacchetti di timeout (sempre inseriti nel buffer condiviso) 
\item Un task ``travasatore'' che prende i pacchetti dal buffer
  primitivo e li copia nella coda di ingresso del task
  principale. Questo task preleverebbe i pacchetti in gruppi dal
  buffer interno, in modo da svuotare la coda. 
\end{enumerate}

La struttura a due task sopra delineata \`e necessaria per evitare che
il processo ``stretto'' aspetti troppo nel tentativo di accedere alla
coda di ingresso (su cui scrivono tutti) 

\item Un task per la gestione dei pacchetti di controllo.  Questo task
  riceve richieste di spedire pacchetti di controllo e riceve
  segnalazioni di ACK.  Gestisce la ritrasmissione dei pacchetti di
  controllo.  In risposta ad un ACK produce un pacchetto che inserisce
  nella coda di ingresso.  Il task principale interpreter\`a  tale
  pacchetto, aggiustando lo stato interno.  La soluzione del pacchetto
  nella coda di ingresso permette una gestione veloce dell'ACK (appena
  ricevuto il task principale si risveglia) e non c'\`e rischio di
  deadlock. Questo task gestisce anche eventuali timeout fatali (dopo
  N tentativi ci rinuncia).  In caso di timeout fatale viene inserito
  nella coda un pacchetto CANCEL. 
\end{enumerate}

\item Ogni porta di uscita (collegata con un opportuno canale) ha
diversi stati: scollegata, attesa ACK porta, attesa ACK default,
pronta.  Gli stati avanzano in risposta agli ACK. 
\end{itemize}




\end{document}
