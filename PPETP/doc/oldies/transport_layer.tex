%-*- mode: latex; ispell-local-dictionary: "english" -*-
\iffalse
\documentclass{rfc}
\else
\documentclass{medusabook}
\newcommand{\fullname}[1]{#1}
\newcommand{\organization}[1]{}
\newcommand{\address}[1]{}
\fi
\usepackage{epsf,epsfig}
\usepackage{amsmath}
\usepackage{amsbsy}
\usepackage{amssymb}
\usepackage{amsthm,amsfonts}
\usepackage{color}
\usepackage{srcltx}
\usepackage{appendix}
../../doc/macrodef.tex
%
\title{Peer-to-Peer Epi-Transport Protocol (\ppmtp;)}

\author{
\fullname{Riccardo Bernardini}
\organization{University of Udine}
\address{street=Via delle Scienze, 208;city=Udine;code=33100;
country=IT}}
%
\setcounter{secnumdepth}{3}
\setcounter{tocdepth}{3}
\begin{document}
  \maketitle
%  \begin{abstract}
%    This document specifies a distributed transport protocol based on
%    a peer-to-peer structure.
%  \end{abstract}
\tableofcontents

\chapter{Pre-introduzione}
\label{sect:6;transport_layer}

\section{Il come e il perch\'e di questo documento}
\label{sub:6.0;transport_layer}

In principio c'era \medusa-, un sistema ``completo'' per la
trasmissione/ricezione in streaming di video live su reti P2P.  Poi
scoprii l'RTSP (Real Time Streaming Protocol) e SDP (Session
Description Protocol) e cominciai una revisione del protocollo
server/client di \medusa- per avvicinarlo a questi due protocolli
gi\`a pronti.  Cos\`\i{} facendo mi resi conto che se l'idea di
\medusa- fosse stata fatta scendere a livello di trasporto, sarebbe
stato possible integrare facilmente \medusa-  con gli strumenti
attualmente disponibili, semplicemente aggiungendo al software
esistente un modulo in grado di interfacciarsi col nuovo protocollo di
trasporto.  

Un esempio chiarir\`a (spero) l'idea.  Supponiamo di voler ascoltare
un brano musicale \ttt la-espero.mp3!\footnote{Esiste\ldots \`e l'inno
esperantista.} presente sul server remoto \ttt
http://www.example.com!.  La prima cosa che il nostro programma
\emph{client} fa \`e di interrogare il server remoto chiedendogli
informazioni sul file MP3.  L'interrogazione avviene tramite il
comando \ttt DESCRIBE! del protocollo RTSP

\begin{verbatim}
  DESCRIBE rtsp://www.example.com/la-espero RTSP/1.0
  CSeq: 1
\end{verbatim}

Il server risponde con una descrizione delle caratteristiche del brano
richiesto

\begin{verbatim}
  RTSP/1.0 200 OK
  CSeq: 1
  Content-Type: application/sdp
  Content-Length: ...
  
  v=0
  o=- 2890844256 2890842807 IN IP4 172.16.2.93
  s=RTSP Session
  i=An Example of RTSP Session Usage
  a=control:rtsp://twister
  t=0 0
  m=audio 0 RTP/AVP 0
  a=control:rtsp://twister/audio
\end{verbatim}
%
La linea pi\`u interessante \`e la penultima (quella che comincia per
\ttt m=!) che informa il client che il file verr\`a trasmesso usando
RTP con profilo AVP (Audio Video Profile \cite{rfc3551}) e il
\emph{payload} ha tipo 0 (1 canale audio a 8~kHz codificato con
PCMU\footnote{Ossia, 8 bit con scalatura logaritmica, secondo
  raccomandazione CCITT G.711 \cite{rfc3551}. Un po' 'na schifezza, ma
insomma\ldots}). 

Il client a questo punto risponde con

\begin{verbatim}
  SETUP rtsp://www.example.com/twister/la-espero.mp3 RTSP/1.0
  CSeq: 1
  Transport: RTP/AVP/UDP;unicast;client_port=3056-3057
\end{verbatim}
%
chiedendo al server di preparsi a trasmettere \ttt la-espero.mp3!
usando RTP con profilo AVP su UDP, in maniera unicast e usando la
porta 3056 per i dati e la 3057 per il controllo (RTCP).  Il server
risponde

\begin{verbatim}
  RTSP/1.0 200 OK
  CSeq: 1
  Session: 12345678
  Transport: RTP/AVP/UDP;unicast;client_port=3056-3057;
             server_port=5000-5001
\end{verbatim}
%
Supponiamo ora di avere un protocollo di epi-trasporto \ppmtp; (che
magari si appoggia su UDP) basato su \medusa-.  In questo caso il
comando di SETUP potrebbe essere

\begin{verbatim}
  SETUP rtsp://www.example.com/twister/la-espero.mp3 RTSP/1.0
  CSeq: 1
  Transport: RTP/AVP/\ppmtp;/UDP;upload_bandwidth=512;
             max_channel=1;control=4536
\end{verbatim}
%
Il comando di \ttt SETUP! comunica al server che siamo disposti a
spendere 512~kbit/s di banda di upload, possiamo aprire al massimo un
canale (vedremo pi\`u avanti cosa intendiamo per ``canale'') e che
aspettiamo eventuali comandi di controllo sulla porta 4536.  Il server
a questo punto potrebbe rispondere con 

\begin{verbatim}
  RTSP/1.0 200 OK
  CSeq: 1
  Session: 12345678
  Transport: RTP/AVP/\ppmtp;/UDP;n_input=16;n_output=3;reduction=12
\end{verbatim}
%
La risposta del server ci chiede di aprire 16 porte di ingresso (per
ricevere da 16 peer), 3 porte di uscita e ci comunica che il fattore
di riduzione che dobbiamo applicare \`e pari a 12.  Il client, usando
un'opportuna API, configurer\`a il proprio \ttt \ppmtp;! secondo le richieste
del server e il player invece di eseguire una \ttt recv! su una porta
UDP eseguir\`a il \ttt recv! sul modulo \ttt \ppmtp;!.

\section{Cosa trovi in questo documento}
\label{sub:6.1;transport_layer}

L'idea balzana ed abbastanza ambiziosa \`e quella di arrivare a creare
un RFC per il \ppmtp;.  Il resto di questo documento \`e quindi scritto
in inglese e in uno stile che si avvicina a quello delle RFC.  La mia
idea sarebbe di dare le specifiche di \medusa- in quattro documenti
differenti

\begin{enumerate}
\item
Un documento (\S\ref{sect:4;transport_layer}) in cui si descrive il
formato dei pacchetti scambiati tra i vari peer, ma non il modo di
eseguire la riduzione e come i ``parametri di riduzione'' (fattore e
vettore di riduzione) vengano comunicati al nodo.  Con un approccio
simile a quanto fatto per l'RTP, si delega la descrizione dei dettagli
di riduzione ad un altro documento.
\item
Un documento (\S\ref{sect:5;transport_layer}) in cui si danno i vari
dettagli sul processo di riduzione.
\item
I primi due documenti descrivono in maniera completa i dettagli di
``basso livello'' del protocollo di trasporto, ma non come i vari peer
entrino in contatto tra loro.  Nonostante tali dettagli non facciano
parte, ovviamente, del protocollo di trasporto, penso sia comunque
utile prevedere un protocollo di controllo della connessione posto ad
un livello pi\`u alto.  Tale protocollo \`e descritto in
\S\ref{sect:7;transport_layer}.
\item
L'ultimo documento (\S\ref{sect:8;transport_layer}) descriver\`a gli
adattamenti da fare a RTSP e SDP per integrarli col nuovo livello di
epi-trasporto.
\end{enumerate}
%
Uno schema di massima di un'applicazione basata su \ppmtp; \`e
visibile in \fref{cpptp}.

\begin{figure}
\centerline{\unafigura{cpptp}}
\caption{Schema di un'applicazione che usa il protocollo di
  epi-trasporto descritto in questo documento.
\label{fig:cpptp}}
\end{figure}
%

\chapter{Introduction}
\label{sect:3;transport_layer}

\section{Motivation}
\label{sub:3.0;transport_layer}

This document proposes a techniques for transmission of large amount
of data (e.g., video streaming) over a network with peer-to-peer (P2P)
structure.  The reason for using a P2P network for streaming video to
a large number of users is that with the P2P approach each new user
acts also a new server, lowering the burden of the video source.

The main drawbacks in using P2P for video streaming to a large number
of home users derive from the fact that home users differ from true
servers in several respects

\begin{itemize}
\item
Home users typically have enough download bandwidth to receive video,
but not enough bandwidth to transmit it.  Therefore, a simple
``mirroring'' of the received data toward other users is not possible.
\item
A home user can leave the network without any warning, maybe because
of a crash.  If countermasures are not taken, this would leave other
users without data.
\item
A home user could act in a malicious way, trying, for example, to
pollute the P2P network with wrong data.
\end{itemize}
%
In \cite{bernardini08:dcc08} a possible solution to the above problems
is proposed.  The key result in \cite{bernardini08:dcc08} is a
technique which allows each node to transmit with a rate which is $R$
times smaller than the received data rate.  Value $R$ is called in
\cite{bernardini08:dcc08} ``reduction factor.''  In order to do the
rate reduction each node interprets each packet as an $R$-rows matrix
$P$ of elements of Galois field $GF(2^n)$.  Such matrix is left
multiplied by a row vector
$$
r(b) = [1, b, b^2, \ldots, b^{R-1}]
$$
%
(called ``reduction vector'') where $b$ is an element of $GF(2^n)$
randomly chosen by the node at startup time.  The bit corresponding to
product $r(b)*P$ are sent to the other peers.  Since the number of
elements of $r(b)*P$ is $R$ times smaller than the number of elements
of $P$, the required upload bandwidth is $R$ times smaller.  In
\cite{bernardini08:dcc08} is shown that if a node can recover the
original packet as soon as it receives $R$ packets relative to $R$
different values of $b$.

The goal of this document is to formalize a pseudo-transport protocol
based on the P2P approach described in \cite{bernardini08:dcc08}.
Transforming the solution  in \cite{bernardini08:dcc08} in a transport
protocol would allow easy adaptation of already available streaming
technologies to the peer-to-peer structure.

\section{\ppmtp; node model}
\label{sub:3.2;transport_layer}

In \ppmtp; we suppose that each peer acts as follows
%
\begin{enumerate}
  \item
     It receives several different reduced versions of a packet.  
  \item
     As soon as enough reduced versions are received, it reconstructs the
     packet. 
  \item
     The reconstructed packet is reduced by multiplying it by the
     reduction vector.  The reduced packet is transmitted to the other
     peers.  
\end{enumerate}
%
Note that a node could have more than one reduction vector.  In such
case it would repeat the third step above for every reduction
vector. We expect that nodes with a large upload bandwidth
(superpeers) will use more than one reduction vector in order to be
able to send more than one reduced stream to a single node.

\section{Scope}
\label{sub:3.1;transport_layer}

The goal of this document is to formalize a pseudo-transport protocol
based on the P2P approach described in \cite{bernardini08:dcc08}.
Actually, the specification is conceptually split into three parts
that will possibly become three different documents

\begin{itemize}
\item
A document which specifies how reduced packets are exchanged among the
peers in the network.  This document leaves unspecified how the
reduction is carried out and it must be completed by a ``reduction
profile'' document describing the reduction technique.
\item
A basic reduction profile document based on the results in
\cite{bernardini08:dcc08}.
\item
A higher-level protocol used to negoziate the connection between
different peers.  Note that this protocol is not part of the
transport protocol, but we feel that the specification would not be
complete without it.
\end{itemize}
%

\chapter{Basic epi-transport specification}
\label{sect:4;transport_layer}

\section{Transport protocol}
\label{sub:4.1;transport_layer}

No specific transport protocol is specified.  It is not necessary that
the transport protocol used by \ppmtp; grants for the delivery of
every packet nor for the preservation of transmission order.  For
example, UDP could be used as a transport protocol for \ppmtp;.

\section{Packet format}
\label{sub:4.0;transport_layer}

The packets exchanged between peers have the following format

\begin{itemize}
  \item
    A fixed 32-bit header with the following structure

    \begin{verbatim}
       -- Structure of a packet exchanged between peers
       for Header'Bit_Order use High_Order_First;
       for Header use
         record
           Version   at 0 range 0..1; -- Field V = 0
           Inline    at 0 range 2..2; -- Field I
           Change    at 0 range 3..3; -- Field C
           Padding   at 0 range 4..4; -- Field P
           Marker    at 0 range 5..7; -- Field M, profile specific

           Profile   at 1 range 0..3;
           Timestamp at 1 range 4..23;
         end record;
    \end{verbatim}
  \item
    An optional header which depends on the chosen profile. This part
    is present only if the \ttt Inline! bit is set.
  \item
    The data payload, possibly padded (if the \ttt Padding! bit is set)
\end{itemize}

In picture form

\vbox{
\begin{verbatim}
 0                   1                   2                   3
 0 1 2 3 4 5 6 7 8 9 0 1 2 3 4 5 6 7 8 9 0 1 2 3 4 5 6 7 8 9 0 1 2
+-+-+-+-+-+-+-+-+-+-+-+-+-+-+-+-+-+-+-+-+-+-+-+-+-+-+-+-+-+-+-+-+-+
|V=0|I|C|P|  M  |Profile|               Timestamp                 |
+=+=+=+=+=+=+=+=+=+=+=+=+=+=+=+=+=+=+=+=+=+=+=+=+=+=+=+=+=+=+=+=+=+
:               Optional Profile-specific data ...                :
+-+-+-+-+-+-+-+-+-+-+-+-+-+-+-+-+-+-+-+-+-+-+-+-+-+-+-+-+-+-+-+-+-+
:                           Payload                               :
+-+-+-+-+-+-+-+-+-+-+-+-+-+-+-+-+-+-+-+-+-+-+-+-+-+-+-+-+-+-+-+-+-+
\end{verbatim}
}

The header fields have the following meaning

\begin{description}
\label{description:4.0.0;transport_layer}
\item[Version (V): 2 bit]
This field identify the protocol version.  This document
corresponds to \ttt V=0!.
\item[Inline  (I): 1 bit]
If this bit is 1, details about the reduction process are specified in
the profile-specific header which follows the first 32-bit word.  If
this bit is 0, the details to be used are the ones fixed by the last
packet with the \ttt Change! bit set.
\item[Change  (C): 1 bit]
If this bit is 1, bit \ttt Inline! must be 1 too and the details given
in the optional part become the default details to be used in the
following.  A packet with bit \ttt Change! implicitly requires that an
ACK be sent back to the packet source.  As long as the source does not
receive the ACK, it cannot assume that the default setting have been
estabilished.   The Acknowledge packet is a 24-bit packet with the
following format

\begin{verbatim}
  for ACK_Packet'Bit_Order use High_Order_First;
  for ACK_Packet use
    record
      Reserved  : at 0 range 0..3 := 1; 
      Timestamp : at 0 range 4..23;
    end record; 
\end{verbatim}

\item[Padding (P): 1 bit] Similarly to the RTP specification
\cite{rfc3550}, if this bit is set, the packet contains one or more
additional padding octets at the end which are not part of the
payload.  The last octet of the padding contains a count of how many
padding octets should be ignored, including itself.
\item[Marker (M): 3 bit] Similarly to the RTP
specification \cite{rfc3550}, the interpretation of this field is defined by a
profile.
\item[Profile    : 4 bit] The profile number.  Currently only profile
  0 (described in the following) is defined.
\item[Timestamp : 20 bit] A sequence number which increments for every
  packet sent by the root source. Packets with the same timestamp
  field MUST derive from the same root packet.\\
  Similarly to the requirements in the RTP specification \cite{rfc3550}, the
  initial value of this field SHOULD be random (unpredictable) to make
  known-plaintext attacks on encryption more difficult
\end{description}
%
Note that the payload section can be empty.  An empty payload can be
useful to set the default reduction parameters without sending any
data (for example, when the source peer makes first contact with the
destination peer).

\section{What a profile must specify}
\label{sect:4.0;transport_layer}

A profile description, in order to be complete, must at least specify

\begin{itemize}
\item
The profile number (field \ttt Profile! in the header)
\end{itemize}
%

\chapter{Basic profile}
\label{chap:0;transport_layer}


\chapter{Vandermonde profile}
\label{sect:5;transport_layer}

The basic reduction profile (corresponding to \ttt Profile! field
equal to 0) is derived from the proposal in \cite{bernardini08:dcc08}.

\section{Reduction details}
\label{sub:5.0;transport_layer}

The reduction operated by a peer is uniquely determined when the
following parameters are specified: the Galois field size, the
reduction factor and the value $b$ used to construct the reduction
vector.  Tyipically the first two parameters remain constant during a
session, while the third one can occasionally change.

\begin{example}
A typical case which require the change of $b$ is when a node is not
able to recover a complete packet, but it wants nevertheless to
propagate the information.  In such a case it can transmit one of the
received packets which cannot be expected to have a reduction vector
equal to the one choose by the node.  In that case the peer needs to
change temporally the reduction vector.
\end{example}
%
The reduction details can be transmitted in two formats: a full format
(64 bit) and a compact one (32 bit).  The definition of the two 
formats are

\begin{verbatim}
  for Full_Format use
    record
      Reply_Port       : at 0 range 0..15;
      GF_Size          : at 2 range 0..1;
      Port_Included    : at 2 range 2..2;
      Reserved         : at 2 range 3..7;
      Reduction_Factor : at 3 range 0..7;

      Reduction_Vector : at 4 range 0..31;
    end record;

  for Compact_Format use
    record
      Reduction_Vector : at 0 range 0..31;
    end record;
\end{verbatim}

In other words, the full format is simply the compact format with a
32-bit word which specifies the field size and the reduction factor.
It is expected that the compact format will be used to change
temporally the reduction vector (i.e., with the \ttt Change! bit equal
to zero), while the full format will be used during the initialization
phase. 

Field \ttt GF_Size! is to be interpreted as follows: the size of the
Galois field used in the reduction algorithm is $2^{8(\text{\ttt
    GF\_Size!}+1)}$, that is, we have the following correspondence

\begin{center}
\begin{tabular}{ccl}
  \ttt GF_Size! & Field size & 
\multicolumn{1}{c}{Polynomial} \\
  0  & $2^8$    & $x^8 + x^4 + x^3 + x^2 + 1$\\
  1  & $2^{16}$ & $x^{16} + x^5 + x^3 + x^2 + 1$\\
  2  & $2^{24}$ & $x^{24} + x^{16} + x^{15} +x^{14} +x^{13}+x^{10}+x^9
  x^7 + x^5 + x^3 + 1$\\
  3  & $2^{32}$ &
$x^{32} + x^{15} + x^{9} + x^7 + x^4 + x^3 + 1$
\end{tabular}
\end{center}
%
The Galois field is implemented as the field of polynomial with binary
coefficients modulo the polynomial given in the table.

The field \ttt Reduction_Vector! specifies the value of $b$ used to
construct the reduction vector.  Only the \ttt GF_Size!+1 less
significative bytes of \ttt Reduction_Vector! are used, the other
SHOULD be set to zero.  The $k$-th less significative bit of \ttt
Reduction_Vector! MUST be interpreted as the coefficient of $x^k$.

\section{Security consideration}
\label{MD5-signing}

A possible threat in a P2P transport solution is that some malicious
node could inject into the network packets with random data, polluting
the transmitted stream. Because of this, we introduce a procedure
which allows the root server to add to the transmitted packets an MD5
hash of the packet.  The use of the hash is optional and it is decided
at the beginning of the session by means external to the protocol
(see, for example, the extension to the RTSP \ttt Transport! header
described in Section~\ref{subsub:0;transport_layer}).


\subsection{MD5 signing}
\label{subsub:5.1.0;transport_layer}

The root packet, before undergoing the first reduction, MUST be
preprocessed as follows

\begin{enumerate}
\item
Root packet P is padded, if necessary, with zeros to make its length a
multiple of 512 bits and the result is processed with the MD5 digest
algorithm.  Let H be the digest value.
\item
Concatenate H with P.  Let Q be the resulting packet.
\item
Packet Q is padded, if necessary, to make its length a multiple of
R*(\ttt GF_Size!+1).  Let N be the number of padding bytes.  (Note
that the maximum number of padding bytes is 1024*4-1 = 4095)
\begin{enumerate}
  \item 
    If N=0, set the most significative bit of Q to 0.
  \item 
    If $128 > N > 0$, set the most significative bit of Q to 1 and set the
    $N$-th padding byte to $N$.
  \item
    If $N \geq 128$,  set the most significative bit of Q to 1, set the
    $N$-th padding byte to $128 + (N \bmod 128)$ and the $N-1$-th
    padding byte to $\floor{N / 128}$.
\end{enumerate}
In other words, the most significative bit of Q play a role similar to
one played by the \ttt Padding! bit.  Packet Q with the changes above
is the result of the preprocessing step.
\end{enumerate}
%
Packet Q produced by the preprocessing step is reduced by interpreting
its byte sequence as a matrix of elements in GF(2**\ttt GF_Size!+1)
stored columnwise and left-multiplying such matrix with the reduction
vector.  The reduction vector is a row vector r with R components. The
k-th component of r is equal to \ttt Reduction_Vector!**(k-1).  In
other words, if L=R*K is the number of \ttt GF_Size!+1-byte words
contained in Q, and $Q_n$, $n=0, \ldots, L-1$ denotes the $n$-th word
of Q, the reduced version of Q is the sequence of values
$$
u_k = Q_{Rk} + b Q_{Rk+1} +  \cdots
+ b^{R-1} Q_{R(k+1)-1},
\qquad k=0, \ldots, K-1
$$ 
%
where the operation are, of course, in GF(2**\ttt GF_Size!+1).  The
reduced version of Q is preceded by the header (the 32-bit fixed part,
plus the optional reduction details if necessary) and transmitted.

Each intermediate node SHOULD verify the correctness of the
reconstructed packet. A possible algorithm for checking the
correctness of the reconstructed packet is the following
%
\begin{enumerate}
  \item
    Remove any padding was added to packet Q
  \item
    Remove from Q the field H, that is,  the 128 most significative
    bits of Q  (which correspond to the digest of P)
  \item
    Pad P as described above and compute its MD5 digest and verify
    that it matchs H (taking in mind that the most significative bit
    of H was changed).
\end{enumerate}
%
If the node finds a mismatch, it MUST NOT reduce and retransmit the
packet.  If more than the strictly necessary packet have been
received, the node may try to use other packet combinations in order
to find a match.

\subsection{Basic check}
\label{subsub:5.1.1;transport_layer}

\notainterna{Solito trucco: prima ricostruisco e poi controllo la riduzione.}


\chapter{Control protocol}
\label{sect:7;transport_layer}

Section~\ref{sect:5;transport_layer} with profile specification in
Section~\ref{sect:6;transport_layer} completely specify the
epi-transport protocol.  However, this epi-transport protocol is
more complex than other transport protocols since it requires the
formation of a network of peers with, possibly, problems such peers
authentication.  Because of this, we propose a higher-level protocol
to control the connections between peers.

The proposed protocol is similar to RTSP \cite{rfc2326}.  The protocol
described here is just a draft to be refined.  The format of requests
and responses are identical to the ones of RTSP, with the only
difference for the the \ttt RTSP-Version! field replaced by
%
\begin{verbatim}
  PPMTCP-Version = "PPMTCP/1.0"
\end{verbatim}

\section{Method definition}
\label{sub:7.0;transport_layer}

This section collects the methods recognized by the control protocol.
If required, each command can include in the header section the \ttt
Authentication! header with the necessary credentials to grant that
the request is legitimate.

Similarly to RTSP, every command MUST carry the \ttt CSeq! header.
Moreover, if authentication is used, the command MUST carry the \ttt
ASeq! field. 

\subsection{CLOSE}
\label{subsub:7.0.2;transport_layer}

This command asks to the node to close an input port.  This command
could be necessary to ask the node to stop receiving data from a node
that was discovered to send bad packets.  With this command the node
avoids wasting resources on a useless input stream.  This method can
include the header \ttt Reopen! in order to make the node open a new
port after closing the old one, without the need of an explicity \ttt
OPEN! request.

\subsection{OPEN}
\label{subsub:7.0.3;transport_layer}

This command asks to the node to open a new input port.  In the reply
message the node must specify with the \ttt Port! header the number of
the opened port.  The URI field in the request line is ignored.

\subsection{SEND}
\label{subsub:7.0.0;transport_layer}

This command requires to the node to transmit a data stream to the
remote node specified in the URI field.  The URI field MUST specify a
port, since no default port can exist in this case.

If a specific data stream is required (maybe because the node has more
than one reduction vector), the stream number is specified with the
\ttt Stream! header.  If no \ttt Stream! header is present, the node
MUST use the first unused stream.

\subsection{STOP}
\label{subsub:7.0.1;transport_layer}

This command requires to the node to stop the transmission to the
remote node specified in the URI field.  The URI field MUST specify a
port, since no default port can exist in this case.


\section{Header field definitions}
\label{sub:7.1;transport_layer}

\subsection{Aseq}
\label{subsub:7.1.3;transport_layer}

The value of this field is an increasing number used for
authentication purposes, in order to avoid unauthorized replies of
already issued commands.  If authentication is used, field \ttt Aseq!
MUST be present and the node MUST ignore any command having an already
process \ttt Aseq! value.

\subsection{Authorization}
\label{subsub:7.1.0;transport_layer}

This header has the same format of the Authorization field in HTTP
\cite[Section~14.8]{rfc2616}.  The syntax of this header is

\begin{verbatim}
  Authorization = "Authorization" ":" credentials
  credentials   = auth-method 
                   ";" aseq-value 
                   ";" method 
                   ";" digest-value

  auth-method  = "Basic-Digest" | token

  method       = send-method |
                 stop-method |
                 open-method |
                 close-method

  send-method  = "SEND" "," ip-dest "," port ("," stream)?
  stop-method  = "STOP" "," ip-dest "," port ("," stream)?
  open-method  = "OPEN"
  close-method = "CLOSE" "," port
\end{verbatim}

Some remarks are in order

\begin{itemize}
  \item
    The values contained in the authorization header uniquely identify
    the required command.  The presence of \ttt Aseq! field grants
    that no two different legitimate commands will have the same \ttt
    Authorization! header.
  \item
    The \ttt stream! field in \ttt send-method! and  \ttt stop-method!
    MUST be present if and only if the \ttt Stream! header is present.
  \item
    The algorithm used to compute \ttt digest-value! is uniquely
    determined by the \ttt auth-method! field.  A basic authentication
    scheme, corresponding to the value \ttt Basic-Digest! for \ttt
    auth-method! is described in Section~\ref{sub:7.3;transport_layer}.
\end{itemize}

\subsection{Cseq}
\label{subsub:7.1.2;transport_layer}

See \cite[Section~12.17]{rfc2326}.

\subsection{Port}
\label{sub:7.2;transport_layer}

This header has syntax
%
\begin{verbatim}
  Port = "Port" ":" port-number 
\end{verbatim}
%
it is used in the reply to the \ttt OPEN! command or \ttt CLOSE!
command with the \ttt Reopen! field equal to 1.  The value \ttt
port-number! is the number of the port opened by the node.

\subsection{Reopen}
\label{subsub:7.1.4;transport_layer}

\begin{verbatim}
  reopen = "Reopen" ":" ("0" | "1")
\end{verbatim}
%
This header can be present only in a \ttt CLOSE! request.  If the
value is \ttt 1!, the node will re-open a new port after closing the
port specified in the \ttt CLOSE! command.  If this header is missing
or the value is \ttt 0!, no new port is open.

\subsection{Stream}
\label{subsub:7.1.1;transport_layer}

\begin{verbatim}
  stream = "Stream" ":" integer
\end{verbatim}
%
Optionally used in methods \ttt SEND! and \ttt STOP!  to specify the
stream number.

\section{Basic authentication protocol}
\label{sub:7.3;transport_layer}

The \ttt Basic-Digest! authentication method compute the \ttt
digest-value! field as follows

\begin{enumerate}
\item
A random seed is computed as follows
   \begin{enumerate}
   \item
     The node computes the \ttt aseq-value!-th value produced by a
     Blum-Blum-Shub generator \cite{blum-blum-shub}.  The two module
     factors and the initial value of the generator are communicated
     to the node by the server after node authentication (possibly
     over a secure channel).  The MD5 hash (converted in hexadecimal)
     of the generated value is the random seed.
   \end{enumerate}
\item
The random seed and the fields \ttt aseq-value! and \ttt method! are
concatenated, the resulting string is converted to lower case and
hashed with MD5. 
\item
The hash obtained at the previous step is concatenated with the random
seed and hashed againg with MD5.  
\item
The result of the final hash,
expressed in hexadecimal, is the value of the \ttt digest-value! field.
\end{enumerate}
%


\chapter{Extensions to RTSP}
\label{sect:8;transport_layer}

In this section we briefly describe few proposed extensions to RTSP in
order to adapt it to \ppmtp;

\section{Headers}

\subsection{Transport}
\label{subsub:0;transport_layer}

\begin{itemize}
  \item A new lower-transport protocol \ppmtp;-BASIC-UDP is added.
  \item The protocol \ppmtp;-BASIC-UDP accepts the following parameters
    \begin{description}
      \item[\ttt upload\_badwidth=integer!] Sent with the \ttt SETUP!
      request.  The value of this parameter is the upload bandwidth
      (in bit/sec) that the node is willing to spend.
      \item[\ttt md5\_signed=(0|1)!] Sent with the reply to \ttt
      SETUP!. If this parameters is present and it is \ttt 1!, the
      packets are signed with the MD5 approach described in
      Section~\ref{MD5-signing}.  
      \item[\ttt n\_input=integer!] Sent with the reply to \ttt SETUP!. \ttt
      n_input! is the number of input ports that the node must open.
      \item[\ttt n\_channel=integer!] Sent with the reply to \ttt SETUP!. \ttt
      n_channel! is the number of channels (i.e., of reduction
      vectors) that the node must prepare.
      \item[\ttt authentication=(none|token)!] Sent with the reply to \ttt
      SETUP!. This parameter MUST be present.  Its value identifies the
      authentication procedure used to create the \ttt Authorization!
      header.  The special value \ttt none! means that no
      authentication is used.
      \item[\ttt authentication-seed=<string without ';'>!]  If \ttt
      authentication! is not \ttt none! and the chosen authentication
      method requires a ``seed,'' this parameter MUST be present.  The
      actual syntax and meaning of the seed is fixed by the the
      document describing the authentication algorithm, with the
      constraint that the seed value cannot contain a '\ttt ;!'.
    \end{description}
\end{itemize}

\chapter{PPETP API}
\label{sect:9;transport_layer}

\input{ppetp-api}

\bibliographystyle{abbrv}
\bibliography{medusa_biblio}

\end{document}
