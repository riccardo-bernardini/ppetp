In this file I collect some old stuff that is maybe obsolete. There is
no LaTeX header because this file is not to be compiled.

\chapter{Ritagli: Control protocol (maybe obsolete)}
\label{sect:2;driver}

Chapter~\ref{sect:0;driver} specifies the epi-transport protocol
\ppetp-.  However, \ppetp- is more complex than other transport
protocols since it requires the formation of a network of peers with,
possibly, problems such peers authentication.  Because of this, we
propose a higher-level protocol to control the connections between
peers.

The proposed protocol is similar to RTSP \cite{rfc2326}, only with a
different set of methods and headers.  Here we simply describe the
differences between RTSP and \ppc- (Epi Transport Control Protocol).

\section{Request and responses}
\label{sect:2.0;driver}

Like in RTSP, nodes communicate through exchange of \emph{requests}
and \emph{responses}.  The format of requests and responses are
identical to the ones of RTSP, with the only difference for the the
\ttt RTSP-Version! field is replaced by
%
\begin{verbatim}
  PPMTCP-Version = "PPMTCP/1.0"
\end{verbatim}

\section{Methods}
\label{sect:2.1;driver}

This section collects the methods recognized by the control protocol.
If required, each command can include in the header section the \ttt
Authentication! header with the necessary credentials to grant that
the request is legitimate.

Similarly to RTSP, every command MUST carry the \ttt CSeq! header.
Moreover, if authentication is used, the command MUST carry the \ttt
ASeq! field. 

\subsection{CLOSE}
\label{subsub:7.0.2;transport_layer}

This command asks to the node to close an input port.  This command
could be necessary to ask the node to stop receiving data from a node
that was discovered to send bad packets.  With this command the node
avoids wasting resources on a useless input stream.  This method can
include the header \ttt Reopen! in order to make the node open a new
port after closing the old one, without the need of an explicity \ttt
OPEN! request.

\subsection{OPEN}
\label{subsub:7.0.3;transport_layer}

This command asks to the node to open a new input port.  In the reply
message the node must specify with the \ttt Port! header the number of
the opened port.  The URI field in the request line is ignored.

\subsection{SEND}
\label{subsub:7.0.0;transport_layer}

This command requires to the node to transmit a data stream to the
remote node specified in the URI field.  The URI field MUST specify a
port, since no default port can exist in this case.

If a specific data stream is required (maybe because the node has more
than one reduction vector), the stream number is specified with the
\ttt Stream! header.  If no \ttt Stream! header is present, the node
MUST use the first unused stream.

\subsection{STOP}
\label{subsub:7.0.1;transport_layer}

This command requires to the node to stop the transmission to the
remote node specified in the URI field.  The URI field MUST specify a
port, since no default port can exist in this case.

This section collects the methods recognized by the control protocol.
If required, each command can include in the header section the \ttt
Authentication! header with the necessary credentials to grant that
the request is legitimate.

Similarly to RTSP, every command MUST carry the \ttt CSeq! header.
Moreover, if authentication is used, the command MUST carry the \ttt
ASeq! field. 

\subsection{CLOSE}
\label{subsub:close;transport_layer}

This command asks to the node to close an input port.  This command
could be necessary to ask the node to stop receiving data from a node
that was discovered to send bad packets.  With this command the node
avoids wasting resources on a useless input stream.  This method can
include the header \ttt Reopen! in order to make the node open a new
port after closing the old one, without the need of an explicity \ttt
OPEN! request.

\subsection{OPEN}
\label{subsub:open;transport_layer}

This command asks to the node to open a new input port.  In the reply
message the node must specify with the \ttt Port! header the number of
the opened port.  The URI field in the request line is ignored.

\subsection{SEND}
\label{subsub:send;transport_layer}

This command requires to the node to transmit a data stream to the
remote node specified in the URI field.  The URI field MUST specify a
port, since no default port can exist in this case.

If a specific data stream is required (maybe because the node has more
than one reduction vector), the stream number is specified with the
\ttt Stream! header.  If no \ttt Stream! header is present, the node
MUST use the first unused stream.

\subsection{STOP}
\label{subsub:stop;transport_layer}

This command requires to the node to stop the transmission to the
remote node specified in the URI field.  The URI field MUST specify a
port, since no default port can exist in this case.


\section{Headers}
\label{sub:7.1;transport_layer}

\subsection{Aseq}
\label{subsub:7.1.3;transport_layer}

The value of this field is an increasing number used for
authentication purposes, in order to avoid unauthorized replies of
already issued commands.  If authentication is used, field \ttt Aseq!
MUST be present and the node MUST ignore any command having an already
process \ttt Aseq! value.

\subsection{Authorization}
\label{subsub:7.1.0;transport_layer}

This header has the same format of the Authorization field in HTTP
\cite[Section~14.8]{rfc2616}.  The syntax of this header is

\begin{verbatim}
  Authorization = "Authorization" ":" credentials
  credentials   = auth-method 
                   ";" aseq-value 
                   ";" method 
                   ";" digest-value

  auth-method  = "Basic-Digest" | token

  method       = send-method |
                 stop-method |
                 open-method |
                 close-method

  send-method  = "SEND" "," ip-dest "," port ("," stream)?
  stop-method  = "STOP" "," ip-dest "," port ("," stream)?
  open-method  = "OPEN"
  close-method = "CLOSE" "," port
\end{verbatim}

Some remarks are in order

\begin{itemize}
  \item
    The values contained in the authorization header uniquely identify
    the required command.  The presence of \ttt Aseq! field grants
    that no two different legitimate commands will have the same \ttt
    Authorization! header.
  \item
    The \ttt stream! field in \ttt send-method! and  \ttt stop-method!
    MUST be present if and only if the \ttt Stream! header is present.
  \item
    The algorithm used to compute \ttt digest-value! is uniquely
    determined by the \ttt auth-method! field.  A basic authentication
    scheme, corresponding to the value \ttt Basic-Digest! for \ttt
    auth-method! is described in Section~\ref{sub:7.3;transport_layer}.
\end{itemize}

\subsection{Cseq}
\label{subsub:7.1.2;transport_layer}

See \cite[Section~12.17]{rfc2326}.

\subsection{Port}
\label{sub:7.2;transport_layer}

This header has syntax
%
\begin{verbatim}
  Port = "Port" ":" port-number 
\end{verbatim}
%
it is used in the reply to the \ttt OPEN! command or \ttt CLOSE!
command with the \ttt Reopen! field equal to 1.  The value \ttt
port-number! is the number of the port opened by the node.

\subsection{Reopen}
\label{subsub:7.1.4;transport_layer}

\begin{verbatim}
  reopen = "Reopen" ":" ("0" | "1")
\end{verbatim}
%
This header can be present only in a \ttt CLOSE! request.  If the
value is \ttt 1!, the node will re-open a new port after closing the
port specified in the \ttt CLOSE! command.  If this header is missing
or the value is \ttt 0!, no new port is open.

\subsection{Stream}
\label{subsub:7.1.1;transport_layer}

\begin{verbatim}
  stream = "Stream" ":" integer
\end{verbatim}
%
Optionally used in methods \ttt SEND! and \ttt STOP!  to specify the
stream number.

\section{Basic authentication protocol}
\label{sub:7.3;transport_layer}

The \ttt Basic-Digest! authentication method compute the \ttt
digest-value! field as follows

\begin{enumerate}
\item
A random seed is computed as follows
   \begin{enumerate}
   \item
     The node computes the \ttt aseq-value!-th value produced by a
     Blum-Blum-Shub generator \cite{blum-blum-shub}.  The two module
     factors and the initial value of the generator are communicated
     to the node by the server after node authentication (possibly
     over a secure channel).  The MD5 hash (converted in hexadecimal)
     of the generated value is the random seed.
   \end{enumerate}
\item
The random seed and the fields \ttt aseq-value! and \ttt method! are
concatenated, the resulting string is converted to lower case and
hashed with MD5. 
\item
The hash obtained at the previous step is concatenated with the random
seed and hashed againg with MD5.  
\item
The result of the final hash,
expressed in hexadecimal, is the value of the \ttt digest-value! field.
\end{enumerate}
%

\section{Altra roba}
\label{sect:2.4;transport_layer}

\begin{itemize}
  \item The protocol \ppmtp;-BASIC-UDP accepts the following parameters
    \begin{description}
      \item[\ttt upload\_badwidth=integer!] Sent with the \ttt SETUP!
      request.  The value of this parameter is the upload bandwidth
      (in bit/sec) that the node is willing to spend.
      \item[\ttt n\_input=integer!] Sent with the reply to \ttt SETUP!. \ttt
      n_input! is the number of input ports that the node must open.
      \item[\ttt n\_channel=integer!] Sent with the reply to \ttt SETUP!. \ttt
      n_channel! is the number of channels (i.e., of reduction
      vectors) that the node must prepare.
      \item[\ttt authentication=(none|token)!] Sent with the reply to \ttt
      SETUP!. This parameter MUST be present.  Its value identifies the
      authentication procedure used to create the \ttt Authorization!
      header.  The special value \ttt none! means that no
      authentication is used.
      \item[\ttt authentication-seed=<string without ';'>!]  If \ttt
      authentication! is not \ttt none! and the chosen authentication
      method requires a ``seed,'' this parameter MUST be present.  The
      actual syntax and meaning of the seed is fixed by the the
      document describing the authentication algorithm, with the
      constraint that the seed value cannot contain a '\ttt ;!'.
    \end{description}
\end{itemize}

