%-*- mode: latex; iso-accents-mode: nil;  ispell-local-dictionary: "english" -*-
\iffalse
\documentclass{rfc}
\else
\documentclass[a4paper,10pt]{medusabook}
\newcommand{\fullname}[1]{#1}
%\newcommand{\organization}[1]{\\#1}
%\newcommand{\address}[1]{}
\fi
\newcommand{\ttbf}[1]{\texttt{\bfseries #1}}
%\documentclass[a4paper,11pt]{article}
%
% text width/height/offset
%
%\textwidth=16.3cm         %%% valore "vero" = 16cm
%\textheight=24.6cm      %%% valore "vero" = 24.5
%\hoffset=-1.64cm
%\voffset=-1in
%%\renewcommand{\baselinestretch}{1.62}
%\renewcommand{\baselinestretch}{1.48} %% Valore ``vero'' per 2interl.= {1.55}
%
% Standard packages...
%
\def\ttt{\catcode`\_=12 \tttii}
\def\tttii#1!{{\tt #1}\catcode`\_=8{}}

\usepackage{amsmath}
\usepackage{amsbsy}
\usepackage{amsthm,amsfonts}
\usepackage{epsf,epsfig}
\usepackage{times}
\usepackage{srcltx}
\usepackage[a4paper,margin=1.2in]{geometry}
\voffset=-0.5in
%
%
%
%%\newtheoremstyle{corto}{no}{n}{\bfseries}{\bfseries}{.}
%%   {\itshape}
%%   {0.25\baselineskip}
%%   {0.25\baselineskip}
%%\theoremstyle{corto}
%% \addtolength{\theorempreskipamount}{-15mm}
%% \addtolength{\theorempostskipamount}{-15mm}
%
%
%
%
% Standard definitions...
%
../../doc/macrodef.tex
%\setlength{\nproofskip}{0pt}
%
% Local definitions
%
%
% Title informations
%
\title{Peer-to-Peer Epi-Transport Protocol (\ppmtp;)}

\author{
\fullname{Riccardo Bernardini}
\organization{DIEGM -- University of Udine}
\address{street=Via delle Scienze, 208;city=Udine;code=33100;country=IT;email=bernardini@uniud.it}}

%
% True document
%
\begin{document}
\maketitle

\tableofcontents

\chapter{Introduction}
\label{sect:intro;driver}

Due parolae (alla latina) di descrizione (molto) informale in
italiano\ldots

\section{Perch\'e \ppetp-?}
In principio c'era \medusa-, un sistema ``completo'' per la
trasmissione/ricezione in streaming di video live su reti P2P.  Poi
scoprii l'RTSP (Real Time Streaming Protocol) e SDP (Session
Description Protocol) e cominciai una revisione del protocollo
server/client di \medusa- per avvicinarlo a questi due protocolli
gi\`a pronti.  Cos\`\i{} facendo mi resi conto che se l'idea di
\medusa- fosse stata fatta scendere a livello di trasporto, sarebbe
stato possible integrare facilmente \medusa-  con gli strumenti
attualmente disponibili, semplicemente aggiungendo al software
esistente un modulo in grado di interfacciarsi col nuovo protocollo di
trasporto.  

\section{Perch\'e ``epi''?} L'idea \`e che \ppetp- dovrebbe essere
visto dal livello applicativo come una specie di protocollo di
trasporto, anche se non \`e un vero livello di transporto, poich\'e si
appoggia a sua volta su protocolli di trasporto di livello pi\`u
basso.  Ecco quindi l'uso del prefisso \emph{epi} (sopra).


\section{Cosa trovo in questo documento?}
In questo papero descrivo una prima bozza del \ppetp-.  La descrizione
\`e divisa in diverse sotto-descrizioni

\begin{description}
\item[Protocollo di trasporto] La descrizione del protocollo di
  trasporto vero e   proprio viene divisa in due sotto-descrizioni
\begin{enumerate}
  \item
    La prima sotto-descrizione specifica il formato dei pacchetti
    scambiati tra i nodi, ma  non come avviene la ``riduzione'' dei
    pacchetti da trasmettere.  
  \item
    La seconda sotto-descrizione descrive come 
    ridurre e ricostruire i pacchetti.
\end{enumerate}
Tale descrizione in due tempi permette di usare diversi algoritmi di
riduzione con la stesso protocollo di epi-trasporto.  Si osservi che
questa tecnica di ``descrizione in due tempi'' \`e simile a quella
usata per l'RTP \cite{rfc3550} (che ha bisogno di essere completato
dalla descrizione in un \emph{profilo} specifico) e il DCCP
\cite{rfc4336} (che ha bisogno di essere completato dalla descrizione
dell'algoritmo usato per il controllo del rate). 
Per analogia con quanto fatto nell'RTP, chiameremo la descrizione
degli algoritmi di riduzione \emph{profilo}.

\`E importante osservare che non \`e necessario che il tipo di
elaborazione introdotta dal profilo sia di tipo ``riduttivo,'' ma
pu\`o essere di qualsiasi tipo.  Per esempio, un profilo diverso da
quello descritto qui potrebbe trasmettere sulla rete combinazioni
lineari di pacchetto, nello spirito del  \emph{network coding}.

\item[Protocollo di controllo] La descrizione del protocollo di
trasporto non specifica come i nodi della rete P2P vengano in contatto
tra loro, poich\'e il problema della formazione della rete appartiene
ad un livello che non \`e quello di trasporto (in maniera analoga, le
specifiche di UDP non spiegano come un nodo venga conoscenza dell'IP
di un altro nodo).

A completamento della descrizione del protocollo di trasporto,
introduciamo un protocollo di controllo del collegamento di nodi
appartenenti alla stessa rete P2P.  Tale protocollo \`e simile
all'HTTP, solo con un diverso insieme di metodi e intestazioni.  Lo
scopo del protocollo di trasporto \`e di permettere ai nodi di venire
in contatto, di autenticarsi, di negoziare eventuali parametri di
connessione e cos\`\i{} via.
\item[Estensioni a RTSP] A scopo dimostrativo descriviamo brevemente
  come si pu\`o estendere il valore dell'intestazione \ttt Transport!
  di RTSP per includere \ppetp- come possibile opzione per il
  trasporto.
\item[Proposta di API]  Una proposta per una possibile API per
  l'accesso a \ppetp- da programma.
\end{description}
%
\section{Esempio di sessione}

Un breve esempio di come si svolgerebbe una sessione di streaming
video su \ppetp-

\begin{enumerate}
\item
Un utente che vuole accedere ad un certo contenuto video fa partire il
suo player preferito e gli comunica l'URL del video.
\item
Il player contatta il server ed inizia una fase di colloquio.
\item
Per prima cosa (in risposta ad un \ttt DESCRIBE! RTSP o tramite una
risposta HTTP) il server manda la descrizione in formato (diciamo) SDP
con le istruzioni per accedere al video.  Se un filmato deve essere
trasmesso su PPETP, l'attributo SDP \ttt m=! specificher\`a il
protocollo di trasporto \ttt RTP/PPETP! (con porta 0).
\item
In risposta alla descrizione SDP, il client apre una sessione (o due
adiacenti) PPETP contatta il server con \ttt SETUP! specificando
nell'attributo \ttt client_port! dell'header \ttt Transport! il numero
della/e porta/e PPETP aperte.
\item
Il server, dopo un'eventuale autenticazione, accetta il \ttt SETUP!.
\item
Il client, in seguito all'accettazione, risponde con un \ttt PLAY!  A
questo punto possono darsi due scenari differenti
\begin{description}
  \item[Il server fa tutto] In risposta al metodo \ttt PLAY! il server
  contatta i peer che dovranno servire il nuovo nodo e tramite il
  protocollo di controllo comunica loro la richiesta di trasmettere un
  sotto-stream al nuovo arrivato.
  \item[Il client si arrangia] In risposta al metodo \ttt PLAY! il
  server risponde con una lista di indirizzi di peer a cui il nodo
  deve rivolgersi.  Il nodo contatta i peer suggeriti dal server e
  tramite il protocollo di controllo instaura una comunicazione.
\end{description}
Qualunque sia lo scenario, il server comunica nel corpo della risposta
al \ttt PLAY! le seguenti informazioni
\begin{itemize}
  \item Il numero di canali da aprire, quali profili usare ed
  eventuali parametri dei profili.
  \item Eventuali chiavi per l'autenticazione dei comandi \ttt
  Send_Data! e \ttt Receive_Data!
\end{itemize}
\item
A questo punto gli strati \ppetp- dei nodi appartenenti alla rete P2P
comunicano tra loro.  Gli applicativi che si appoggiano su \ppetp-
leggerano i pacchetti in arrivo facendo semplicemente delle \ttt recv!
sui ``socket'' \ppetp-.  Da questo momento in poi, dal punto di vista
dell'applicazione, \ppetp- diventa praticamente indistinguibile da
qualsiasi altro protocollo di trasporto non affidabile (come l'UDP).
\end{enumerate}
%
In \fref{cpptp} \`e possibile vedere uno schema di una possibile
applicazione basata su \ppetp-.

\begin{figure}
\centerline{\unafigura{cpptp}}
\caption{Esempio di applicazione basata su \ppetp-.
\label{fig:cpptp}}
\end{figure}
%

\section{Perch\'e \`e interessante?}
\label{sub:intro.0;driver}

Il motivo per cui potrebbe essere interessante portare \medusa- a
livello di trasporto \`e che in questo modo qualunque applicazione e
qualunque tecnica di codifica potrebbe essere portata su P2P,
semplicemente aggiungendo il codice necessario ad interfacciarsi con
\ppetp-. Schemi a descrizione multipla, scalabili, con FEC ed
eventuali loro combinazioni convesse implementati in programmi in
grado di comunicare su UDP potrebbero essere portati su P2P con un
semplice ``cambio di libreria.'' 

\section{Appunti}
\label{sect:intro.0;transport_layer}

Qualche idea in ordine sparso\ldots{} Non so dove scriverle e me le
appunto qui per non dimenticarmente

\begin{itemize}
\item
Una sessione PPETP \`e individuata dalla sua \emph{porta \ppetp-} 
che coincide con la porta UDP di controllo.  Anche l'applicazione
semplicemente ``apre una porta \ppetp-.''
\begin{commento}[Appunto vecchio e forse non pi\`u valido]
Vale la seguente equazione
%
\begin{equation}
\label{eq:0;transport_layer}
\text{porta PPETP} = (2^{16}/8) \cdot \text{canale} + (\text{porta UDP}-1024)
\end{equation}
%
Ogni nodo ha quindi al massimo 8 canali di uscita e le porte UDP
utilizzabili vanno da 1024 a $1024 + 2^{16}/8 = 9216$.
\end{commento}
%
\item
Ad ogni canale \`e assegnato uno ed uno solo profilo di elaborazione
con il suo insieme di parametri.  Tipicamente il server dir\`a al nodo
quanti canali aprire, quali profili usare e il valore di (alcuni)
parametri. 
\item
A livello di applicazione deve essere possibile chiedere di aprire il
canale con un certo numero ed assegnarli profilo e parametri.  Devo
quindi rivedere l'API.
\item
Sarebbe possibile, con qualche mastruzzo del terz'ordine, creare
un'API C che mappa le porte \ppetp- in socket? (Forse s\`\i{},
giocando sporco con le macro\ldots{} ma vorrei evitare\ldots)
\end{itemize}
%

\chapter{Transport protocol}
\label{sect:0;driver}

\section{Introduction}
\label{subsub:0.0.1;driver}

\paragraph{The \ppetp- model} \ppetp- supposes that the P2P network is
made of several nodes which communicate with each other.  The nodes in
a \ppetp- network are interested in the same type of \emph{content}
which we suppose was split in packets from a \emph{root server}.  The
packets transmitted from peer to peer do not necessarily coincide with
the packets which make the original content.  For example, packets can
be \emph{reduced} with the approach described in
\cite{bernardini08:dcc08}, in order to allow each node to use an
upload bandwidth smaller than the download bandwidth.  Alternatively,
packets can be processed by network coding techniques.  It is
reasonable to assume that the processing goal will be typically an
increased reliability or transmission efficiency.

In a \ppetp- network each node receives one or more input stream from
the other nodes, recovers the packets and retransmit them to the other
nodes after having processed them.  Note that, depending on the chosen
processing procedure, one could need to specify some \emph{processing
  parameters} in order to uniquely identify the processing carried out
by a node.  For example, the processing described in
\cite{bernardini08:dcc08} requires the specification of a Galois field
$\gf{2^n}$, an element $b\in \gf{2^n}$ and a positive integer $R$
called in \cite{bernardini08:dcc08} \emph{reduction factor}.  The
\ppetp- does not describe how the processing parameters are chosen by
each node.

The \ppetp- description is split into two parts

\begin{description}
\item[Pure Transport Level (PTL)] This part describes how data are exchanged
  between nodes (e.g., handshaking procedures and packet formats), but
  not how the packet processing is done.  The description of packet
  processing procedures is demanded to companion documents describing
  \ppetp- \emph{profiles}
\item[Profile description (PD)] This part describes how packet are
  processed and how the processed data is embedded in the PTL
  packets. Note that while there is only one PTL specification , it is
  possible to devise that in a future several PD will be available
  (e.g. a PD based on the approach of \cite{bernardini08:dcc08} and a
  PD based on a network coding approach).
\end{description}
%
Note that such two-step description allows one to adopt new processing
procedure without changing the basic transport level.  A similar
two-step approach is used also in the description of the Real-time
Transport Protocol (RTP) \cite{rfc3550} and of the Data Congestion
Control Protocol (DCCP) \cite{rfc4336}.

\section{Underneath transport protocol}
\label{subsub:0.0.0;driver}

\ppetp- is not a transport protocol by itself and it needs a true
transport protocol to transmit packets between peers.  It is not
necessary that the transport protocol used by \ppmtp; grants for the
delivery of every packet nor for the preservation of transmission
order.  For example, UDP and DCCP could be used as a transport
protocol for \ppmtp;.

\section{Type of packets}
\label{sub:0.2;driver}


The packets exchanged by \ppetp- nodes can be classified as \emph{data
packet} or \emph{control packet}.  Data packets are the most common
ones, they carry the processed version of the original content packets
and require no acknowledge.  Control packets are typically used during
a session setup and they require an acknowledge reply.

Each packet (be it of data or control type) carry a 20-bit \emph{sequence
number} or \emph{timestamp}.  The sequence number \emph{together with
the packet type and the source and destination peers} uniquely
identify the packet.  Note that sequence numbers are \emph{not} unique
on a \emph{global} level and that peer A can send to peer B a control
packet and a data packet, both with the same sequence number.  The
reason for allowing a data and a control packet sharing the same
sequence number is that in the multimedia streaming context one can
expect the timestamp associated with a given data packet to be related
with the temporal information associated to the corresponding
multimedia packet.  If control and data packet shared the same
sequence number space, the transmission of a control packet would
destroy the correspondence between sequence number and temporal
information.   

Sequence number are strictly sequential and, similarly to the
requirements in the RTP specification \cite{rfc3550}, it is suggested
the initial value of this field SHOULD be random (unpredictable) to
make known-plaintext attacks on encryption more difficult.

\subsection{Data packet format}
\label{subsub:0.0.2;driver}

The data packets have the following format

\begin{itemize}
  \item
    A fixed 32-bit header with the following structure

    \begin{verbatim}
      type Unsigned_2  is mod 2**2;   --  2-bit unsigned int
      type Unsigned_3  is mod 2**3;   --  3-bit unsigned int
      type Unsigned_4  is mod 2**4;   --  4-bit unsigned int
      type Unsigned_20 is mod 2**20;  -- 20-bit unsigned int

      type Data_Header is
         record
            Version   : Unsigned_2;
            Control   : Boolean;       -- False for data packets
            Inline    : Boolean;
            Padding   : Boolean;
      
            Marker    : Unsigned_3;
      
            Profile   : Unsigned_4;
            Timestamp : Unsigned_20;
         end record;


       -- Structure of a packet exchanged between peers
       for Data_Header'Bit_Order use High_Order_First;
       for Data_Header use
          record
             Version    at 0 range 0..1;
             Control    at 0 range 2..2;
             Inline     at 0 range 3..3;
             Padding    at 0 range 4..4;
             Marker     at 0 range 5..7;

             Profile    at 1 range 0..3;
             Timestamp  at 1 range 4..23;
          end record;
    \end{verbatim}
\notainterna{La descrizione Ada-like \`e vecchia, i disegni sono aggiornati}
  \item
    An optional header which depends on the chosen profile. This part
    is present only if the \ttt Inline! bit is set.
  \item
    The data payload, possibly padded (if the \ttt Padding! bit is set)
\end{itemize}

A pictorial representation of the data header is given in \fref{data-header}.

\begin{figure}
\begin{center}
\begin{verbatim}
 0                   1                   2                   3
 0 1 2 3 4 5 6 7 8 9 0 1 2 3 4 5 6 7 8 9 0 1 2 3 4 5 6 7 8 9 0 1
+-+-+-+-+-+-+-+-+-+-+-+-+-+-+-+-+-+-+-+-+-+-+-+-+-+-+-+-+-+-+-+-+
|V=1|C|I|P|Flags|Channel|            Sequence number            |
+-+-+-+-+-+-+-+-+-+-+-+-+-+-+-+-+-+-+-+-+-+-+-+-+-+-+-+-+-+-+-+-+
|                          Source ID                            |
+-+-+-+-+-+-+-+-+-+-+-+-+-+-+-+-+-+-+-+-+-+-+-+-+-+-+-+-+-+-+-+-+
:                           Payload                             :
+-+-+-+-+-+-+-+-+-+-+-+-+-+-+-+-+-+-+-+-+-+-+-+-+-+-+-+-+-+-+-+-+
:                                                               :
:                Signature (0, 32, 64 or 128 bit)               :
:                                                               :
+=+=+=+=+=+=+=+=+=+=+=+=+=+=+=+=+=+=+=+=+=+=+=+=+=+=+=+=+=+=+=+=+
\end{verbatim}
\end{center}
\caption{\ppetp- data header \label{fig:data-header}}
\end{figure}


The header fields have the following meaning

\begin{description}
\label{description:4.0.0;transport_layer}
\item[Version (V): 2 bit]
This field identify the protocol version.  This document
corresponds to \ttt V=01!.
\item[Control (C): 1 bit] This bit is always 0 for data packets.
\item[Inline  (I): 1 bit]
If this bit is 1, details about the reduction process are specified in
the profile-specific header which follows the first 32-bit word.  
\item[Padding (P): 1 bit] Similarly to the RTP specification
\cite{rfc3550}, if this bit is set, the packet contains one or more
additional padding octets at the end which are not part of the
payload.  The last octet of the padding contains a count of how many
padding octets should be ignored, including itself.
\item[Flags (F): 3 bit] Similarly to the RTP
specification \cite{rfc3550}, the interpretation of this field is defined by a
profile.
\item[Channel : 4 bit] The channel number.
\item[Sequence number : 20 bit] A sequence number which increments for every
  packet sent by the root source. Packets with the same timestamp
  field MUST derive from the same root packet.\\
  By calling $B_1$, $B_2$ and $B_3$ the octects corresponding to bits
  8--15, 16--23 and 24--31, respectively, the sequence number value
  can be computed as
$$
   B_1 \wedge \text{\tt 16\#0F\#} + 
   2^8 B_2 + 2^{16} B_3
$$ 
%
where $\wedge$ is the bitwise AND.  The most significant bit of every
octet is the one with the smallest index (that is, the most
significant bit of $B_1$, $B_2$ and $B_3$ are, respectively, bit 8,
bit 16 and bit 24).

Similarly to the requirements in the RTP specification \cite{rfc3550},
the initial value of this field SHOULD be random (unpredictable) to
make known-plaintext attacks on encryption more difficult
\item[Source ID] A single \ppetp- session can carry packets
  originating from different sources.  Each source has its own ID that
  is a 32-bit number.  The ID could be randomly generated by the host
  or it could be set by external means (e.g., assigned by a
  centralized server).
\item[Payload] An opaque sequence of octects whose meaning depends on
  the profile employed.  Note that the payload section can be empty.
  An empty payload can be useful to set the default reduction
  parameters without sending any data (for example, when the source
  peer makes first contact with the destination peer).
%
\item[Signature] It can be 0, 32, 64 or 128 bits long (depending on
the security and bandwidth requirements) and it is computed as
follows: the packet is hashed with MD5 (to obtain a 128-bit string)
and the result is encrypted with AES using a key agreed between the
nodes at the handshaking time.  The first 1, 2 or 4 quadwords (32-bit
words) of the encrypted hash are inserted in the \texttt{Signature}
field.  The size of this field is agreed by out-of-band means.
\end{description}
%
The data packet structure makes its parsing quite easy: the first 64
bits are a fixed header, the last 0-4 quadwords are used to
authenticate the packet and everything in between is handed to the
profile procedures.

\subsection{Control packet format}
\label{subsub:0.0.3;driver}

The control packets have the following format

\begin{itemize}
\item
Control packets, similarly to data packet, begin with a 32-bit header.
The control packet header has the 
following structure

\begin{verbatim}
   type Control_Header is
    record
       Version   : Version_Type;
       Command   : Boolean;       -- True on Control packets
       Request   : Unsigned_5;

       Flags     : Unsigned_4;
       Timestamp : Unsigned_20;
    end record;


 for Control_Header'Bit_Order use System.High_Order_First;
 for Control_Header use
    record
       Version     at 0 range 0..1;
       Command     at 0 range 2..2;
       Request     at 0 range 3..7;

       Flags       at 1 range 0..3;
       Timestamp   at 1 range 4..23;
    end record;
\end{verbatim}
\notainterna{La descrizione Ada-like \`e vecchia, i disegni sono aggiornati}
The meaning of field \ttt Flags! depends on the specific \ttt
Request!. 
\item
The packet payload whose content depends on the requested action.  The
payload part can be empty.
\end{itemize}

The header fields have the following meaning

\begin{description}
\item[Version (V): 2 bit] Same as data packets
\item[Control (C): 1 bit] This bit is always 1 for control packets.
\item[Request : 5 bit] The actual request.  Profiles can define
  request values (strictly) greater than 16 and (strictly) lower than
  31. Other request values are reserved.
\item[Flags : 4 bit] Its meaning depends on the value of \ttt
  Request!. If a request does not use this field, this field SHOULD be
  equal to 0.
\item[Sequence Number : 20 bit] The packet sequence number.  The same
  remarks of data packets apply also in this case.
\item[Signature] Same as data packets
\item[Payload] Its meaning depends on the specific request
\end{description}

A pictorial representation of a control packet is given
\fref{control-header}. 

\begin{figure}
\begin{center}
\begin{verbatim}
 0                   1                   2                   3
 0 1 2 3 4 5 6 7 8 9 0 1 2 3 4 5 6 7 8 9 0 1 2 3 4 5 6 7 8 9 0 1
+-+-+-+-+-+-+-+-+-+-+-+-+-+-+-+-+-+-+-+-+-+-+-+-+-+-+-+-+-+-+-+-+
|V=1|C| Request | Flags |         Sequence number               |
+-+-+-+-+-+-+-+-+-+-+-+-+-+-+-+-+-+-+-+-+-+-+-+-+-+-+-+-+-+-+-+-+
:                      Optional Payload                         :
+-+-+-+-+-+-+-+-+-+-+-+-+-+-+-+-+-+-+-+-+-+-+-+-+-+-+-+-+-+-+-+-+
:                                                               :
:                Signature (0, 32, 64 or 128 bit)               :
:                                                               :
+-+-+-+-+-+-+-+-+-+-+-+-+-+-+-+-+-+-+-+-+-+-+-+-+-+-+-+-+-+-+-+-+
\end{verbatim}
\end{center}
\caption{\ppetp- control packet \label{fig:control-header}}
\end{figure}

\begin{table}
  \caption{Obfuscation mask.  Observe that \ttt Mask(n)!$=197 \cdot
  21^n \bmod 256$. \label{tbl:obfuscated}} 
\begin{center}
\small
  \begin{tabular}{|c|*{16}r|}
\hline
\ttt n! &
0&  1&  2&  3&  4&  5&  6&  7&  8&  9&  10&  11&  12&  13&  14&  15\\
\hline
\ttt Mask(n)! &
    197&  41&  93& 161&
    53&  89&  77&  81&
    165& 137&  61&   1&
    21& 185&  45& 177\\
\hline
  \end{tabular}
\end{center}
\end{table}
%
This document defines the followings values for the \ttt Request!
field 

\begin{description}
\item[\ttt Hello! (\ttt Request=0!)] This request is used to
  communicate to the remote peer the profile and the port used by
  the node to receive commands and acknowledges.  The payload is a
  32-bit integer which coincides with the port number.  This control
  does not use the \ttt Flags! field.
\item[\ttt Set\_Default! (\ttt Request=1!)] 
  It is expected that the processing procedure specified by the
  profile will require some parameter in order to uniquely identify
  the processing carried out by the node.   Since it is reasonable to
  assume that such parameters will be chosen at start-up time and
  never changed, it is convenient to introduce this control packet
  which allows a node to communicate to its target
  peers its parameters just at the beginning.  Parameters can be
  temporally changed by using the \ttt Inline! field in data packets.
  The payload format of this request depends on the profile. The \ttt
  Profile! is stored in the \ttt Flags! field.
\item[\ttt Acknowledge! (\ttt Request=2!)] This type of control packet is
  used to acknowledge the receipt of other control packets. The
  payload is a 32-bit number which contails the sequence number of the
  acknowledged packet.  \textbf{The meaning of \texttt{Flags} depends
  on the command 
  acknowledged, but the zero value has always the meaning of
  ``positive acknowledge'' (i.e., no error occurred).}
\item[\texttt{Data\_Control} (\texttt{Request=3})] This command is used to
  request a node to send data to another node, to stop the data
  transmission to another node or to start the hole punching
  procedure.  The payload (see Fig.~\ref{fig:send_payload}) contains the
  following fields
     \begin{itemize}
       \item Bit 0 (flag \ttbf{V}) is set to 1 if the \texttt{IP\_address}
       field stores an IPV6 address.
       \item Bit 1 and 2 (\ttbf{SC} field) specify the actual
       ``sub-command.''  It is equal to 0 for a ``\emph{start}'', to 1
       for a ``\emph{stop}'' and to 2 for a ``\emph{punch}.''  Value 3
       is reserved.
       \item Bits from 3 to 6 are the \ttbf{Channel} field and
       represent the output channel number of the source peer to be
       used. 
       \item Bits from 7 to 15 are reserved.  They SHOULD be set to
       zero by the transmitter and SHOULD be ignored by the receiver.
       \item Bits from 16 to 31 are the \ttbf{Port} field and
       represent the PPETP port of   the destination node
       \item Field \ttbf{IP Address} can be 32 or 128 bits long and 
       contains the IP address of the destination node.  If flag \texttt{
       V} is 0, this field contains a 32 bits long IPv4 address; if
       flag \texttt{V} is 1, this field contains a 128 bits long IPv6
       address.
       \item The last field (\ttbf{Diffie-Hellman public key}) has
       variable size, it is present only with the \emph{start}
       sub-command and it contains the public key of the new lower peer.
     \end{itemize}
  Note that the \texttt{Data\_Control} command is \textbf{not} signed by
  the lower peer, but by the server.  If the signature is correct, the
  node replies with an \texttt{Acknowledge} packet with \texttt{Flags}
  equal to zero and carries out the requested action; otherwise, it
  replies with a packet with \texttt{Flags} equal to 1.

% \item[\ttt Negative\_Acknowledge! (\ttt Request=3!)] Used to ``answer
%   no'' to a request.  For example, in response to a request \ttt
%   Send_Data! with invalid authentication credentials.
% This control does not require an acknowledge.  Currently the \ttt
%   Flags! field is not used, but in a future could be used to
%   distinguish between different reasons for a NACK. 
% \item[\ttt Send\_Data! (\ttt Request=4!)] This command is used to
%   request a node to send data to another node.  The payload (see
%   \fref{send_payload}) contains the following fields
%      \begin{description}
%        \item[Channel (4 bits)] The channel to be used.
%        \item[Flag O (obfuscated)] It seems that some NATs try to
%        change anything which looks like an IP address.  If this flag
%        is set, the \ttt IP_address! field is obfuscated by \ttt
%        XOR!-ing it with the values in Table~\ref{tbl:obfuscated}.
%        \item[Flag V (IPV6)] Set to 1 if the \ttt IP_address! field
%        stores an IPV6 address.
%        \item[Port (16 bits)] The PPETP port where to send the
%        data. 
%        \item[IP Address (32 or 128 bits)] If flag \ttt V! is 0, this
%        field is 32 bits long and it holds the IPv4 address of the
%        target. If flag \ttt V! is 1, this field is 128 bits long and
%        it holds the IPv6 address of the target.
%        \item[Credential length (16 bits)] This is the length of the
%        credential field (it can be 0)
%        \item[Credential profile (8 bits)]
%        \item[Credential payload] Its size and format depend on the
%        profile. 
%      \end{description}
%   If the authentication credentials are
%   correct, the node replies with an \ttt Acknowledge! packet and
%   begins sending data to the required address; otherwise, it replies
%   with a \ttt Negative_Acknowledge!.  Note that it is not necessary
%   that the sender address be equal to the destination address included
%   in the request.  This control does not use the \ttt
%   Flags! field.
% \item[\ttt Receive\_Data! (\ttt Request=5!)] This packet is used to
%   ask a node to prepare itself for data reception.  If the
%   authentication credentials are not correct, the node replies with a
%   \ttt Negative_Acknowledge!, otherwise it sends a \ttt Send\_Data!
%   command to the remote node.  The payload for this command is shown
%   in \fref{receive_payload} and it is similar 
%   to the payload for the \ttt Send_Data! command, with the difference
%   that now there are \emph{two} possibile authentication schemes: the
%   first one are the credentials for the \ttt Receive\_Data! packet and
%   the second one are to be copied verbatim to the \ttt Send\_Data!
%   request.   Fields \ttt Port! and \ttt IP address! correspond to the
%   control port of the remote node. This control does not use the \ttt
%   Flags! field.
% \item[\ttt Stop! (\ttt Request=6!)] This packet closes the data
%   transmission toward the target described in the payload.  The
%   payload for this command is equal to the payload of \ttt
%   Send_Data!. 
\item[\ttt Nul! (\ttt Request=7!)] This packet is a no-op. It does not
  use the \ttt Flags! field.  Any byte following the basic header is
  considered a payload. It is used in hole-punching
  procedures.   
% \item[\ttt Punch! (\ttt Request=8!)]  This command starts a
%   hole-punching procedure towards another node.  The payload is the
%   same of the \ttt Send_Data! command.  The \ttt CH! field SHOULD be
%   zero. 
\end{description}
%
In general, when a node transmits a control packet, it waits for an
acknowledge to return.  If the acknowledge does not arrive after a
suitable amount of time, the repeats the request \emph{using the same
  sequence number}.

\begin{figure}
  \begin{center}
    \begin{verbatim}
Data_Control payload:
 0                   1                   2                   3
 0 1 2 3 4 5 6 7 8 9 0 1 2 3 4 5 6 7 8 9 0 1 2 3 4 5 6 7 8 9 0 1
+-+-+-+-+-+-+-+-+-+-+-+-+-+-+-+-+-+-+-+-+-+-+-+-+-+-+-+-+-+-+-+-+
|V| SC|        Channel          |              Port             |
+-+-+-+-+-+-+-+-+-+-+-+-+-+-+-+-+-+-+-+-+-+-+-+-+-+-+-+-+-+-+-+-+
:               :               :               :               :
:               :   IP address (32 or 128 bits) :               :
:               :               :               :               :
+=+=+=+=+=+=+=+=+=+=+=+=+=+=+=+=+=+=+=+=+=+=+=+=+=+=+=+=+=+=+=+=+
|             Optional Diffie-Hellman public key                :
+-+-+-+-+-+-+-+-+-+-+-+-+-+-+-+-+-+-+-+-+-+-+-+-+-+-+-+-+-+-+-+-+
    \end{verbatim}
  \end{center}
\caption{Payload of the \texttt{Data\_Control} request
  \label{fig:send_payload}} 
\end{figure}

% \begin{figure}
%   \begin{center}
%     \begin{verbatim}
%  0                   1                   2                   3
%  0 1 2 3 4 5 6 7 8 9 0 1 2 3 4 5 6 7 8 9 0 1 2 3 4 5 6 7 8 9 0 1 2
% +-+-+-+-+-+-+-+-+-+-+-+-+-+-+-+-+-+-+-+-+-+-+-+-+-+-+-+-+-+-+-+-+-+
% |        Reserved   |  CH   |O|V|              Port               |
% +-+-+-+-+-+-+-+-+-+-+-+-+-+-+-+-+-+-+-+-+-+-+-+-+-+-+-+-+-+-+-+-+-+
% :               :               :               :                 :
% :               :            IP address         :                 :
% :               :               :               :                 :
% +=+=+=+=+=+=+=+=+=+=+=+=+=+=+=+=+=+=+=+=+=+=+=+=+=+=+=+=+=+=+=+=+=+
% |    Reserved   |  Auth_Profile Payload Length  |  Auth_Profile   |
% +-+-+-+-+-+-+-+-+-+-+-+-+-+-+-+-+-+-+-+-+-+-+-+-+-+-+-+-+-+-+-+-+-+
% :                                                                 :
% :                         Auth_Profile Payload                    :
% :                                                                 :
% +-+-+-+-+-+-+-+-+-+-+-+-+-+-+-+-+-+-+-+-+-+-+-+-+-+-+-+-+-+-+-+-+-+
%     \end{verbatim}
%   \end{center}
% \caption{Payload of the \ttt Send\_Data! and \ttt Stop! commands
%   \label{fig:send_payload}} 
% \end{figure}
%
% \begin{figure}
%   \begin{center}
%     \begin{verbatim}
%  0                   1                   2                   3
%  0 1 2 3 4 5 6 7 8 9 0 1 2 3 4 5 6 7 8 9 0 1 2 3 4 5 6 7 8 9 0 1 2
% +-+-+-+-+-+-+-+-+-+-+-+-+-+-+-+-+-+-+-+-+-+-+-+-+-+-+-+-+-+-+-+-+-+
% |        Reserved   |  CH   |O|V|              Port               |
% +-+-+-+-+-+-+-+-+-+-+-+-+-+-+-+-+-+-+-+-+-+-+-+-+-+-+-+-+-+-+-+-+-+
% :               :               :               :                 :
% :               :            IP address         :                 :
% :               :               :               :                 :
% +=+=+=+=+=+=+=+=+=+=+=+=+=+=+=+=+=+=+=+=+=+=+=+=+=+=+=+=+=+=+=+=+=+
% |    Reserved   |  Auth_Profile Payload Length  |  Auth_Profile   |
% +-+-+-+-+-+-+-+-+-+-+-+-+-+-+-+-+-+-+-+-+-+-+-+-+-+-+-+-+-+-+-+-+-+
% :                                                                 :
% :                        Auth_Profile Payload                     :
% :                                                                 :
% +=+=+=+=+=+=+=+=+=+=+=+=+=+=+=+=+=+=+=+=+=+=+=+=+=+=+=+=+=+=+=+=+=+
% |    Reserved   | Auth_Profile_2 Payload Length |  Auth_Profile_2 |
% +-+-+-+-+-+-+-+-+-+-+-+-+-+-+-+-+-+-+-+-+-+-+-+-+-+-+-+-+-+-+-+-+-+
% :                                                                 :
% :                        Auth_Profile_2 Payload                   :
% :                                                                 :
% +-+-+-+-+-+-+-+-+-+-+-+-+-+-+-+-+-+-+-+-+-+-+-+-+-+-+-+-+-+-+-+-+-+
%     \end{verbatim}
%   \end{center}
% \caption{Payload of the \ttt Send\_Data! command \label{fig:receive_payload}}
% \end{figure}

\section{What a profile must specify}
\label{subsub:0.0.4;driver}

\subsection{Processing profiles}
\label{sub:0.0.4.0;transport_layer}

As already said, the description of PTL of \ppetp- must be completed
with a \emph{profile description} which specifies the processing
applied to the packets.  In order to be complete a profile description
must specify

\begin{enumerate}
\item
The profile number (field \ttt Profile! in the header)
\item
The processing applied to the content packets and, possibly, how the
content packets are recovered by their processed versions.
\item
If the processing requires some parameters, the profile must specify
\begin{enumerate}
  \item The payload of the \ttt Set_Default! request
  \item The profile-specific header added when the \ttt Inline! bit in
  a data packet is 1.
\end{enumerate}
\item
Any profile-specific requests (field  \ttt Request! in the control
packet header)
\item
The meaning of the \ttt Marker! field.
\end{enumerate}
%
\subsubsection{Currently defined profiles.}
\label{subsub:0.0.4.0.0;transport_layer}

It is worth  giving an informal description of the currently defined
profiles. 

\begin{description}
\item[Basic]  Within this profile the packets are not processed at
  all, but sent as they are.
\item[Vandermonde] This profiles processes the packets by using the
  approach described in \cite{bernardini08:dcc08}.  The goal of this
  profile is to obtain from the original data stream a new stream
  whose band requirement is $1/R$-th of the band requirement of the
  original stream.  The processing is done in a way such that a node
  can recover the original stream as soon as it receives $R$ different
  streams. 
\end{description}
%
\begin{table}
\caption{Numbers associated to defined processing profiles
\label{table:proc_profiles}}
\begin{center}
  \begin{tabular}{|c|l|}
\hline
    \multicolumn{1}{|p{3.3em}}{Profile Number} &
    \multicolumn{1}{|c|}{Profile name} \\
\hline
\hline
   ~0 & Basic \\ \hline
   ~1 & Vandermonde \\ \hline
   15 & Reserved for extensions \\ 
\hline
  \end{tabular}
\end{center}
\end{table}
%
\subsection{Authentication profiles}
\label{sub:0.0.4.1;transport_layer}

In order to be complete, any authentication profile description must
specify

\begin{enumerate}
\item
The profile number
\item
The payload format
\item
The authentication algorithm
\end{enumerate}
%

\begin{table}
\caption{Numbers associated to defined authentication profiles
\label{table:auth_profiles}}
  \begin{center}
    \begin{tabular}{|c|l|}
\hline
    \multicolumn{1}{|p{3.3em}}{Profile Number} &
    \multicolumn{1}{|c|}{Profile name} \\
\hline
\hline
  ~~0 & Void  \\ \hline
  ~~1 & Token \\ \hline
  ~~2 & Signed \\ \hline
  255 & Reserved for extensions \\ 
\hline
    \end{tabular}
  \end{center}
\end{table}

\subsubsection{Currently defined profiles}

It is worth  giving an informal description of the currently defined
profiles. 

\begin{description}
  \item[Void] With this profile no authentication is done and every
  request is accepted.  Since this profile is very unsecure, it should
  be used only in very controlled context (i.e., testing).
  \item[Token] With this protocol an ``authority'' (e.g., the root
  server in a video streaming application) gives at each node a list
  of randomly generated ``tokens'' (64 bit words).  Every time one
  node wants to contact another one, it asks to the authority (which
  will check the identity of the node) for a token to be included in
  the \ttt Send_Data! or \ttt Receive_Data! request.  Each token is to
  be used only once in order to avoid replay attacks.
\begin{commento}
Another possible attack that could be mounted if token
  reuse was allowed is the following: one person pay the subscription
  to a video streaming service and send the received tokens to a
  friend of his.  The friend can use the tokens to watch the video
  without having to pay the subscription
\end{commento}
  \item[Signed] The token approach has two drawbacks
    \begin{itemize}
      \item
        It requires to the server to store all the token sent to every
        clients 
      \item
        Two persons could share the same subscription, but with a
        smaller quality, if one sends to the other part of the
        received tokens.  Since the tokens are not ``bound'' to the
        node which issues the \ttt Send_Data! request, the second
        person can use the tokens to access the data.
    \end{itemize}
   The \textbf{Signed} profile overcomes those drawbacks.  Within this
   profile the authority sends to the node a common key that will be
   used to sign the IP address of the authorized node.
\end{description}

\chapter{Basic processing profile}
\label{sect:0.0;transport_layer}

\section{Profile number}
\label{sub:0.0.0;transport_layer}

The number assigned to the Basic profile is 0.

\section{Data processing}
\label{sect:0.0.0;transport_layer}

The data to be transmitted is simply copied verbatim to the payload. 

\section{Meaning of the \marker- field}
\label{sect:0.0.3;transport_layer}

No meaning is defined for the \marker- field.  It SHOULD be zero.

\section{Profile parameters format}
\label{sect:0.0.1;transport_layer}

Since no processing is done, there is no need for profile specific
parameters.

\section{Profile-specific requests}
\label{sect:0.0.2;transport_layer}

The Basic profile does not have any profile-specific requests.


\chapter{Vandermonde processing profile}
\label{subsub:0.1.1;driver}

\section{Profile number}
\label{sub:0.1.1.0;transport_layer}

The number assigned to the Vandermonde profile is 1.

\section{Data processing}
\label{subsub:0.1.1.0;driver}

This profile is based on the ideas of
\cite{bernardini08:dcc08}. The processing procedure is as follows

\begin{description}
\item[At startup] 
  \begin{enumerate}
    \item
      At startup the node chooses a finite field
$\gf{2^{8d}}$ an element $b \in \gf{2^{8d}}$ and a positive integer $R$
called the \emph{reduction factor}.  
\begin{commento}
Only values $d \in \{1,2,4\}$
are allowed in this profile.  Fields $\gf{2^{8d}}$ are constructed as
polynomial in coefficients in $\{0,1\}$ modulo a specific polynomial.
Table~\ref{table:poly} shows the polynomial to be used for every
choice of $d$.  \ttt GF_Size! in Table~\ref{table:poly} refers to the
name of the corresponding field in the profile-specific header.    
\end{commento}
   \item
The node constructs the
$R$-dimensional row vector
%
\begin{equation}
\label{eq:0;driver}
\bfr = \vriv{1}{b} \cdots {b^{R-1}}
\end{equation}
%
called \emph{reduction vector}
  \end{enumerate}

\begin{table}
\caption{Polynomials used to construct the finite fields used in the
  reduction profile. To obtain the actual polynomial, convert to
  binary the hexadecimal number in the third column.  The bit
  corresponding to $2^n$ corresponds to the coefficient of $x^n$.  For
  example, in the $\gf{2^{8}}$ case, the binary word correspondent of
  \texttt{16\#1\_1D\#} is \texttt{1\_0001\_1101} which corresponds to
  the polynomial $x^{8} + x^4 + x^3 + x^2 + 1$.
 \label{table:poly}}
\begin{center}
\begin{tabular}{ccr}
  \ttt GF_Size!$=d-1$ & Field size & 
\multicolumn{1}{c}{Polynomial} \\
  0  & $2^8$    & \texttt{16\#1\_1D\#}\\
%\texttt{1\_0001\_1101}\\
% $x^8 + x^4 + x^3 + x^2 + 1$\\
  1  & $2^{16}$ & \texttt{16\#1\_002D\#} \\
%\texttt{1\_0000\_0000\_0010\_1101} \\
%$x^{16} + x^5 + x^3 + x^2 + 1$\\
%  2  & $2^{24}$ & $x^{24} + x^{16} + x^{15} +x^{14} +x^{13}+x^{10}+x^9
%  x^7 + x^5 + x^3 + 1$\\
  3  & $2^{32}$ & \texttt{16\#1\_0000\_4299\#}
%\texttt{1\_0000\_0000\_0000\_0000\_1000\_0010\_1001\_1001}
%$x^{32} + x^{15} + x^{9} + x^7 + x^4 + x^3 + 1$
\end{tabular}
\end{center}
\end{table}

\item[For every packet] Let $P$ be the packet to be processed.  Let
  $N$ the number of bytes in $P$.  Let $P(n)$, $n=0,\ldots,N-1$ the
  $n$-th byte of $P$.
  \begin{enumerate}
    \item If $N$ is a multiple of $Rd$, let $Q=P$, otherwise create
    $Q$ by padding $P$ with the following algorithm
    \begin{enumerate}
      \item
        Create a new packet $Q$ by concatenating $P$ with $S \perdef
        Rd - [N \bmod Rd]$ zeros.  Let $L=S+N$ be the number of bytes of
        $Q$. Let $Q(k)$, $k=0, \ldots, L-1$, denote the $k$-th byte
        of $Q$.
      \item If $S < 128$, write $S$ in $Q(L-1)$
      \item If $S \geq 128$ write $128+S \mod 128$ in $Q(L-1)$ and $(S-S
      \mod 128) / 128$ in $Q(L-2)$.

      \begin{commento}
	In other words, if $S < 128$, $S$ is directly written to
	$Q(L-1)$, if $S \geq 128$, its lower seven bits are written to
	$Q(L-1)$ and the remaining bits to $Q(L-2)$.  The most
	significative bit of $Q(L-1)$ acts as a flag and it is 1 if and
	only if $S \geq 128$.  This allows us to store values fof $S
	\leq 2^{15}-1=32,767$, which should suffices for any application.  Note
	that we cannot store $S$ directly in $Q(L)$ and $Q(L-2)$ as a
	16-bit value since if $S=1$ byte $Q(L-2)$ is used by data.
      \end{commento}
    \end{enumerate}
  \item
    For every $m=0, \ldots, L/d-1$ the $d$-byte block $Q(md
    \ldots md+(d-1))$ is mapped to an element $w_m$ of $\gf{2^{8d}}$
    as follows: if $Q_{\ell,k}$ denotes the bit of $Q_\ell$
    corresponding to $2^k$, $k=0, \ldots, 7$, $w_m$ is the element of
    $\gf{2^{8d}}$ corresponding to polynomial
$$
\sum_{c=1}^d \sum_{k=0}^{7} Q_{m d+c,k}\; x^{8c+k}
$$
   \item
     The sequence of words $w_m$, $m=0, \ldots, L/d-1$ is arranged
     (columnwise) in an $R$-row matrix.  More precisely, if $\bfM$
     denotes the resulting matrix, the element of row $r$ and column
     $c$ is
$$
\bfM_{r,c} = w_{r + R c}
$$
\begin{commento}
    Note that the number of columns of $\bfM$ is $L/(Rd)$ which is
    an integer number since $P$ was padded in order to have its length
    equal to a multiple of $Rd$
\end{commento}
   \item
     Matrix $\bfM$ is left-multiplied by the reduction vector $\bfr$
     constructed by the node at start-up time.  The resulting vector
     $\bfu = \bfr \bfM$ is called the \emph{reduced packet}.
   \item Vector $\bfu$ is mapped back to a sequence of bytes which
   will make the payload of the data packet.
  \end{enumerate}
\end{description}

\paragraph{Reconstruction algorithm}
Let $\bfu_1$, \ldots, $\bfu_R$ be $R$ different reduced packets
relative to the same content (padded) packet $Q$ and let $\bfr_1$,
\ldots, $\bfr_R$ be the corresponding reduction vectors.  Let vector
$\bfr_\ell$ be based on $b_\ell$, that is, $\bfr_\ell = [1, b_\ell,
  \ldots, b_\ell^{R-1}]$.  Suppose $b_n \ne b_k$ for every $n\ne k$.
In order to recover packet $Q$ it sufficies to solve the following
linear system (in $\gf{2^{8d}}$!)
%
\begin{equation}
\label{eq:1;driver}
\vciv
  {\bfu_1}
  {\bfu_2}
  \vdots
  {\bfu_R}
= \vciv
  {\quad\bfr_1\quad}
  {\bfr_2}
  \vdots
  {\bfr_R}
  \bfM
= 
\soprasotto{\underbrace{\left[
  \begin{array}{cccc}
    1 & b_1 & \cdots & b_1^{R-1} \\
    1 & b_2 & \cdots & b_2^{R-1} \\
    \vdots & \vdots && \vdots \\
    1 & b_R & \cdots & b_R^{R-1} \\
  \end{array}
 \right]}}{\bfR}
  \bfM
\end{equation}
%
Matrix $\bfR$ in \er{eq:1;driver} is a Wandermonde matrix and the
hypothesis on $b_k$ grants that $\bfR$ is invertible, allowing for the
reconstruction of $\bfM$ and, consequently, of $P$.

\section{Meaning of the \protect\marker- field}
\label{subsub:0.1.1.2;driver}

The three bits of the \ttt Marker! field in the data packet header are
to be interpreted as follows

\begin{description}
  \item[Bit 0 (most significative)] Unused.  It SHOULD be zero.
  \item[Bit 1 (\ttt Padding!)] If this bit is 1 it means that the
  original packet $P$ has been padded before reducing it by using the
  procedure described in Section~\ref{subsub:0.1.1.0;driver}.
  \item[Bit 2 (\ttt Full\_Format!)] Unused.  It SHOULD be zero.
\end{description}


\section{Profile parameters format}
\label{subsub:0.1.1.1;driver}

\subsection{Overview}
\label{sub:0.1.1.1.0;transport_layer}

This section specifies the format of the profile-specific header in
data packet and the payload of request \ttt Set_Default!.

The reduction operated by a peer is uniquely determined when the
following parameters are specified: the Galois field $\gf{2^{8d}}$,
the reduction factor $R$ and the value $b$ used to construct the
reduction vector.  Typically the first two parameters remain constant
during a session, while the third one can occasionally change.

\begin{example}
A typical case which require the change of $b$ is when a node is not
able to recover a complete packet, but it wants nevertheless to
propagate the information.  In such a case it can transmit one of the
received packets which cannot be expected to have a reduction vector
equal to the one choose by the node.  In that case the peer needs to
change temporally the reduction vector.
\end{example}
%
The reduction details can be transmitted in two formats: a full format
(64 bit) and a compact one (32 bit).  

\subsubsection{Full format} The full format specifies the complete set of
parameters. 

\begin{verbatim}
   type Full_Reduction_Header is
     record
        Reserved         : Unsigned_22;

        GF_Size          : Unsigned_2;

        Reduction_Factor : Unsigned_8;
        Reduction_Vector : Unsigned_32;
     end record;

   for Full_Reduction_Header'Bit_Order use System.High_Order_First;
   for Full_Reduction_Header use
     record
        Reserved         at 0 range 0..21;
        GF_Size          at 2 range 6..7;
        Reduction_Factor at 3 range 0..7;
        Reduction_Vector at 4 range 0..31;
     end record;
\end{verbatim}

The meaning of the fields is as follows

\begin{description}
  \item[\ttt GF\_Size!] If $\gf{2^{8d}}$ is the chosen Galois field,
  the value of \ttt GF_Size! is equal to $d-1$.  The correspondence
  between \ttt GF_Size! and the chosen Galois field can be seen in
  Table~\ref{table:poly}.
  \item[{\ttt Reduction\_Factor!}] This is the value of $R$.
  \item[\ttt Reduction\_Vector!] The field \ttt Reduction_Vector!
specifies the value of $b$ used to construct the reduction vector.
Only the \ttt GF_Size!+1 less significative bytes of \ttt
Reduction_Vector! are used, the other SHOULD be set to zero.  The
value of \ttt Reduction_Vector! MUST be
interpreted as explained in the descrition of the construction of
matrix $\bfM$.
\end{description}


\begin{figure}
\begin{center}
\begin{verbatim}
Full format:

   0                :  1            :      2        :          3
   0 1 2 3 4 5 6 7 8:9 0 1 2 3 4 5 6:7 8 9 0 1 2 3 4:5 6 7 8 9 0 1 2
  +-+-+-+-+-+-+-+-+-+-+-+-+-+-+-+-+-+-+-+-+-+-+-+-+-+-+-+-+-+-+-+-+-+
  |                  Reserved                   |GF |  R-Factor     |
  +-+-+-+-+-+-+-+-+-+-+-+-+-+-+-+-+-+-+-+-+-+-+-+-+-+-+-+-+-+-+-+-+-+
  |                      Reduction Vector                           |
  +-+-+-+-+-+-+-+-+-+-+-+-+-+-+-+-+-+-+-+-+-+-+-+-+-+-+-+-+-+-+-+-+-+

Compact format:

   0                :  1            :      2        :          3
   0 1 2 3 4 5 6 7 8:9 0 1 2 3 4 5 6:7 8 9 0 1 2 3 4:5 6 7 8 9 0 1 2
  +-+-+-+-+-+-+-+-+-+-+-+-+-+-+-+-+-+-+-+-+-+-+-+-+-+-+-+-+-+-+-+-+-+
  |                      Reduction Vector                           |
  +-+-+-+-+-+-+-+-+-+-+-+-+-+-+-+-+-+-+-+-+-+-+-+-+-+-+-+-+-+-+-+-+-+
\end{verbatim}
\end{center}
\caption{Full and compact format for profile-specific header \label{fig:full}}
\end{figure}

\subsubsection{Compact format} The compact format specifies only the
reduction vector

\begin{verbatim}
    type Compact_Reduction_Header is
    record
       Reduction_Vector : Unsigned_32;
    end record;

  for Compact_Reduction_Header'Bit_Order use System.High_Order_First;
  for Compact_Reduction_Header use
    record
       Reduction_Vector at 0 range 0..31;
    end record;
\end{verbatim}
%
In other words, the compact format is simply the full format without
the first 32-bit word.

\subsection{Data header format for \ttt Inline!=1} 
If bit \ttt Inline! in the data header is 1, a profile-specific header
is inserted after the data header.  The format used in this case is
the compact one.

\subsection{\ttt Set\_Default! payload} The payload of \ttt
Set_Default! request is always in the full format.


\section{Profile-specific requests}
\label{sect:0.1.1.0;transport_layer}

The Vandermonde profile does not have any profile-specific requests.

\section{Security}
\label{sub:0.1.1.4;driver}

A well-known threat in P2P network is ``stream poisoning'' where a
malicious node injects bougs packets in the stream.  This profile
allows for a simple defense against stream poisoning if one allows a
node to receive a number of input stream $M$ greater than the
reduction factor $R$.  To check if the received data  are correct a
node can 

\begin{enumerate}
\item
  Reconstruct the packet by using $R$ reduced packets.  Let $\bfM$ be the
  recovered matrix.
\item
  For every reduced packet $\bfu_\ell$ not used in reconstructing
  $\bfM$ compute $\bfv_\ell = \bfr_\ell \bfM$, where $\bfr_\ell$ is
  the reduction vector associated with $\bfu_\ell$.
\item
  If $\bfu_\ell=\bfv_\ell$ for every $\ell$, the received data are
  correct.  If such equality is not verified for few values of $\ell$,
  the corresponding reduced packets $\bfu_\ell$ are uncorrect.  If the
  equality is never verified, it is probable that an uncorrect reduced
  packet has been used in recovering $\bfM$.
\end{enumerate}
%
In order to contrast stream poisoning, intermediate node SHOULD check
the correctness of received data (with the algorithm above or in some
other way) and SHOULD NOT propagate uncorrect packets.

\chapter{Void authentication profile}
\label{chap:0;transport_layer}

\section{Profile number}
\label{sect:0.1;transport_layer}

The profile number for the \ttt Void! profile is 0.

\section{Authentication algorithm}
\label{sect:0.3;transport_layer}

This profile makes no authentication and every request is accepted. 

\section{Payload format}
\label{sect:0.2;transport_layer}

This profile has no payload.


\section{Security considerations}
\label{sect:0.4;transport_layer}

Obviously this profile is totally insecure and its use is strongly
discouraged.  It is to be used only in very controlled contexts
(e.g. testing purposes).


\chapter{Token authentication profile}
\label{chap:1;transport_layer}

\section{Profile number}
\label{sect:1.0;transport_layer}

This profile has number 1.

\section{Authentication algorithm}
\label{sect:1.1;transport_layer}

Within this profile we suppose that there is a trusted ``authority''
which grants the access to the P2P network.  (Note that it is not
necessary that the authority is a node of the P2P network).  For
example, in a video streaming context the authority could be a node
managed by the video provider.

\begin{itemize}
\item
When a node $N$ enters the network, the authority, after an
authentication step, gives to $N$ a list of \emph{tokens} represented
by 64-bit random numbers.  
\item
When a node $A$ wants to contact $N$, it
first contacts the authority and ask for a token for $N$.  
\item
$A$ sends the request to $N$ by using the token as authentication
  credential. 
\item
The authentication succedes if the token received from $A$ is in the
token list received from the server.
\item
The token received from $A$ is removed from the token list.  
\end{itemize}


\section{Payload format}
\label{sect:1.2;transport_layer}

The payload is the 64-bit token.

\section{Security considerations}
\label{sect:1.3;transport_layer}

The reason for using every token only once is to avoid ``replay
attacks.'' Consider for example a provider which offers video
streaming subject to subcription.  Alice and Bob could both access to
the service, paying only one subscription.  In order to do that, Alice
subscribes to the service, receives the authentication tokens and
gives them to Bob.  Now both Bob and Alice can watch the video by
using the same tokens.




\chapter{Signed authentication profile}
\label{chap:2;transport_layer}

\section{Profile number}
\label{sect:2.0;transport_layer}

This profile has number 2.

\section{Authentication algorithm}
\label{sect:2.1;transport_layer}

Within this profile we suppose that there is a trusted ``authority''
which grants the access to the P2P network. 


\begin{itemize}
  \item When a node $A$ connects to the network, the authority, after
authentication, gives it a secret key $K$
  \item When a node $B$ wants to receive data from $A$, first contacts
  the server and ask for an authentication token.  $B$ communicates to
  the server the port which will be using for receiving data.
  \item The server concatenate the IP number of $B$ and the port
  number and hashes it with an hash algorithm which uses the secret
  key $K$ communicated to $A$.  The actual hash function used is
  determined by the value of a byte in the payload.  The result of the
  hashing is the authentication credential.
  \item Node $A$ includes the authentication credential in its \ttt
  Send_Data! request
  \item Node $B$ computes the hash by using the address of $A$ and the
  secret key $K$.  The authentication succedes if the computed hash
  matches the token received from $A$.
\end{itemize}

\begin{commento}
  The way the key $K$ is transmitted to the node is outside the scope
  of this specification.  Key transmission will happen by some
  external means.  For example, it can be given in the body of the
  response to an RTSP \ttt PLAY! command, it can be a shared secret
  (e.g. a user password) between server and client or it can be
  obtained by means of some key-exchange procedure (e.g.,
  Diffie-Hellman).
\end{commento}

\section{Payload format}
\label{sect:2.2;transport_layer}

The payload for this authentication profile has the following fields
(see \fref{signed_payload})

\begin{description}
\item[\ttt Hash\_funct! (8 bits)]  This value specifies the hash function used
  to compute the token.  Table~\ref{table:hashes} shows the currently
  defined values.
\item[\ttt Reserved! (24 bits)] Those bits are currently unused and
  reserved for future applications.  They SHOULD be zero.
\item[Hashing result (variable length)] This part depends on the
  hashing algorithm.  Unless specified otherwise, this field is the
  output of the hashing function taken verbatim.
\end{description}
%

\begin{figure}
  \begin{center}
    \begin{verbatim}
 0                :  1            :      2        :          3
 0 1 2 3 4 5 6 7 8:9 0 1 2 3 4 5 6:7 8 9 0 1 2 3 4:5 6 7 8 9 0 1 2
+-+-+-+-+-+-+-+-+-+-+-+-+-+-+-+-+-+-+-+-+-+-+-+-+-+-+-+-+-+-+-+-+-+
|   Hash_funct    |               :    Reserved   :               |
+-+-+-+-+-+-+-+-+-+-+-+-+-+-+-+-+-+-+-+-+-+-+-+-+-+-+-+-+-+-+-+-+-+
:                   Result of the hashing function                :
+-+-+-+-+-+-+-+-+-+-+-+-+-+-+-+-+-+-+-+-+-+-+-+-+-+-+-+-+-+-+-+-+-+
    \end{verbatim}
  \end{center}
\caption{Payload for the \ttt Signed! authentication profile
  \label{fig:signed_payload}} 
\end{figure}

\begin{table}
\caption{Hash algorithms currently defined \label{table:hashes}}
  \begin{center}
    \begin{tabular}{|c|l|l|}
\hline
    \multicolumn{1}{|c}{\ttt Hash\_funct!} &
    \multicolumn{1}{|c|}{Name} &
    \multicolumn{1}{|c|}{Description} \\
\hline
\hline
      0 & HMAC-SHA1 & \cite{HMAC, SHA1} \\
      1 & HMAC-SHA256 & \cite{HMAC, SHA1} \\
      2 & HMAC-SHA384 & \cite{HMAC, SHA1} \\
      3 & HMAC-SHA512 & \cite{HMAC, SHA1} \\
\hline
    \end{tabular}
  \end{center}
\end{table}

\section{Security considerations}
\label{sect:2.3;transport_layer}



\chapter{RTSP and SDP Extensions}
\label{sect:3;driver}

In this chapter we describe few proposed extensions to RTSP and SDP in
order to integrate them with \ppmtp;.

\section{Extensions to SDP}

\begin{itemize}
\item
We propose the addition of the new \ttt <proto>! value \ttt RTP/PPETP!
to the \ttt m=! field (media description field) of SDP
\cite{rfc4566}.  
\item
As with RTSP \cite{rfc2326}, the \ttt <port>!
sub-field of \ttt m=! SHOULD be 0.  Moreover, the same remarks made in
\cite{rfc2326} about the \ttt c=! field apply.
\end{itemize}


\section{Extensions to RTSP}
\label{subsub:0;transport_layer}

\subsection{Extensions to Transport Header}
\label{sub:0.0;transport_layer}

\begin{itemize}
\item
New value \ttt\ppetp-! for \ttt lower-transport! \cite[Section 12.39,
page 61]{rfc2326} is added.  If other transport protocols other than
UDP are used, name of the form \ttt\ppetp-!-\ttt<proto>! can be used.
\item
The port numbers in the \ttt client_port! parameter are to be
interpreted as \ppetp- ports.
\end{itemize}

\section{New header \protect\ttt PPETP-Authentication!}
\label{sect:3.0;transport_layer}

A new header \ttt PPETP-Authentication! is added

\begin{verbatim}
  PPETP-Authentication = "PPETP-Authentication" ":" auth-spec
  auth-spec            = "none"
                       | "token"
                       | "signed" "/" hash-spec "/" key-exchange-spec
  key-exchange-spec    = 1#key-echange-name
  key-exchange-name    = "clear"
                       | "password"
                       | "password-encrypted"
                       | "diffie-hellman"
  hash-spec            = 1#hash-name
  hash-name            = "HMAC-MD5"
                       | "HMAC-SHA1"
                       | "HMAC-SHA256"
\end{verbatim}

This header is sent 
\begin{description}
\item[\ttt S -> C!]
In response to a \ttt SETUP! command.  The values \ttt hash-spec! and
\ttt key-exchange-spec! include a list of methods accepted by the
server.  The client MUST choose one combination.   If the \ttt
key-exchange-spec! field includes \ttt diffie-hellman! the \ttt Key!
field with the server half-key and the Diffie-Hellman parameters MUST
be present.
\item[\ttt C -> S!]  With a \ttt PLAY! command.  In this case both \ttt
hash-spec! and \ttt key-exchange-spec! must include only one
choice. If key exchange procedure \ttt diffie-hellman! is chosen, then
the \ttt PLAY! command MUST include a \ttt PPETP-Key! header.
\end{description}
%
The meaning of the \ttt key-exchange-spec! values are as follows

\begin{description}
\item[\ttt clear!] The key will be sent in clear with the answer to
  the \ttt PLAY! command.
\item[\ttt password!] The key is the user password
\item[\ttt password-encrypted!] The key will be sent with the \ttt
  PLAY! command encrypted with the user password
\item[\ttt diffie-hellman!] The Diffie-Hellman key exchange procedure
  is to be used.  The client will send its half-key in the \ttt
  PPETP-Key!  header with the \ttt PLAY! command and the server will
  send its half-key in the answer to the \ttt PLAY! command.
\end{description}
%
\section{Header \protect\ttt \ppetp--Key!}
\label{sect:3.1;transport_layer}

\subsection{Syntax}
\label{sub:3.1.0;transport_layer}

A new header \ttt\ppetp--key! is added.

\begin{verbatim}
  PPETP-Key         = "PPETP-Key" ":" ppetp-key-value *("," ppetp-key-spec)
  ppetp-key-value   = "key=" 1*HEXDIG
  ppetp-key-spec    = ppetp-field-value
                    | ppetp-basis-value
                    | ppetp-enc-method

  ppetp-field-value = "field=" field-name
  ppetp-field-name  = ppetp-basic-field [ "/" ppetp-polynomial ]
  ppetp-polynomial  = ppetp-monomial 1*("+" [1*DIGIT] ppetp-monomial)
                      [ "+" 1*DIGIT ]
  ppetp-monomial    = "x" "^" 1*DIGIT 
  ppetp-basic-field = "mod" 1*DIGIT
                    = "GF" 1*DIGIT "^" 1*DIGIT

  ppetp-basis-value = "basis=" 1*HEXDIG

  ppetp-enc-name    = "enc=" ppetp-enc-name
  ppetp-enc-name    = "DES"
                    | "AES"
\end{verbatim}

\subsection{Remarks}
\label{sub:3.1.1;transport_layer}

The \ttt\ppetp--Key! header is used for estabilishing a common key
between client and server.  Its actual meaning depends on the value of
\ttt key-exchange-spec! subfield.

\begin{description}
\item[\ttt key-exchange-spec=clear!] Only subfield \ttt key=! SHOULD
  be present.  The value of  \ttt key=! is the actual key value.
\item[\ttt key-exchange-spec=password!] In this case the field
  \ttt\ppetp-Key! SHOULD NOT be present, since a secret shared between
  client and server (e.g. the user password) is going to be used. 
\item[\ttt key-exchange-spec=password-encrypted!] In this case both
  \ttt key=! and \ttt enc=! subfields MUST be present and the other
  subfields SHOULD NOT be present.  The value of  \ttt key=! is key
  encrypted with the algorithm specified in \ttt enc=! using the user
  password to create the key as described in
  Section~\ref{subsect:key-creation}. 
\item[\ttt key-exchange-spec=diffie-hellman!] In this case fields 
  \ttt key!,   \ttt field! and \ttt basis! MUST be present while other
  fields SHOULD NOT be present.  The meaning of fields is described in
  detail in   Section~\ref{subsect:diffie}.
\end{description}
%

\subsection{Key creation}
\label{subsect:key-creation}

\subsection{Diffie-Hellman implementation}
\label{subsect:diffie}

Here we recall briefly the Diffie-Hellman key exchange protocol as
described in RFC 2631 \cite{rfc2631}.

\section{Body for reply to \protect\ttt PLAY! command}
\label{sect:3.2;transport_layer}

The body of the reply to \ttt PLAY! is used to transmit informations
about the \ppetp- parameters as: the number of channels, the
processing profile to use and its parameters, the addresses of the
remote peers and so on.  The body is in XML format with the following
syntax

\input{body-syntax}

% \begin{verbatim}
% start     = Configure
% Configure = element configure { Session+ }
% Session   = element session { 
%               attribute port { xsd:integer },
%               Output, Input }
% 
% Output    = element output { Channel+ }
% Channel   = element channel { 
%               attribute id { xsd:integer },
%               attribute max-target { xsd:integer }?
%               Profile 
%             }
% Profile   = element profile {
%               (attribute name {text} | attribute id { xsd:integer })
%               Parameter* 
%             }
% Parameter = element parameter {
%                attribute name  { text },
%                attribute value { text }
%             }
% 
% Input     = element input { Peer+ }
% Peer      = element peer  { Address, Auth-data* }
% Address   = element address {
%                attribute address { text },
%                attribute port    { xsd:integer },
%                attribute channel { xsd:integer }
%             } 
% Auth-data = element auth-data {
%                attribute command { text }?
%                (attribute profile-name { text }
%               | attribute profile-id { xsd:integer })
%                (attribute value { text } 
%               | element   value { text })
%             }
% \end{verbatim}

\begin{description}
\item[Configure] This is the ``root'' of the XML document which
  describe the required configuration.  Inside a \ttt configure! node
  are contained one or more \ttt session! description.
\item[Session] This node contains the informations necessary to
  configure a session.  Attribute \ttt port! is the number of the
  control port associated to the session.  The sources are described
  in the \ttt input! node and the channels are descirbed in the \ttt output!
  node. 
\item[Output] The content of the \ttt output! node is a sequence of
  \ttt channel!s.
\begin{description}
\item[Channel] Each \ttt channel! describes the configuration required
  for a single channel.  The following informations are given 
  \begin{itemize}
    \item \ttt id! The channel ID, used in some control packets such
    as  \ttt Send_Data! and \ttt Receive_data!
    \item \ttt max-target! Optional.  The maximum number of targets
    which will be attached to this channel
    \item The profile associated to the channel and the corresponding
    processing parameters.  Those informations are contained in a \ttt
    profile! node.
  \end{itemize}
\item[Profile] The profile node specifies the profile name and any
  parameter necessary to the packet processing.  More precisely,
  \begin{itemize}
    \item The profile name is specified with the attribute \ttt name!
    (whose value is the profile name, currently defined names are \ttt
    basic! or \ttt vandermonde!) or with the attribute \ttt id! (whose
    value is the integer corresponding to the profile, currently: 0
    for \ttt basic! and 1 for \ttt vandermonde!)
    \item Each profile parameter is specified with a \ttt param! node.
  \end{itemize}
\item[Param] A \ttt param! node is used to specify a profile
  parameter. A \ttt param! node has two attributes: \ttt name! and
  \ttt value!.  The set of valid parameter names and how the value is
  to be interpreted, depends on the profile.
\end{description}
\item[Input] The node \ttt input! contains a sequence of \ttt peer!
  nodes
  \begin{description}
    \item[Peer]  This node contains  informations on how to contact a
    peer: the peer address in node \ttt address! and
    authentication-related informations in the node \ttt auth-data!. 
    \item[Address] The peer address is described by means of the
    attributes
    \begin{description}
      \item[\ttt address!] The IP address in string form (both IPv4
      and IPv6 addresses are acceptable)
      \item[\ttt port!] The peer control port, in decimal.
      \item[\ttt channel!] The channel number, in decimal
    \end{description}
   \item[Auth-data] Nodes of this type contain the tokens to be used
   to for authentication purposes to the remote peer.  It is necessary
   to specify
   \begin{itemize}
     \item The authentication profile.  This is given in the attribute
     \ttt profile-name!.  Currently defined profile names are \ttt
     void!, \ttt token!, \ttt signed!.  The authentication profile can
     also be given with the attribute \ttt profile-id! whose value is
     the ID number associated to the profile.
     \item The command the token is to be used for (attribute \ttt
     command!).  Currently defined \ttt command! values are \ttt
     start! and \ttt stop!.
    \item The token value, given in the attribute \ttt value! or in a
    node \ttt value!.  The interpretation of this field can be defined
    by the profile and the default value is to interpret this field as
    an hexadecimal number corresponding to the payload to be attacched
    to the control packet.
   \end{itemize}
  \end{description}
\end{description}
%

An example of XML description

\begin{verbatim}
<configure>
  <session port="12345">
    <output>
      <channel id="1" max-target="3">
          <profile name="vandermonde">
             <param name="field" value="16" />
             <param name="reduction-factor" value="8" />
          </profile>
      </channel>
      <channel id="2">
          <profile name="basic"/>
      </channel>
    </output>
    <input>
       <peer>
         <address address="127.0.0.1" port="3214" channel="4" />
         <auth-data profile-name="signed" command="start">
           <value>
             0123456789abcdef
           </value>
         </auth-data>
         <auth-data profile-name="token" command="stop" value="034af45">
       </peer>
       <peer>
       </peer>
    </input>
  </session>
</configure>
\end{verbatim}

\chapter{NAT traversal}
\label{chap:3;transport_layer}

In this chapter we describe briefly the NAT traversal procedure. This
chapter is currently only ``a bunch of ideas.'' \fref{nat-trav} shows
schematically how UDP hole punching is integrated in PPETP.

\begin{enumerate}
  \item Suppose a new user contacts the server.  The server sends to
  the user an address (port and IP) and an ID (an hexadecimal value).
  In the figure it is supposed that the extended RTSP described in
  Chapter~\ref{sect:3;driver} is used, but this is not necessary.  The
  user sends \textbf{using its session port} to the received address a
  \ttt HELO! packet with the ID as payload.  This allows the server to
  know the the pair ``visible'' port and IP associated to the user
  session port.
  \item Another node (B) contacts the server and the same protocol is
  carried out.   In this case we suppose that the server assigns node
  A as a source peer of B.
  \item Upon receipts of the \ttt HELO! command from B, the server
  sends a \ttt Punch! command to A.  The \ttt ACK! to \ttt HELO! sent
  to B implies a \ttt Punch! command for every Peer assigned to B.  A
  and B executes the \ttt Punch! command by sending each other an \ttt
  HELO! packet.  The nodes should \ttt ACK! the \ttt HELO! packets.
\end{enumerate}
%
Note that this require that each node has only one input port.

\begin{figure}
\centerline{\unafigura{nat-trav}}
\caption{Hole punching in PPETP
\label{fig:nat-trav}}
\end{figure}
%

\appendix

\chapter{\ppetp- API}
\label{sect:1;driver}

In this chapter we describe a possible API for \ppetp-.  Currently the
``body'' of this chapter is automatically generated from the actual
\ttt .ads! files.  

\section{User functions}
\label{sect:1.1;driver}

\input{ppetp-api}



\bibliographystyle{alpha}
\bibliography{medusa_biblio}
\end{document}


% LocalWords:  iso english parolae
