\documentclass[a4paper]{medusabook}
\usepackage{amsmath}
\usepackage{amsbsy}
\usepackage{amssymb}
\usepackage{amsthm,amsfonts}
\usepackage{color}
\usepackage{srcltx}
\usepackage{appendix}
%-*- mode: latex; iso-accents-mode: nil; -*-

%
% Macro definition for medusa_book.  The following definitions have
% been ``cut-and-paste''d from my LaTeX definition file, so every now
% and then there are some macros with italian names.  Maybe one day
% I'll do some renaming.  At least, name clashes become very
% improbable...
%

\newif\ifhaschapter
\makeatletter
\@ifundefined{c@chapter}{\haschapterfalse}{\haschaptertrue}
\makeatother

%
% Shorthand for \texttt, used for source code & simila.  Usually I do
% not love an excessive use of shorthand, but this is too frequent...
%
%\def\ttt#1!{\texttt{#1}}
%
% New version of the above comand which makes '_' category equal to
% ``letter.''  This allows me to write \ttt Multimedia_Timestamp!
% instead of \ttt Multimedia\_Timestamp!
%
\def\ttt{\catcode`\_=12 \tttii}
\def\tttii#1!{{\tt #1}\catcode`\_=8{}}
%
% -- Lettere bold
%
\newcommand{\bfa}{{\mathbf a}}
\newcommand{\bfb}{{\mathbf b}}
\newcommand{\bfc}{{\mathbf c}}
\newcommand{\bfd}{{\mathbf d}}
\newcommand{\bfe}{{\mathbf e}}
\newcommand{\bff}{{\mathbf f}}
\newcommand{\bfg}{{\mathbf g}}
\newcommand{\bfh}{{\mathbf h}}
\newcommand{\bfi}{{\mathbf i}}
\newcommand{\bfj}{{\mathbf j}}
\newcommand{\bfk}{{\mathbf k}}
\newcommand{\bfl}{{\mathbf l}}
\newcommand{\bfm}{{\mathbf m}}
\newcommand{\bfn}{{\mathbf n}}
\newcommand{\bfo}{{\mathbf o}}
\newcommand{\bfp}{{\mathbf p}}
\newcommand{\bfq}{{\mathbf q}}
\newcommand{\bfr}{{\mathbf r}}
\newcommand{\bfs}{{\mathbf s}}
\newcommand{\bft}{{\mathbf t}}
\newcommand{\bfu}{{\mathbf u}}
\newcommand{\bfv}{{\mathbf v}}
\newcommand{\bfw}{{\mathbf w}}
\newcommand{\bfx}{{\mathbf x}}
\newcommand{\bfy}{{\mathbf y}}
\newcommand{\bfz}{{\mathbf z}}
%
\newcommand{\bfA}{{\mathbf A}}
\newcommand{\bfB}{{\mathbf B}}
\newcommand{\bfC}{{\mathbf C}}
\newcommand{\bfD}{{\mathbf D}}
\newcommand{\bfE}{{\mathbf E}}
\newcommand{\bfF}{{\mathbf F}}
\newcommand{\bfG}{{\mathbf G}}
\newcommand{\bfH}{{\mathbf H}}
\newcommand{\bfI}{{\mathbf I}}
\newcommand{\bfJ}{{\mathbf J}}
\newcommand{\bfK}{{\mathbf K}}
\newcommand{\bfL}{{\mathbf L}}
\newcommand{\bfM}{{\mathbf M}}
\newcommand{\bfN}{{\mathbf N}}
\newcommand{\bfO}{{\mathbf O}}
\newcommand{\bfP}{{\mathbf P}}
\newcommand{\bfQ}{{\mathbf Q}}
\newcommand{\bfR}{{\mathbf R}}
\newcommand{\bfS}{{\mathbf S}}
\newcommand{\bfT}{{\mathbf T}}
\newcommand{\bfU}{{\mathbf U}}
\newcommand{\bfV}{{\mathbf V}}
\newcommand{\bfW}{{\mathbf W}}
\newcommand{\bfX}{{\mathbf X}}
\newcommand{\bfY}{{\mathbf Y}}
\newcommand{\bfZ}{{\mathbf Z}}
%
\newcommand{\bfomega}{{\boldsymbol\omega}}
\newcommand{\bflambda}{{\boldsymbol\lambda}}
\newcommand{\bfepsilon}{{\boldsymbol\epsilon}}
\newcommand{\bfdelta}{{\boldsymbol\delta}}
\newcommand{\bfDelta}{{\boldsymbol\Delta}}
\newcommand{\bfalpha}{{\boldsymbol\alpha}}
\newcommand{\bfeta}{{\boldsymbol\eta}}
\newcommand{\bfzero}{{\boldsymbol 0}}
\newcommand{\bfuno}{{\boldsymbol 1}}
\newcommand{\bfell}{{\boldsymbol \ell}}


\newcommand{\fref}[2][.]
{Fig.~\ref{fig:#2}\senondot{#1}{({#1})}}%\fref@ccoda{#2}}
%
% \sedotelse {cond} {then} {else}
%
%     Espande a {then} se cond=., a {else} altrimenti
%
\newcommand{\sedotelse}[3]
    {\def\sedottmpa{#1}\def\sedottmpb{.}%
     \ifx\sedottmpa\sedottmpb{#2}\else{#3}\fi}
%
\newcommand{\senondot}[3][\relax]{\sedotelse{#2}{#1}{#3}}
%
\newcommand{\notainterna}[1]{$<<<$ \emph{#1} $>>>$}
\def\medusa-{{\scshape Medusa}}
\def\medusa-{{\scshape Medu$\chi$a}}
\def\medusa-{{\scshape Medu$\Sigma$a}}
%
% -- Comando \unafigura[dim]{filename}
%   default dim = 10cm
%   se dim = .<n> (<n> intero) dim. standard per tabular con righe
%   di <n> figure
%
\newlength{\unafigstdx}
\setlength{\unafigstdx}{10cm}
\newlength{\unafigstdy}
\setlength{\unafigstdy}{10cm}
\newcommand{\sceglidim}[1]{\ifcase #1 10cm\or 10cm\or 7.5 cm\else 4cm\fi}
\def\stddim#1#2;{\ifx .#1\sceglidim{#2}\else #1#2\fi}
\newcommand{\unafigura}[2][\unafigstdx]
   {\resizebox{\stddim #1;}{!}{\includegraphics{#2}}}
\newcommand{\unafiguray}[2][\unafigstdy]
   {\resizebox{!}{\stddim #1;}{\includegraphics{#2}}}
%\newcommand{\unafigurax}[2][\unafigstdx]{\fbox{file=`{#2}', hsize=\stddim
%#1; (to be included)}}
\newcommand{\unafigurax}[2][10cm]{\framebox[\stddim #1;][c]{file=`{#2}', hsize=\stddim #1; (to be included)}}
\newlength{\pippolen}

%
%
%

%"esempio" = italian word for "example". 
\newtheoremstyle{esempio} 
   {\topsep}{\topsep}{\small\listparindent 1.5em
   \advance \linewidth-2em
   \parshape 1 1em \linewidth}
   {}{\hspace{-1em}\itshape}{}{\newline}
   {\thmname{#1}\thmnumber{ #2}\thmnote{ (#3)}}

\theoremstyle{esempio}

\ifhaschapter
\newtheorem{example}{Example}[chapter]
\newtheorem{commento}{Remark}[chapter]
\else
\newtheorem{example}{Example}[section]
\newtheorem{commento}{Remark}[section]
\fi
\newcommand{\soprasotto}[2]{\mathrel{\mathop{#1}\limits_{#2}}}
\newcommand{\commenta}[2]{\soprasotto{\underbrace{#1}}{#2}}
\newcommand{\er}[1]{(\ref{#1})}

%
% List of remarks (\`a la "Ten lectures...")
%
\newenvironment{remarks}
{\par
  \normalfont
  \textbf{Remarks.}
  \begin{enumerate}}
{\end{enumerate}}
\newcommand{\remark}{\item}

\newcommand{\norma}[2][]{\|{#2}\|_{#1}}
\let\norm\norma
\newcommand{\normainf}[1]{\norma[\infty]{#1}}
\newcommand{\abs}[1]{\lvert{#1}\rvert}
\newcommand{\floor}[1]{\left\lfloor{#1}\right\rfloor}
\newcommand{\ceil}[1]{\lceil{#1}\rceil}
\newcommand{\round}[1]{\lceil{#1}\rfloor}
%%%%%%%%%%%%%%%%%%%%%%%%%%%%%%%%%%%%%%%%%%%%%%%%%%%%%%%%%%%%%%%%%%%%%%
\newcommand{\gf}[1]{\operatorname{GF}(#1)}

\newcommand{\redfac}{R}
\newcommand{\redroot}{b}
\newcommand{\redvec}{\bfr}
\newcommand{\reduced}{\bfu}
\newcommand{\redmtx}{\bfR}
\newcommand{\npeers}{N}
\newcommand{\peercap}{M}

\newcommand{\peer}[1]{P_{#1}}
\newcommand{\idvec}[1]{\operatorname{ID}_{#1}}
\newcommand{\locap}{C_{\text{lo}}}
\newcommand{\hicap}{C_{\text{hi}}}
\newcommand{\defect}{D}
\newcommand{\excess}{E}

\newenvironment{API}
               {\list{}{\labelwidth 0pt \itemindent-\leftmargin
                        \let\makelabel\functionlabel}}
               {\endlist}
\newcommand*\functionlabel[1]{\hspace\labelsep
                                \tt\bfseries {#1}}
\def\function{\catcode`\_=12 \funzioneii}
\newcommand\funzioneii[1]{\item[{#1}]\catcode`\_=8}

\def\ppmtp;{PPETP}
\def\ppetp-{PPETP}
\def\ppc-{ETCP}
\newcommand{\vriv}[5][cccc]
{\left[
  \begin{array}{#1}
  {#2} &{#3} &{#4}&{#5} \\
  \end{array} 
\right]}
\newcommand{\vciv}[5][c]
  {\left[
     \begin{array}{#1} 
      {#2} \\ 
      {#3} \\
      {#4} \\
      {#5}
     \end{array}
   \right]}
\newcommand{\perdef}{:=}
\def\marker-{\ttt Marker!}
\DeclareTextCommandDefault{\adarange}
   {{.\kern\fontdimen3\font .\kern\fontdimen3\font}}


%
%\titleimage{chestCN9697}% clarita1000@gmail.com
\title{\ppmtp; internals}
\author{R. Bernardini\\ {\today}}
%
\setcounter{secnumdepth}{3} 
\setcounter{tocdepth}{3}
\begin{document}
%  \maketitle
\tableofcontents
\chapter{Introduction (read me, please!)}

\section{Why this document?}
\label{sect:0;medusa_book}

The goal of this book is to document the internal details of
\ppmtp;. Yes, I know, you already guessed that by reading the
title\ldots{} (if you didn't, you can stop reading here).  Maybe it is
worth spending a few words about \emph{why} I felt the need to write
this document.

First, \emph{what is} \ppmtp;? \ppmtp; means \emph{Peer-to-Peer
  Epi-Transport Protocol} and it is a project started at the University
  of Udine (Italy).  The goal of \ppmtp; is to develop a socket-like
  layer for efficient peer-to-peer (P2P) \emph{live streaming}.  I am
  not going into details here why this is a cool idea and which kind
  of problems must be solved.  The important point that I want to
  emphasize here is that the \ppmtp; library is quite complex a piece
  of software that (I hope) will be maintained in the future by other
  developers (beside me).  Therefore, we need some documentation for
  those future developers.

I already modified open source software (sometimes for necessity,
sometimes for fun) and my experience is that when you are trying to
change a piece of software wrote by someone else, you do not need a
verbose description of every single variable and line of code, but
rather the ``big picture'' of the software.  Instead, it is not
uncommon to find code which looks like (no kidding)

\begin{verbatim}
    int i, j;  /* Loop variables */       (really?!?)
    count ++;  /* Increment count by 1 */ (Wow !)
\end{verbatim}

\noindent
but no description of the ``high level structure'' of the software and
you are left to yourself in finding the part you need to
change\footnote{That can also be fun, mind you, unless you are in a
rush\ldots}.  It is like entering a wood with a description of every
single leaf, but no map.  Well, this document is the ``wood map.''

\chapter{Rough overview}
\label{chap:0;overview}

This chapter gives you a \emph{very} rough overview of the internal
structure of the \ppmtp; library.  Many details (even fairly important
ones) are omitted for the sake of making the high-level structure more
evident.  If you want all the gory details, go to the following
chapters or look at the comments inside the code.

\section{Sessions}
\label{sect:0.1;overview}


\begin{itemize}
\item
A single PC can join one or more P2P network at the time.  In \ppmtp;
jargon each P2P network joined corresponds to a \textbf{Session}.
\item
A Session has (see \fref{session})
\begin{itemize}
  \item
    One or more \textbf{Sources}.  Each source corresponds to a remote
    peer which sends us data.
  \item
    One \textbf{Output queue} which holds the reconstructed packets
    that will be read by the application program
  \item
    One or more \textbf{Channels}.  Each channel has one or more
    \textbf{Targets} and each target corresponds to a remote peer that
    receives data from us.  The targets in a same channel share the
    same reduction parameters, i.e., Galois field, reduction vector
    and reduction factor. Equivalently, one could say that the
    reduction details are associated to the channel.

    We decided to introduce both channels and targets in order to be
    able to send different reduced version to different peers.  It is
    reasonable to expect that the average home user will have only one
    channel. 
\end{itemize}
\end{itemize}

%
\begin{figure}
\centerline{\unafigura[0.8\textwidth]{session}}
\caption{Overview of the internal structure of a Session 
\label{fig:session}}
\end{figure}
%
\subsection{The structure of a channel}
\label{sub:0.1.0;overview}

\fref{channel} shows in greater detail the internal structure of a
Channel.  The Channel receive from the application level the packets
to be transmitted.  Those packets are given to the \ttt Reducer! block
which\ldots{} reduces them (no surprise here).  The reduction details
are contained in the \ttt Reducer! block.  The reduced packets are
finally sent to the Targets which transmits them to the remote peers.

It is worth observing that a channel can also accept already reduced
packets which bypass the \ttt Reducer! block (dashed line in
\fref{channel}).  The explanation of why this can be useful is beyond
the scope of this chapter.

\begin{figure}
\centerline{\unafigura{channel}}
\caption{Internal structure of a Channel
\label{fig:channel}}
\end{figure}
%

\section{The doom of a packet}
\label{sect:0.0;overview}

\fref{main_task} shows a rough overview of what happens to the
received packets.  The packets are received by the sources (each
source is a \ttt task!).  The received packets are written to a common
queue (implemented as a protected object in \ttt Ada!). The queue is
read by the \ttt main_task! which gives the packet to the \ttt
Reconstructor! block.  As soon as enough reduced packets are received,
the \ttt Reconstructor! recover the packet and gives it back to the
main task which in turn hands it to the Session.  
Inside the Session the packet is sent to every Channel which reduces
it and route it toward the remote peers.

\begin{figure}
\centerline{\unafigura{main_task}}
\caption{Path of a packet
\label{fig:main_task}}
\end{figure}
%
\section{Code structure}
\label{sect:0.2;overview}

How the just outlined structure maps itself in the code?  Well, more
or less, every ``block'' shown in the figures above corresponds to a
directory just belove the top-level \ttt PPETP! directory.  There are few
exceptions; for example, the code relative to the \ttt Reducer!  and
\ttt Reconstructor! blocks are belove the \ttt Profiles/Basic!
directory.  This is because the \ttt PPETP! specs do not specify how
the ``reduction'' is done, but delegate such decisions to a
``profile'' document (an approach similar to the one used for RTP).
Therefore, we decided that the code relative to a profile will be
in a subdirectory of \ttt Profiles!.  So far, only two profiles (\ttt
Basic! and \ttt Vandermonde!) are defined.

\chapter{Processing Profiles}


The basic PPETP document describes only the basic packet format and
how command/acknowledge are handled.  The description of how the
original packets are processed and few details about the PPETP packet
format are demanded to a different document (called \emph{profile
  document}).  This type of two-level description is quite common and
it is used, for example, for RTP \cite{rfc3550} and DCCP
\cite{rfc4340} and it is interesting because it allows to add new
functionality to the basic protocol by just adding a new profile.
This chapter explains how the profile idea is implemented in the
actual code.

\section{What a profile is about}
\label{sect:0;overview}

A profile document describes, at least, 

\begin{itemize}
\item
How the original stream of packet is processed in order to obtain the
packets exchanged between peers  (\emph{entangling})
\item
Format details left unspecified by the transport level document, that
is,
\begin{itemize}
\item
The interpretation of the payload of the data packet
\item
The meaning of the profile-specific flags in the data header
\item
The interpretation of the detail payload (in the \ttt Set_Default!
command packet).
\end{itemize}
\end{itemize}
%
\fref{profile} shows a detailed view of the path of a packet received
from the network.  The green boxes/labels are relative to parts
specified by the profile.

\begin{figure}
\centerline{\unafigura[0.9\textwidth]{profile}}
\caption{Path of a network packet.  The green boxes/labels are
  relative to parts decided by the specific profile.
\label{fig:profile}}
\end{figure}
%

\begin{itemize}
\item
The packet received from the network is called a \textbf{network packet}
and it is nothing more than a sequence of bytes.  Such data are
processed by a \textbf{Parser} which includes the knowledge of both the
basic transport level and the profile level.
\item
The result of the parsing \emph{of a data packet} is a \textbf{entangled
  packet}. It is reasonable to expect that this packet will be
  represented by a \ttt record! structure.  The internal details of
  such a structure depends entirely on the profile and only the
  profile-specific functions need to be aware of them.  However, every
  entangled packet will necessarily carry informations about
\begin{itemize}
\item
The packet timestamp
\item
The packet session
\end{itemize}
%
\item
The sequence of entangled packets is sent to a \textbf{disentangler} whose
duty is to recover the original packet stream from the received
entangled streams. Note that the disentangler is specific to a single
session.   The disentangler output is a stream of \textbf{binary} packets
which are \emph{almost} sequences of bytes since they still carry the
information about their timestamp and session.
\item
Each binary packet is sent both to the application level and to each
channel associated with the session
\item
Each channel has its own \textbf{entangler} which processes the binary
packets.  
\item
The entangled packets are sent to a \textbf{builder} which transforms
them into network packets which in turn are sent to the targets
associated with the channel.  Note that the builder, as the parser,
must have knowledge of both the transport and the profile description.
\end{itemize}
%
\begin{commento}
\label{commento:0.0;overview}
It is not necessary that the entangler in \fref{profile} is relative
to the same profile of the disentangler, although we expect that this
will often be the case.
\end{commento}
%
\begin{commento}
\label{commento:0.1;overview}
Sometimes it could be convenient to send to the builder an entangled
packet received from the network (dashed arrow in \fref{profile}).  Of
course, this is possible only if entangler and disentangler are
relative to the same profile.
\end{commento}
%
\section{How a profile is implemented}
\label{sect:0.3;overview}

In order to mirror the flexibility given by the use of profiles, we
decided to implement them as a hierarchy of objects.  More precisely,
a profile is described by the following five ``active'' objects (i.e.,
objects which ``do'' something) and two ``passive'' ones (i.e.,
objects which are pure data structures)

\begin{description}
\label{description:0.3.0;overview}
\item[Active]~
  \begin{description}
    \item[Parser]
    \item[Builder]
    \item[Entangler]
    \item[Disentangler]
    \item[Configuration Parser]
  \end{description}
\item[Passive]~
  \begin{description}
    \item[Entangled]
    \item[Parameters]
  \end{description}
\end{description}
%
\begin{commento}
  The meaning of the classes above should be clear from the discussion
about \fref{profile}.  The only object which is not present in
\fref{profile} is \emph{Parameters} which represents the parameters
used in the (dis-)entangling process.
\end{commento}
%
\begin{commento}
  To be honest, all the object listed above are derived from abstract
  tagged type \ttt Root_Profile_Handler!.  Although such a
  relationship could seem not necessary at first, it is convenient it
  allows to add to every profile-related object a \ttt Profile! field.
  Currently the only method \ttt Root_Profile_Handler! is \ttt
  Profile!. \fref{classes} gives a pictorial representation of the
  class hierarchy related with processing profiles.
\end{commento}
%
\begin{figure}
\centerline{\unafigura[0.95\textwidth]{profile-classes}}
\caption{Hierarchy of the classes related with processing profiles.
  In order to not clog the figure, only the descendants of
  \protect\ttt Root\_Entangler! are shown.
\label{fig:classes}}
\end{figure}
%
In the code we define a ``root'' \emph{abstract} object for each
class.  Such abstract classes are collected, in the \ttt Ada! version
in a single package.  It is expected that every new profile will be
implemented in a \emph{child package} and it will specialize the
abstract classes to actual leaf classes.  Note that the use of
hierarchy of classes allows us to have, for example, arrays of \ttt
Parser! which can contain parsers of different profiles (To be honest:
the array will actually be an array of \emph{access}-es to class-wide
type \ttt Parser'Class!\footnote{Do you ignore what ``class-wide
type \ttt Parser'Class!'' means? Go reading an \ttt Ada!
tutorial\ldots})

The conceptual actions required to the classes above are

\begin{description}
  \item[Parser]  The parser will be called to parse the ``tail'' of a
  command/data packet after the header has been parsed.  More
  precisely, when processing a data packet the parser will receive
  \begin{itemize}
    \item The received network packet
    \item The value of the profile-specific flags
    \item The value of the in-line flag
  \end{itemize}
  and it will return an \ttt Entangled! packet (of the type suitable for the
  \ttt Parser! profile); when processing a command \ttt Set_Default!
  packet it will receive the network packet and it will return a
  record of type \ttt Parameters!.
%
  \item[Builder] The \ttt Builder! duty is, in some sense, the
  opposite of the duty of the \ttt Parser!.  When processing data, it
  will be called with an \ttt Entangled! packet and it will return (i)
  the byte sequence corresponding to the payload (ii) the value of the
  profile-specific flags and (iii) the value of the in-line flag.  When
  processing \ttt Set_Default! command packet, it will be called with
  the \ttt Parameter! data and it will return the command packet
  payload.
%
  \item[Entangler] The duty of this object is to apply the entangling
  processing to each packet.  It receives in input a \ttt
  Binary_Packet! and returns an \ttt Entangled! packet.
%
  \item[Disentangler] This object is slightly more complex than the
  objects above.  The reason is that while the objects above can be
  seen (more or less) as ``functions'' which return a result each time
  are used, a \ttt Disentangler! may require to receive more than one
  \ttt Entangled! packet before returning a \ttt Binary! packet.  For
  example, in the \ttt Vandermonde! profile, a \ttt Disentangler! must
  receive a number of \ttt Entangled! packets which is greater or
  equal than the reduction factor.

  In order to allow for this non 1-to-1 behavior, we suppose that a
  \ttt Disentangler! has an ``internal queue'' of recovered packets.
  The \ttt Disentangler! receives \ttt Entangled! packets, process
  them and, when possible, recovers the corresponding \ttt Binary!
  packet.  Every recovered \ttt Binary! packet is inserted in the
  internal queue of the \ttt Disentangler!.  The queue can be read by
  means of suitable methods.  

  More in detail, a \ttt Disentangler! will respond to the following
  methods
  \begin{description}
    \item[Process] This method receives in input an \ttt Entangled!
    packet, process it and, if possible, recover one or more \ttt
    Binary! packets which are inserted in the internal queue.
    \item[Force] By calling this method we ask the \ttt Disentangler!
    to insert into the internal queue a previously received \ttt
    Entangled! packet. This method requires in input the timestamp of
    the required packet. The reason for this is that sometimes we
    could be not able to recover in time a packet, but we want
    nevertheless to propagate the corresponding information.  
    \item[Get] Extract the first packet (which can be a \ttt Binary!
    one or an \ttt Entangled!, depending on which method [\ttt
    Process! or \ttt Force!] wrote it) from the internal queue.  If the
    queue is empty, it raises \ttt Empty_Queue!.  
    \item[Any\_Ready] Return \ttt True! if there is at least one
    packet in the internal queue.
    \item[Forget] In some applications, a packet is useless if it
    cannot be recovered in time.  By calling method \ttt Forget! we
    say to the \ttt Disentangler! that packets with a given timestamp
    is not required anymore.  This allows the \ttt Disentangler! to
    recover the memory used for that packet.  After a call to \ttt
    Forget! the \ttt Disentangler! is allowed to discard any packet
    with the same timestamp passed to \ttt Forget!.
  \end{description}
  \item[Configuration Parser]  This object is used to map externally
  received strings into profile parameters.  This object must
  implement a method \ttt Parse! which expect as input a ``table'' of
  \ttt (name, value)! pairs, where \ttt name! is the parameter name
  and \ttt value! is the parameter value (quite obvious, isn't?).
  Method \ttt Parse! will return any errors in a list of error
  descriptors.  Each error descriptor is a \ttt record! with at least
  the following fields: \ttt Name!, \ttt Value! and \ttt Reason!.  The
  first two are the values found in the table, while the last one
  describe what kind of error occured.  Currently, the following
  values for \ttt Reason! are possible
  \begin{description}
  \item[\ttt Param\_Unknown!] The parameter name is not known
  \item[\ttt Invalid\_Value!] The parameter value is not valid
  \item[\ttt Param\_Missing!] A mandatory parameter is not present.
  Field \ttt Name! contains the name of the missing parameter, while
  field \ttt Value! contains the empty string.
  \item[\ttt Multiple\_Param!] A parameter was given more than once.
  \end{description}
%
\end{description}

\chapter{Authentication profiles}
\label{chap:1;overview}

\section{Motivation}
\label{sect:1.0;overview}

The \ppetp- specs include also a mechanism for contacting remote nodes
and ask them to open a new channel toward a new destination.  That
mechanism can be used by a node to request data or also by a ``root
server'' to comand a node to send data to another node.  Such a
mechanism is the potential source of several security problems, for
example 

\begin{itemize}
  \item
    Node $A$ could saturate the output band of node $B$ (causing a
    DOS\footnote{Denial of Service}) by asking to $B$ an excessive
    number of streams.
  \item
    Node $A$ could flood $C$ by asking to $B$ to send data to $C$
  \item
    In a scalable context, where one can get a better quality by
    receiving more data, one user with a low-quality subscription
    could obtain a better quality by asking to its peer more than its
    share of streams.
  \item
    A paying user could give to a non-paying one the addresses of the
    peers on the network allowing the non-paying user an unauthorized
    access to the data stream.
\end{itemize}

In order to avoid such problems, the \ppetp- specs includes means to
authenticate the connection request.  Typically, the authentication
credentials are created by a ``supernode'' which both peers trust and
are appended as payload to the packet with the connection request. 
In order to allow for future adaptation of new authentication schemes,
the \ppetp- document does not describe the authentication details, but
it delegates separate ``profile'' descriptions (similarly to what is
done for the processing profile).

Currently the \ppetp- specs include three authentication profiles

\begin{description}
  \item[Void] With this profile no authentication is done and every
  request is accepted.  This profile is obviously very unsecure and it
  is to be used only in the most controlled
  situations (e.g. testing).
  \item[Token] With this protocol an ``authority'' (e.g., the root
  server in a video streaming application) gives at each node a list
  of ``tokens'' (randomly generated 64 bit bitstrings).  Every time
  one node wants to contact another one, it asks to the authority
  (which will check the identity of the node) for a token which will be
  included in the connection request.  
  \item[Signed] The token approach has two drawbacks
    \begin{itemize}
      \item
        It requires to the server to store all the token sent to every
        clients 
      \item
        Two persons could share the same subscription, but with a
        smaller quality, if one sends to the other part of the
        received tokens.  Since the tokens are not ``bound'' to the
        node which issues the \ttt Send_Data! request, the second
        person can use the tokens to access the data.
    \end{itemize}
   The \textbf{Signed} profile overcomes those drawbacks.  Within this
   profile the authority sends to the node a common key that will be
   used to sign the IP address of the authorized node.
\end{description}

\section{Authentication profile implementation}
\label{sect:1.1;overview}

Similarly to the case of processing profiles, we mantain the
flexibility of the \ppetp- specs by declaring ``root'' abstract types
which correspond to the authentication primitive operations and
implementing each profile by deriving from the root types.

Actually, the only action that is required for profile implementation
is credential checking.  Therefore, we define an abstract
\textbf{Root\_Checker} with an abstract method \textbf{Check} which
accepts a \ttt Stream_Element_Array! in input and returns \ttt True! if
the credentials are correct (\ttt False! otherwise).  The root type is
defined in package \ttt Auth_Profile! in the \ttt Auth! directory.  It
is expected that authentication profiles will be implemented as child
packages of \ttt Auth_Profile! in subdirectories of \ttt
Auth!. \notainterna{These are just rough notes.  Things may change
  when we will actually implement the profiles}

\chapter{Internal data structures}
\label{chap:2;overview}

In the PPETP library there are several data structures which represent
the data exchanged with peers.  The goal of this chapter is to give a
rough overview of those data structures.

The most important data structures are

\begin{description}
\item[\ttt Binary\_Packet!] An abstract type which represents ``raw''
  arrays of bytes.  This type is used to derive three concrete types
  \begin{description}
     \item[\ttt Network\_Packet!] It contains the data received from a
       peer/ready to be sent to a peer.  It is just an array of bytes
       with attached the sender's address.
        \begin{description}
          \item[Parsing] It is parsed by the procedures in \ttt
            packets-protocol-parsing! and, for the profile-specific part,
            by the methods of descendants \ttt Root_Parser!.
          \item[Building] It is created by the procedures in
            \ttt packets-protocol-building! and, for the profile-specific part,
            by the methods of descendants \ttt Root_Builder!.
        \end{description}
        %
     \item[\ttt Application\_Packet!] It represents the ``packets''
       exchanged with the upper layer application.  More or less, it is a
       byte array with a timestamp attached.  Packets of this type are
       both received from the upper layer application and created by
       profile disentanglers.  This type of packet is processed by
       profile entanglers in order to create entangled data.
  \end{description}
\item[\ttt Protocol\_Packet!] An abstract type which represents a
  generic PPETP packet.  This type is specialized in two concrete
  types
  \begin{description}
    \item[\ttt Data\_Packet!]  
    \item[\ttt Control\_Packet!]
  \end{description}
\item[\ttt Internal\_Command!] Packets of this type are created by the
  input tasks (which first convert \ttt Received_Packet!s to \ttt
  Protocol_Packet!s and successively embeds the result into an \ttt
  Internal\_Command! structure).  Records of type \ttt
  Internal\_Command! can hold also informations which do not derive
  from a \ttt Protocol_Packet!.  For example, a \ttt Internal_Command!
  is used to signal a timeout event on an input source.
\item[\ttt Parsing\_Buffer!]  The goal of an object of this type is to
  contain the data received from the remote peer and allow for a
  simple access for the parsing procedures.  Although it could seem
  that a \ttt Stream_Element_Array! would suffice, a \ttt
  Parsing\_Buffer! has several methods which makes easier to convert
  sequence of bytes to other types, to check that there are enough
  bytes, and so on.   It is created by parsing procedures  from the
  data contained in a \ttt Network_Packet!.  It is used by all the
  parsing procedures (both profile-related or not).
\end{description}
%

\bibliographystyle{alpha}
\bibliography{medusa_biblio}


\end{document}

% LocalWords:  Epi Udine Reconstructor PPETP RTP Vandermonde DCCP timestamp
% LocalWords:  Parser'Class medusa biblio
