\documentclass[a4paper,10pt]{medusabook}
\newcommand{\fullname}[1]{#1}
\newcommand{\ttbf}[1]{\texttt{\bfseries #1}}
%\documentclass[a4paper,11pt]{article}
%
% text width/height/offset
%
%\textwidth=16.3cm         %%% valore "vero" = 16cm
%\textheight=24.6cm      %%% valore "vero" = 24.5
%\hoffset=-1.64cm
%\voffset=-1in
%%\renewcommand{\baselinestretch}{1.62}
%\renewcommand{\baselinestretch}{1.48} %% Valore ``vero'' per 2interl.= {1.55}
%
% Standard packages...
%
\def\ttt{\catcode`\_=12 \tttii}
\def\tttii#1!{{\tt #1}\catcode`\_=8{}}

\usepackage{amsmath}
\usepackage{amsbsy}
\usepackage{amsthm,amsfonts}
\usepackage{epsf,epsfig}
\usepackage{times}
\usepackage{srcltx}
\usepackage[a4paper,margin=1.2in]{geometry}
\voffset=-0.5in
%
%
%
%%\newtheoremstyle{corto}{no}{n}{\bfseries}{\bfseries}{.}
%%   {\itshape}
%%   {0.25\baselineskip}
%%   {0.25\baselineskip}
%%\theoremstyle{corto}
%% \addtolength{\theorempreskipamount}{-15mm}
%% \addtolength{\theorempostskipamount}{-15mm}
%
%
%
%
% Standard definitions...
%
%-*- mode: latex; iso-accents-mode: nil; -*-

%
% Macro definition for medusa_book.  The following definitions have
% been ``cut-and-paste''d from my LaTeX definition file, so every now
% and then there are some macros with italian names.  Maybe one day
% I'll do some renaming.  At least, name clashes become very
% improbable...
%

\newif\ifhaschapter
\makeatletter
\@ifundefined{c@chapter}{\haschapterfalse}{\haschaptertrue}
\makeatother

%
% Shorthand for \texttt, used for source code & simila.  Usually I do
% not love an excessive use of shorthand, but this is too frequent...
%
%\def\ttt#1!{\texttt{#1}}
%
% New version of the above comand which makes '_' category equal to
% ``letter.''  This allows me to write \ttt Multimedia_Timestamp!
% instead of \ttt Multimedia\_Timestamp!
%
\def\ttt{\catcode`\_=12 \tttii}
\def\tttii#1!{{\tt #1}\catcode`\_=8{}}
%
% -- Lettere bold
%
\newcommand{\bfa}{{\mathbf a}}
\newcommand{\bfb}{{\mathbf b}}
\newcommand{\bfc}{{\mathbf c}}
\newcommand{\bfd}{{\mathbf d}}
\newcommand{\bfe}{{\mathbf e}}
\newcommand{\bff}{{\mathbf f}}
\newcommand{\bfg}{{\mathbf g}}
\newcommand{\bfh}{{\mathbf h}}
\newcommand{\bfi}{{\mathbf i}}
\newcommand{\bfj}{{\mathbf j}}
\newcommand{\bfk}{{\mathbf k}}
\newcommand{\bfl}{{\mathbf l}}
\newcommand{\bfm}{{\mathbf m}}
\newcommand{\bfn}{{\mathbf n}}
\newcommand{\bfo}{{\mathbf o}}
\newcommand{\bfp}{{\mathbf p}}
\newcommand{\bfq}{{\mathbf q}}
\newcommand{\bfr}{{\mathbf r}}
\newcommand{\bfs}{{\mathbf s}}
\newcommand{\bft}{{\mathbf t}}
\newcommand{\bfu}{{\mathbf u}}
\newcommand{\bfv}{{\mathbf v}}
\newcommand{\bfw}{{\mathbf w}}
\newcommand{\bfx}{{\mathbf x}}
\newcommand{\bfy}{{\mathbf y}}
\newcommand{\bfz}{{\mathbf z}}
%
\newcommand{\bfA}{{\mathbf A}}
\newcommand{\bfB}{{\mathbf B}}
\newcommand{\bfC}{{\mathbf C}}
\newcommand{\bfD}{{\mathbf D}}
\newcommand{\bfE}{{\mathbf E}}
\newcommand{\bfF}{{\mathbf F}}
\newcommand{\bfG}{{\mathbf G}}
\newcommand{\bfH}{{\mathbf H}}
\newcommand{\bfI}{{\mathbf I}}
\newcommand{\bfJ}{{\mathbf J}}
\newcommand{\bfK}{{\mathbf K}}
\newcommand{\bfL}{{\mathbf L}}
\newcommand{\bfM}{{\mathbf M}}
\newcommand{\bfN}{{\mathbf N}}
\newcommand{\bfO}{{\mathbf O}}
\newcommand{\bfP}{{\mathbf P}}
\newcommand{\bfQ}{{\mathbf Q}}
\newcommand{\bfR}{{\mathbf R}}
\newcommand{\bfS}{{\mathbf S}}
\newcommand{\bfT}{{\mathbf T}}
\newcommand{\bfU}{{\mathbf U}}
\newcommand{\bfV}{{\mathbf V}}
\newcommand{\bfW}{{\mathbf W}}
\newcommand{\bfX}{{\mathbf X}}
\newcommand{\bfY}{{\mathbf Y}}
\newcommand{\bfZ}{{\mathbf Z}}
%
\newcommand{\bfomega}{{\boldsymbol\omega}}
\newcommand{\bflambda}{{\boldsymbol\lambda}}
\newcommand{\bfepsilon}{{\boldsymbol\epsilon}}
\newcommand{\bfdelta}{{\boldsymbol\delta}}
\newcommand{\bfDelta}{{\boldsymbol\Delta}}
\newcommand{\bfalpha}{{\boldsymbol\alpha}}
\newcommand{\bfeta}{{\boldsymbol\eta}}
\newcommand{\bfzero}{{\boldsymbol 0}}
\newcommand{\bfuno}{{\boldsymbol 1}}
\newcommand{\bfell}{{\boldsymbol \ell}}


\newcommand{\fref}[2][.]
{Fig.~\ref{fig:#2}\senondot{#1}{({#1})}}%\fref@ccoda{#2}}
%
% \sedotelse {cond} {then} {else}
%
%     Espande a {then} se cond=., a {else} altrimenti
%
\newcommand{\sedotelse}[3]
    {\def\sedottmpa{#1}\def\sedottmpb{.}%
     \ifx\sedottmpa\sedottmpb{#2}\else{#3}\fi}
%
\newcommand{\senondot}[3][\relax]{\sedotelse{#2}{#1}{#3}}
%
\newcommand{\notainterna}[1]{$<<<$ \emph{#1} $>>>$}
\def\medusa-{{\scshape Medusa}}
\def\medusa-{{\scshape Medu$\chi$a}}
\def\medusa-{{\scshape Medu$\Sigma$a}}
%
% -- Comando \unafigura[dim]{filename}
%   default dim = 10cm
%   se dim = .<n> (<n> intero) dim. standard per tabular con righe
%   di <n> figure
%
\newlength{\unafigstdx}
\setlength{\unafigstdx}{10cm}
\newlength{\unafigstdy}
\setlength{\unafigstdy}{10cm}
\newcommand{\sceglidim}[1]{\ifcase #1 10cm\or 10cm\or 7.5 cm\else 4cm\fi}
\def\stddim#1#2;{\ifx .#1\sceglidim{#2}\else #1#2\fi}
\newcommand{\unafigura}[2][\unafigstdx]
   {\resizebox{\stddim #1;}{!}{\includegraphics{#2}}}
\newcommand{\unafiguray}[2][\unafigstdy]
   {\resizebox{!}{\stddim #1;}{\includegraphics{#2}}}
%\newcommand{\unafigurax}[2][\unafigstdx]{\fbox{file=`{#2}', hsize=\stddim
%#1; (to be included)}}
\newcommand{\unafigurax}[2][10cm]{\framebox[\stddim #1;][c]{file=`{#2}', hsize=\stddim #1; (to be included)}}
\newlength{\pippolen}

%
%
%

%"esempio" = italian word for "example". 
\newtheoremstyle{esempio} 
   {\topsep}{\topsep}{\small\listparindent 1.5em
   \advance \linewidth-2em
   \parshape 1 1em \linewidth}
   {}{\hspace{-1em}\itshape}{}{\newline}
   {\thmname{#1}\thmnumber{ #2}\thmnote{ (#3)}}

\theoremstyle{esempio}

\ifhaschapter
\newtheorem{example}{Example}[chapter]
\newtheorem{commento}{Remark}[chapter]
\else
\newtheorem{example}{Example}[section]
\newtheorem{commento}{Remark}[section]
\fi
\newcommand{\soprasotto}[2]{\mathrel{\mathop{#1}\limits_{#2}}}
\newcommand{\commenta}[2]{\soprasotto{\underbrace{#1}}{#2}}
\newcommand{\er}[1]{(\ref{#1})}

%
% List of remarks (\`a la "Ten lectures...")
%
\newenvironment{remarks}
{\par
  \normalfont
  \textbf{Remarks.}
  \begin{enumerate}}
{\end{enumerate}}
\newcommand{\remark}{\item}

\newcommand{\norma}[2][]{\|{#2}\|_{#1}}
\let\norm\norma
\newcommand{\normainf}[1]{\norma[\infty]{#1}}
\newcommand{\abs}[1]{\lvert{#1}\rvert}
\newcommand{\floor}[1]{\left\lfloor{#1}\right\rfloor}
\newcommand{\ceil}[1]{\lceil{#1}\rceil}
\newcommand{\round}[1]{\lceil{#1}\rfloor}
%%%%%%%%%%%%%%%%%%%%%%%%%%%%%%%%%%%%%%%%%%%%%%%%%%%%%%%%%%%%%%%%%%%%%%
\newcommand{\gf}[1]{\operatorname{GF}(#1)}

\newcommand{\redfac}{R}
\newcommand{\redroot}{b}
\newcommand{\redvec}{\bfr}
\newcommand{\reduced}{\bfu}
\newcommand{\redmtx}{\bfR}
\newcommand{\npeers}{N}
\newcommand{\peercap}{M}

\newcommand{\peer}[1]{P_{#1}}
\newcommand{\idvec}[1]{\operatorname{ID}_{#1}}
\newcommand{\locap}{C_{\text{lo}}}
\newcommand{\hicap}{C_{\text{hi}}}
\newcommand{\defect}{D}
\newcommand{\excess}{E}

\newenvironment{API}
               {\list{}{\labelwidth 0pt \itemindent-\leftmargin
                        \let\makelabel\functionlabel}}
               {\endlist}
\newcommand*\functionlabel[1]{\hspace\labelsep
                                \tt\bfseries {#1}}
\def\function{\catcode`\_=12 \funzioneii}
\newcommand\funzioneii[1]{\item[{#1}]\catcode`\_=8}

\def\ppmtp;{PPETP}
\def\ppetp-{PPETP}
\def\ppc-{ETCP}
\newcommand{\vriv}[5][cccc]
{\left[
  \begin{array}{#1}
  {#2} &{#3} &{#4}&{#5} \\
  \end{array} 
\right]}
\newcommand{\vciv}[5][c]
  {\left[
     \begin{array}{#1} 
      {#2} \\ 
      {#3} \\
      {#4} \\
      {#5}
     \end{array}
   \right]}
\newcommand{\perdef}{:=}
\def\marker-{\ttt Marker!}
\DeclareTextCommandDefault{\adarange}
   {{.\kern\fontdimen3\font .\kern\fontdimen3\font}}


%\setlength{\nproofskip}{0pt}
%
% Local definitions
%
%
% Title informations
%
\title{HTTP-controlled PPETP network manager}

\author{
\fullname{Riccardo Bernardini}
\organization{DIEGM -- University of Udine}
\address{street=Via delle Scienze, 208;city=Udine;code=33100;country=IT;email=bernardini@uniud.it}}

%
% True document
%
\begin{document}
\maketitle

\tableofcontents

\chapter{Introduction}
\label{sect:intro;driver}

As known, \ppetp- does not mandate a specific overlay network
structure and the duty of organizing the network is demanded to an
external actor called the \emph{network manager}.  In our experiments
we found it convenient to control the network manager via an HTTP
interface.  This allowed us to write the network manager in script
languages (more precisely, we used Ruby) called via CGI
interface. Since the protocol we used is language independent, we
document it here for future reference.

\chapter{Protocol description}
\label{chap:0;protocol}

\section{The network model}
\label{sect:0.0;protocol}

A \emph{network manager} is in charge of several \emph{streams}.  A
stream is uniquely identified by its \emph{stream-name} and the name of the
\emph{server} that is producing the stream.  At each time there can be
only one stream with a given pair (server, stream-name).

Each stream has one or more \emph{users}.  Each user is uniquely
identified by its \emph{username}.  A \emph{user instance} is the
triple (username, stream-name, server).  

\subsection{Stream attributes}
\label{sub:0.0.0;protocol}

A stream is characterized by the following attributes

\begin{description}
\item[\ttt stream\_name!]  The name of the stream.  The network manager uses
  this as an \emph{opaque} identifier that, together with the server
  name, uniquely identifies a stream.
\item[\ttt server!] The server name.  This too is treated as an opaque ID.
\item[\ttt bandwidth!] The  bandwidth required by the stream in bit/second.  
\item[\ttt profile\_name!] The name of the \ppetp- profile used for
  streaming.  Currently defined names are \ttt basic! and \ttt
  vandermonde! 
\item[\ttt profile\_data!] A profile-dependent set of session-level
  profile parameters.   For the two currently defined profiles the set
  is
  \begin{description}
    \item[basic profile] No session-level profile parameter defined
    \item[vandermonde profile] This profile has two session level
    parameters
    \begin{description}
      \item[\ttt gf\_size!] The basis 2 logarithm of the size of the
      Galois field used (e.g., if $\gf{256}$ is used, then \ttt
      gf\_size!=8). Currently ammisible values are $4$ (?), $8$, $16$
      and $32$.
      \item[\ttt red\_fac!] The reduction factor used.
    \end{description}
  \end{description}
\end{description}

\subsection{User attributes}
\label{sub:0.0.1;protocol}

A user session is characterized by the following attributes

\begin{description}
\item[\ttt username!] The name of the user.  The network manager
  uses this as an \emph{opaque} identifier that, together with the
  stream name and the server name, uniquely identifies a user session.
\item[\ttt stream\_name!]  The name of the stream seen by the user. 
This too is treated as an opaque ID.
\item[\ttt server!] The server name.  This too is treated as an opaque
  ID.
\item[\ttt upload\_bw!] The maximum user's upload bandwidth (in bit/s).
\item[\ttt n\_inputs!] The number of input streams required by the
  user. 
\item[\ttt n\_channels!] The number of \ppetp- channels available at
  the user
\item[\ttt peer\_id!] The ID that identifies the user in the \ppetp-
  session.  It is an opaque 32-bit integer.
\item[\ttt address\_type!] The type of user's address description.
  The user address can be described in several ways.  For example, if
  the user has a public IP address, it sufficies to specify an IP
  address and an UDP port; if the user is behind a NAT a more complex
  description (e.g., a ICE-based one) can be necessary.  The value of
  this attribute specifies which description is used.  Currently only
  two description formats are defined: \ttt ip! and \ttt ice!
\item[\ttt address!] The peer address.  The actual content of this
  attribute depends on the value of \ttt address\_type!.  More
  precisely,
  \begin{description}
    \item[case \ttt address\_type!=\ttt ip!]  In this case the \ttt
    address! attribute has the form 

\centerline{\ttt ip\_address! \ttt SP! \ttt port!}

where \ttt ip\_address! is the address of the node (in numeric form or
not, IPv4 or IPv6), \ttt SP! is a whitespace and \ttt port! is a
decimal integer.
    \item[case \ttt address\_type!=\ttt ice!] In this case the \ttt
    address! attribute is a \emph{list} of ICE candidates in the same
    format used for SDP description of ICE candidates \cite{ice-draft-19}.
  \end{description}
\end{description}

\section{Command format}
\label{sect:0.1;protocol}

The network manager accepts command via HTTP.  Both GET and POST
method can be used, although POST is suggested since the NM commands
have side effects.  The commands are transmitted in the usual
``\&''-string used in HTTP request.  From an abstract point of view, a
command is just a collection of parameters.  The actual command is
given in the parameter \ttt command!, other parameters depend on \ttt
command! value.

The currently defined commands (i.e., possible values for \ttt
command!) are

\begin{description}
\item[\ttt create\_stream!]  Add a new stream to the NM DB.  
  \begin{description}
\item[Parameters]
Required
  parameters are
  \begin{description}
    \item[\ttt stream!] The stream name
    \item[\ttt server!] The name of the server
    \item[\ttt bandwidth!]  See Section~\ref{sub:0.0.0;protocol}
    \item[\ttt profile\_name!] See Section~\ref{sub:0.0.0;protocol}
    \item[\ttt profile\_data!] See Section~\ref{sub:0.0.0;protocol}
  \end{description}
\item[Reply]
The reply to this request will have an empty body.
  \end{description}

\item[\ttt destroy\_stream!]
Remove a stream from the NM DB.  This removes also any user connected
to the stream.  Note that this \textbf{does not stop} the streaming,
but simply remove the stream reference from the DB.  
\begin{description}
\item[Parameters]
Required parameters are
  \begin{description}
    \item[\ttt stream!] The stream name
    \item[\ttt server!] The name of the server
  \end{description}
\item[Reply]
The reply to this request will have an empty body.
\end{description}

\item[\ttt join\_user!]
Add a new user to a streaming session.  
\begin{description}
\item[Parameters]
Required parameters are
  \begin{description}
    \item[\ttt stream!] The stream name
    \item[\ttt server!] The name of the server
    \item[\ttt username!]  See Section~\ref{sub:0.0.1;protocol}
    \item[\ttt upload\_bw!] See Section~\ref{sub:0.0.1;protocol}
    \item[\ttt address\_type!] See Section~\ref{sub:0.0.1;protocol}
    \item[\ttt n\_inputs!] See Section~\ref{sub:0.0.1;protocol}
    \item[\ttt n\_channels!] See Section~\ref{sub:0.0.1;protocol}
  \end{description}
\item[Reply]
The body of the reply to this request will contain informations that
allow the node to contact other peers.  The description  of the body
format is outside the scope of this document.
\end{description}

\item[\ttt remove\_user!]  
Remove a user from a streaming session.
Note that this does not stop the data stream toward the user, but
simply removes the user from the database. 
\begin{description}
\item[Parameters]
Required parameters are
  \begin{description}
    \item[\ttt stream!] 
    \item[\ttt server!] 
    \item[\ttt username!]  
    \item[\ttt repair!]  This parameter can be 0 or 1. If it is equal
    to 1, reallocate peers to the lower peers of the removed user.
  \end{description}
\item[Reply]
The reply to this request will have an empty body if \ttt repair!=0,
it will have reparation instructions if \ttt repair!=1.
\end{description}
\end{description}
%
\subsection{Authentication}
\label{sub:0.1.0;protocol}

The protocol supports optional authentication based on a secret key
shared between a ``command issuer'' and the NM.  In order to
authenticate a request parameter \ttt auth! must be added to the
request.  The value of \ttt auth! matchs the following regular
expression

\begin{center}
\tt
([-a-zA-Z0-9]+):([-a-zA-Z0-9]+):([0-9]+):([a-fA-F0-9]+)
\end{center}

where
\begin{itemize}
\item
The first field is the name of the command issuer
\item
The second field is a ``nonce''
\item
The third field is a ``timestamp.''  More precisely, it is a decimal
number representing the time the command was issued expressed as the
number of seconds since 1/1/2010 (it is equal to NTP value minus
3\_409\_784\_048).  NM should refuse too old commands or commands with a
timestamp too far into the future.
\item
The fourth field is an hexadecimal representation of the HMAC-SHA256
obtained by hashing with the shared key the message obtained by
removing from the query the fourth field of the value of \ttt auth!.
For example, if the query was

\begin{center}
\small
command=destroy\_stream\&stream=s1\&auth=root:x256rtU:12345:123456\ldots123\&server=example.com
\end{center}

the message to be hashed would be
\begin{center}
\small
command=destroy\_stream\&stream=s1\&auth=root:x256rtU:12345:\&server=example.com
\end{center}
\end{itemize}

\bibliographystyle{i3e}
\bibliography{biblio,di-jelena,miei,rfc}

\end{document}
