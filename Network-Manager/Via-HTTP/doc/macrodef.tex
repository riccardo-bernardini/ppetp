%-*- mode: latex; iso-accents-mode: nil; -*-

%
% Macro definition for medusa_book.  The following definitions have
% been ``cut-and-paste''d from my LaTeX definition file, so every now
% and then there are some macros with italian names.  Maybe one day
% I'll do some renaming.  At least, name clashes become very
% improbable...
%

\newif\ifhaschapter
\makeatletter
\@ifundefined{c@chapter}{\haschapterfalse}{\haschaptertrue}
\makeatother

%
% Shorthand for \texttt, used for source code & simila.  Usually I do
% not love an excessive use of shorthand, but this is too frequent...
%
%\def\ttt#1!{\texttt{#1}}
%
% New version of the above comand which makes '_' category equal to
% ``letter.''  This allows me to write \ttt Multimedia_Timestamp!
% instead of \ttt Multimedia\_Timestamp!
%
\def\ttt{\catcode`\_=12 \tttii}
\def\tttii#1!{{\tt #1}\catcode`\_=8{}}
%
% -- Lettere bold
%
\newcommand{\bfa}{{\mathbf a}}
\newcommand{\bfb}{{\mathbf b}}
\newcommand{\bfc}{{\mathbf c}}
\newcommand{\bfd}{{\mathbf d}}
\newcommand{\bfe}{{\mathbf e}}
\newcommand{\bff}{{\mathbf f}}
\newcommand{\bfg}{{\mathbf g}}
\newcommand{\bfh}{{\mathbf h}}
\newcommand{\bfi}{{\mathbf i}}
\newcommand{\bfj}{{\mathbf j}}
\newcommand{\bfk}{{\mathbf k}}
\newcommand{\bfl}{{\mathbf l}}
\newcommand{\bfm}{{\mathbf m}}
\newcommand{\bfn}{{\mathbf n}}
\newcommand{\bfo}{{\mathbf o}}
\newcommand{\bfp}{{\mathbf p}}
\newcommand{\bfq}{{\mathbf q}}
\newcommand{\bfr}{{\mathbf r}}
\newcommand{\bfs}{{\mathbf s}}
\newcommand{\bft}{{\mathbf t}}
\newcommand{\bfu}{{\mathbf u}}
\newcommand{\bfv}{{\mathbf v}}
\newcommand{\bfw}{{\mathbf w}}
\newcommand{\bfx}{{\mathbf x}}
\newcommand{\bfy}{{\mathbf y}}
\newcommand{\bfz}{{\mathbf z}}
%
\newcommand{\bfA}{{\mathbf A}}
\newcommand{\bfB}{{\mathbf B}}
\newcommand{\bfC}{{\mathbf C}}
\newcommand{\bfD}{{\mathbf D}}
\newcommand{\bfE}{{\mathbf E}}
\newcommand{\bfF}{{\mathbf F}}
\newcommand{\bfG}{{\mathbf G}}
\newcommand{\bfH}{{\mathbf H}}
\newcommand{\bfI}{{\mathbf I}}
\newcommand{\bfJ}{{\mathbf J}}
\newcommand{\bfK}{{\mathbf K}}
\newcommand{\bfL}{{\mathbf L}}
\newcommand{\bfM}{{\mathbf M}}
\newcommand{\bfN}{{\mathbf N}}
\newcommand{\bfO}{{\mathbf O}}
\newcommand{\bfP}{{\mathbf P}}
\newcommand{\bfQ}{{\mathbf Q}}
\newcommand{\bfR}{{\mathbf R}}
\newcommand{\bfS}{{\mathbf S}}
\newcommand{\bfT}{{\mathbf T}}
\newcommand{\bfU}{{\mathbf U}}
\newcommand{\bfV}{{\mathbf V}}
\newcommand{\bfW}{{\mathbf W}}
\newcommand{\bfX}{{\mathbf X}}
\newcommand{\bfY}{{\mathbf Y}}
\newcommand{\bfZ}{{\mathbf Z}}
%
\newcommand{\bfomega}{{\boldsymbol\omega}}
\newcommand{\bflambda}{{\boldsymbol\lambda}}
\newcommand{\bfepsilon}{{\boldsymbol\epsilon}}
\newcommand{\bfdelta}{{\boldsymbol\delta}}
\newcommand{\bfDelta}{{\boldsymbol\Delta}}
\newcommand{\bfalpha}{{\boldsymbol\alpha}}
\newcommand{\bfeta}{{\boldsymbol\eta}}
\newcommand{\bfzero}{{\boldsymbol 0}}
\newcommand{\bfuno}{{\boldsymbol 1}}
\newcommand{\bfell}{{\boldsymbol \ell}}


\newcommand{\fref}[2][.]
{Fig.~\ref{fig:#2}\senondot{#1}{({#1})}}%\fref@ccoda{#2}}
%
% \sedotelse {cond} {then} {else}
%
%     Espande a {then} se cond=., a {else} altrimenti
%
\newcommand{\sedotelse}[3]
    {\def\sedottmpa{#1}\def\sedottmpb{.}%
     \ifx\sedottmpa\sedottmpb{#2}\else{#3}\fi}
%
\newcommand{\senondot}[3][\relax]{\sedotelse{#2}{#1}{#3}}
%
\newcommand{\notainterna}[1]{$<<<$ \emph{#1} $>>>$}
\def\medusa-{{\scshape Medusa}}
\def\medusa-{{\scshape Medu$\chi$a}}
\def\medusa-{{\scshape Medu$\Sigma$a}}
%
% -- Comando \unafigura[dim]{filename}
%   default dim = 10cm
%   se dim = .<n> (<n> intero) dim. standard per tabular con righe
%   di <n> figure
%
\newlength{\unafigstdx}
\setlength{\unafigstdx}{10cm}
\newlength{\unafigstdy}
\setlength{\unafigstdy}{10cm}
\newcommand{\sceglidim}[1]{\ifcase #1 10cm\or 10cm\or 7.5 cm\else 4cm\fi}
\def\stddim#1#2;{\ifx .#1\sceglidim{#2}\else #1#2\fi}
\newcommand{\unafigura}[2][\unafigstdx]
   {\resizebox{\stddim #1;}{!}{\includegraphics{#2}}}
\newcommand{\unafiguray}[2][\unafigstdy]
   {\resizebox{!}{\stddim #1;}{\includegraphics{#2}}}
%\newcommand{\unafigurax}[2][\unafigstdx]{\fbox{file=`{#2}', hsize=\stddim
%#1; (to be included)}}
\newcommand{\unafigurax}[2][10cm]{\framebox[\stddim #1;][c]{file=`{#2}', hsize=\stddim #1; (to be included)}}
\newlength{\pippolen}

%
%
%

%"esempio" = italian word for "example". 
\newtheoremstyle{esempio} 
   {\topsep}{\topsep}{\small\listparindent 1.5em
   \advance \linewidth-2em
   \parshape 1 1em \linewidth}
   {}{\hspace{-1em}\itshape}{}{\newline}
   {\thmname{#1}\thmnumber{ #2}\thmnote{ (#3)}}

\theoremstyle{esempio}

\ifhaschapter
\newtheorem{example}{Example}[chapter]
\newtheorem{commento}{Remark}[chapter]
\else
\newtheorem{example}{Example}[section]
\newtheorem{commento}{Remark}[section]
\fi
\newcommand{\soprasotto}[2]{\mathrel{\mathop{#1}\limits_{#2}}}
\newcommand{\commenta}[2]{\soprasotto{\underbrace{#1}}{#2}}
\newcommand{\er}[1]{(\ref{#1})}

%
% List of remarks (\`a la "Ten lectures...")
%
\newenvironment{remarks}
{\par
  \normalfont
  \textbf{Remarks.}
  \begin{enumerate}}
{\end{enumerate}}
\newcommand{\remark}{\item}

\newcommand{\norma}[2][]{\|{#2}\|_{#1}}
\let\norm\norma
\newcommand{\normainf}[1]{\norma[\infty]{#1}}
\newcommand{\abs}[1]{\lvert{#1}\rvert}
\newcommand{\floor}[1]{\left\lfloor{#1}\right\rfloor}
\newcommand{\ceil}[1]{\lceil{#1}\rceil}
\newcommand{\round}[1]{\lceil{#1}\rfloor}
%%%%%%%%%%%%%%%%%%%%%%%%%%%%%%%%%%%%%%%%%%%%%%%%%%%%%%%%%%%%%%%%%%%%%%
\newcommand{\gf}[1]{\operatorname{GF}(#1)}

\newcommand{\redfac}{R}
\newcommand{\redroot}{b}
\newcommand{\redvec}{\bfr}
\newcommand{\reduced}{\bfu}
\newcommand{\redmtx}{\bfR}
\newcommand{\npeers}{N}
\newcommand{\peercap}{M}

\newcommand{\peer}[1]{P_{#1}}
\newcommand{\idvec}[1]{\operatorname{ID}_{#1}}
\newcommand{\locap}{C_{\text{lo}}}
\newcommand{\hicap}{C_{\text{hi}}}
\newcommand{\defect}{D}
\newcommand{\excess}{E}

\newenvironment{API}
               {\list{}{\labelwidth 0pt \itemindent-\leftmargin
                        \let\makelabel\functionlabel}}
               {\endlist}
\newcommand*\functionlabel[1]{\hspace\labelsep
                                \tt\bfseries {#1}}
\def\function{\catcode`\_=12 \funzioneii}
\newcommand\funzioneii[1]{\item[{#1}]\catcode`\_=8}

\def\ppmtp;{PPETP}
\def\ppetp-{PPETP}
\def\ppc-{ETCP}
\newcommand{\vriv}[5][cccc]
{\left[
  \begin{array}{#1}
  {#2} &{#3} &{#4}&{#5} \\
  \end{array} 
\right]}
\newcommand{\vciv}[5][c]
  {\left[
     \begin{array}{#1} 
      {#2} \\ 
      {#3} \\
      {#4} \\
      {#5}
     \end{array}
   \right]}
\newcommand{\perdef}{:=}
\def\marker-{\ttt Marker!}
\DeclareTextCommandDefault{\adarange}
   {{.\kern\fontdimen3\font .\kern\fontdimen3\font}}

