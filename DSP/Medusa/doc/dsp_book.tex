%-*- mode: latex; ispell-local-dictionary: "english" -*-
\documentclass{medusabook}

\usepackage{amsmath}
\usepackage{amsbsy}
\usepackage{amssymb}
\usepackage{amsthm,amsfonts}
\usepackage{color}
\usepackage{srcltx}
\usepackage{appendix}
%-*- mode: latex; iso-accents-mode: nil; -*-

%
% Macro definition for medusa_book.  The following definitions have
% been ``cut-and-paste''d from my LaTeX definition file, so every now
% and then there are some macros with italian names.  Maybe one day
% I'll do some renaming.  At least, name clashes become very
% improbable...
%

\newif\ifhaschapter
\makeatletter
\@ifundefined{c@chapter}{\haschapterfalse}{\haschaptertrue}
\makeatother

%
% Shorthand for \texttt, used for source code & simila.  Usually I do
% not love an excessive use of shorthand, but this is too frequent...
%
%\def\ttt#1!{\texttt{#1}}
%
% New version of the above comand which makes '_' category equal to
% ``letter.''  This allows me to write \ttt Multimedia_Timestamp!
% instead of \ttt Multimedia\_Timestamp!
%
\def\ttt{\catcode`\_=12 \tttii}
\def\tttii#1!{{\tt #1}\catcode`\_=8{}}
%
% -- Lettere bold
%
\newcommand{\bfa}{{\mathbf a}}
\newcommand{\bfb}{{\mathbf b}}
\newcommand{\bfc}{{\mathbf c}}
\newcommand{\bfd}{{\mathbf d}}
\newcommand{\bfe}{{\mathbf e}}
\newcommand{\bff}{{\mathbf f}}
\newcommand{\bfg}{{\mathbf g}}
\newcommand{\bfh}{{\mathbf h}}
\newcommand{\bfi}{{\mathbf i}}
\newcommand{\bfj}{{\mathbf j}}
\newcommand{\bfk}{{\mathbf k}}
\newcommand{\bfl}{{\mathbf l}}
\newcommand{\bfm}{{\mathbf m}}
\newcommand{\bfn}{{\mathbf n}}
\newcommand{\bfo}{{\mathbf o}}
\newcommand{\bfp}{{\mathbf p}}
\newcommand{\bfq}{{\mathbf q}}
\newcommand{\bfr}{{\mathbf r}}
\newcommand{\bfs}{{\mathbf s}}
\newcommand{\bft}{{\mathbf t}}
\newcommand{\bfu}{{\mathbf u}}
\newcommand{\bfv}{{\mathbf v}}
\newcommand{\bfw}{{\mathbf w}}
\newcommand{\bfx}{{\mathbf x}}
\newcommand{\bfy}{{\mathbf y}}
\newcommand{\bfz}{{\mathbf z}}
%
\newcommand{\bfA}{{\mathbf A}}
\newcommand{\bfB}{{\mathbf B}}
\newcommand{\bfC}{{\mathbf C}}
\newcommand{\bfD}{{\mathbf D}}
\newcommand{\bfE}{{\mathbf E}}
\newcommand{\bfF}{{\mathbf F}}
\newcommand{\bfG}{{\mathbf G}}
\newcommand{\bfH}{{\mathbf H}}
\newcommand{\bfI}{{\mathbf I}}
\newcommand{\bfJ}{{\mathbf J}}
\newcommand{\bfK}{{\mathbf K}}
\newcommand{\bfL}{{\mathbf L}}
\newcommand{\bfM}{{\mathbf M}}
\newcommand{\bfN}{{\mathbf N}}
\newcommand{\bfO}{{\mathbf O}}
\newcommand{\bfP}{{\mathbf P}}
\newcommand{\bfQ}{{\mathbf Q}}
\newcommand{\bfR}{{\mathbf R}}
\newcommand{\bfS}{{\mathbf S}}
\newcommand{\bfT}{{\mathbf T}}
\newcommand{\bfU}{{\mathbf U}}
\newcommand{\bfV}{{\mathbf V}}
\newcommand{\bfW}{{\mathbf W}}
\newcommand{\bfX}{{\mathbf X}}
\newcommand{\bfY}{{\mathbf Y}}
\newcommand{\bfZ}{{\mathbf Z}}
%
\newcommand{\bfomega}{{\boldsymbol\omega}}
\newcommand{\bflambda}{{\boldsymbol\lambda}}
\newcommand{\bfepsilon}{{\boldsymbol\epsilon}}
\newcommand{\bfdelta}{{\boldsymbol\delta}}
\newcommand{\bfDelta}{{\boldsymbol\Delta}}
\newcommand{\bfalpha}{{\boldsymbol\alpha}}
\newcommand{\bfeta}{{\boldsymbol\eta}}
\newcommand{\bfzero}{{\boldsymbol 0}}
\newcommand{\bfuno}{{\boldsymbol 1}}
\newcommand{\bfell}{{\boldsymbol \ell}}


\newcommand{\fref}[2][.]
{Fig.~\ref{fig:#2}\senondot{#1}{({#1})}}%\fref@ccoda{#2}}
%
% \sedotelse {cond} {then} {else}
%
%     Espande a {then} se cond=., a {else} altrimenti
%
\newcommand{\sedotelse}[3]
    {\def\sedottmpa{#1}\def\sedottmpb{.}%
     \ifx\sedottmpa\sedottmpb{#2}\else{#3}\fi}
%
\newcommand{\senondot}[3][\relax]{\sedotelse{#2}{#1}{#3}}
%
\newcommand{\notainterna}[1]{$<<<$ \emph{#1} $>>>$}
\def\medusa-{{\scshape Medusa}}
\def\medusa-{{\scshape Medu$\chi$a}}
\def\medusa-{{\scshape Medu$\Sigma$a}}
%
% -- Comando \unafigura[dim]{filename}
%   default dim = 10cm
%   se dim = .<n> (<n> intero) dim. standard per tabular con righe
%   di <n> figure
%
\newlength{\unafigstdx}
\setlength{\unafigstdx}{10cm}
\newlength{\unafigstdy}
\setlength{\unafigstdy}{10cm}
\newcommand{\sceglidim}[1]{\ifcase #1 10cm\or 10cm\or 7.5 cm\else 4cm\fi}
\def\stddim#1#2;{\ifx .#1\sceglidim{#2}\else #1#2\fi}
\newcommand{\unafigura}[2][\unafigstdx]
   {\resizebox{\stddim #1;}{!}{\includegraphics{#2}}}
\newcommand{\unafiguray}[2][\unafigstdy]
   {\resizebox{!}{\stddim #1;}{\includegraphics{#2}}}
%\newcommand{\unafigurax}[2][\unafigstdx]{\fbox{file=`{#2}', hsize=\stddim
%#1; (to be included)}}
\newcommand{\unafigurax}[2][10cm]{\framebox[\stddim #1;][c]{file=`{#2}', hsize=\stddim #1; (to be included)}}
\newlength{\pippolen}

%
%
%

%"esempio" = italian word for "example". 
\newtheoremstyle{esempio} 
   {\topsep}{\topsep}{\small\listparindent 1.5em
   \advance \linewidth-2em
   \parshape 1 1em \linewidth}
   {}{\hspace{-1em}\itshape}{}{\newline}
   {\thmname{#1}\thmnumber{ #2}\thmnote{ (#3)}}

\theoremstyle{esempio}

\ifhaschapter
\newtheorem{example}{Example}[chapter]
\newtheorem{commento}{Remark}[chapter]
\else
\newtheorem{example}{Example}[section]
\newtheorem{commento}{Remark}[section]
\fi
\newcommand{\soprasotto}[2]{\mathrel{\mathop{#1}\limits_{#2}}}
\newcommand{\commenta}[2]{\soprasotto{\underbrace{#1}}{#2}}
\newcommand{\er}[1]{(\ref{#1})}

%
% List of remarks (\`a la "Ten lectures...")
%
\newenvironment{remarks}
{\par
  \normalfont
  \textbf{Remarks.}
  \begin{enumerate}}
{\end{enumerate}}
\newcommand{\remark}{\item}

\newcommand{\norma}[2][]{\|{#2}\|_{#1}}
\let\norm\norma
\newcommand{\normainf}[1]{\norma[\infty]{#1}}
\newcommand{\abs}[1]{\lvert{#1}\rvert}
\newcommand{\floor}[1]{\left\lfloor{#1}\right\rfloor}
\newcommand{\ceil}[1]{\lceil{#1}\rceil}
\newcommand{\round}[1]{\lceil{#1}\rfloor}
%%%%%%%%%%%%%%%%%%%%%%%%%%%%%%%%%%%%%%%%%%%%%%%%%%%%%%%%%%%%%%%%%%%%%%
\newcommand{\gf}[1]{\operatorname{GF}(#1)}

\newcommand{\redfac}{R}
\newcommand{\redroot}{b}
\newcommand{\redvec}{\bfr}
\newcommand{\reduced}{\bfu}
\newcommand{\redmtx}{\bfR}
\newcommand{\npeers}{N}
\newcommand{\peercap}{M}

\newcommand{\peer}[1]{P_{#1}}
\newcommand{\idvec}[1]{\operatorname{ID}_{#1}}
\newcommand{\locap}{C_{\text{lo}}}
\newcommand{\hicap}{C_{\text{hi}}}
\newcommand{\defect}{D}
\newcommand{\excess}{E}

\newenvironment{API}
               {\list{}{\labelwidth 0pt \itemindent-\leftmargin
                        \let\makelabel\functionlabel}}
               {\endlist}
\newcommand*\functionlabel[1]{\hspace\labelsep
                                \tt\bfseries {#1}}
\def\function{\catcode`\_=12 \funzioneii}
\newcommand\funzioneii[1]{\item[{#1}]\catcode`\_=8}

\def\ppmtp;{PPETP}
\def\ppetp-{PPETP}
\def\ppc-{ETCP}
\newcommand{\vriv}[5][cccc]
{\left[
  \begin{array}{#1}
  {#2} &{#3} &{#4}&{#5} \\
  \end{array} 
\right]}
\newcommand{\vciv}[5][c]
  {\left[
     \begin{array}{#1} 
      {#2} \\ 
      {#3} \\
      {#4} \\
      {#5}
     \end{array}
   \right]}
\newcommand{\perdef}{:=}
\def\marker-{\ttt Marker!}
\DeclareTextCommandDefault{\adarange}
   {{.\kern\fontdimen3\font .\kern\fontdimen3\font}}


%
%\titleimage{chestCN9697}% clarita1000@gmail.com
\title{Processing module}
\author{R. Bernardini}
%
\setcounter{secnumdepth}{3} 
\setcounter{tocdepth}{3}
\begin{document}
  \maketitle
\tableofcontents

\chapter{A quick tour}
\label{chap:0;dsp_book}

\section{Warning}
\label{sect:0.0;dsp_book}

I'll assume that you already know what the processing module is and
what is its duty.  I'll also assume that you are familiar with
\medusa- nomenclature and that you understand the meaning of
expressions such as \emph{analog component}, \emph{reduction vector}
and \emph{reduced packet}.  If you do not, I suggest you to read the
document \emph{Medusa internals} that you'll find a couple of
directories above this\ldots

\section{Overview of the Processing Module}
\label{sect:0.1;dsp_book}

As you (should) know, the processing module is the ``heart'' of
\medusa- since it is the module which receives the reduced packets,
reconstructs the original stream and produces a new reduced stream
which is forwarded to other peers.  Because of this, this module is
maybe the most complex module of \medusa-.  The goal of this document
is to explain the internal structure of the processing module.

An overview of the behaviour of the processing module is the following
\begin{itemize}
\item
A new \emph{network packet} is received from a peer
\item
The network packet is split into its \emph{encoded components}.  The
type of each component is known since the server gave us such an
information with the \ttt MMEDIA! command.
\item
Each component is  \emph{decoded}.  This step is typically trivial for
a binary component, but it is necessary for an analog component.
\item
Each decoded component is given to the
corresponding \emph{processor} which 
\begin{itemize}
  \item
    Stores the reduced component in an internal table
  \item
    If there are enough reduced components to reconstructs the original
    component 
    \begin{itemize}
      \item
        Reconstruct the component
      \item
        Compute your own reduced component and encode it. 
      \item
        If all the reduced components with a given timestamp have been
        computed, create a new network packet and transmit it to the
        other peers.
      \item
        If all the components of a multimedia packet have been
        reconstructed, recreate the packet and send it to the player.
    \end{itemize}
\end{itemize}
\end{itemize}
%
With reference to the \medusa- multimedia model, we can say that a
processor processes a single stream.  In the case of contents with
more than one constituent or more than one stream, several different
processors will be allocated.

\subsection{The multitasking structure}
\label{sub:0.1.0;dsp_book}

Beside the steps described above, the processing module needs also to
send data to other peers and receive data and commands from different
TCP/UDP ports.  The program is much simpler if the duty of the
processing module are partitioned among different threads of
execution implemented as Ada \ttt task!s.

In \fref{structure} one can see a ``low resolution'' view of the
processing module. Tasks are shown with continuous line boxes, Ada
protected objects are shown with double rounded boxes and ``internal
parts'' of tasks are shown with dashed line boxes.  As shown in
\fref{structure}, there are several \ttt Reader! tasks which listen to
TCP/UDP ports for data or commands.  When a \ttt Reader! receives some
data, writes it to the \ttt Input Queue! (which can store both
commands and multimedia data).  The \ttt Input Queue! is read by the \ttt Main!
task.  The data read is given to an internal interpreter (if it is a
command) or to the \ttt Processor! (if it is multimedia data, i.e., a
\ttt network\_packet!).  The results produced by the \ttt Processor!
(reconstructed multimedia packets and new reduced \ttt
network\_packet!) are sent to two packet queues which are read by the
two \ttt Writer!s.  One \ttt Writer! sends packets to the other peers,
while the other sends data to the player.

\begin{figure}
\centerline{\unafigura{structure}}
\caption{Structure of the processing module
\label{fig:structure}}
\end{figure}
%
Both the \ttt Network Queue! and the \ttt Player Queue! shown in
\fref{structure} are special queues since they receives components
(reduced and encoded components the \ttt Network Queue!, reconstruced
components the \ttt Player Queue!) and returns ``complete'' packets.
Since the reconstruction of the packet from its components is done
inside the queue, those queues are called \emph{synthesis queues} or
\emph{S-queues}. 

\subsection{Processor structure}
\label{sub:0.1.1;dsp_book}

Going into even deeper details, \fref{processor-structure} shows the
internal structure of the \ttt Processor! block of \fref{structure}.
The \ttt network\_packet!s read from the \ttt Input Queue! are first
split into their components by block \ttt Splitter!.  Succesively they
are (eventually) decoded by block \ttt Decoder! and sent to the
corresponding processor (\emph{analog} or \emph{binary}). 

\begin{figure}
\centerline{\unafigura{processor-structure}}
\caption{Internal structure of the processor block
\label{fig:processor-structure}}
\end{figure}
%
Both processors share the same structure: 
\begin{itemize}
  \item A \emph{Decoder} whose duty is to decode the encoded
  components extracted by the network packet.  It is reasonable to
  expect that in the binary case this module will be trivial.
  \item A \emph{Recover} block whose
   duty is the reconstruction of the unreduced components. The output
   of the recover block goes both to the succesive block and to the
   player S-queue.
  \item A \emph{Reducer} block which takes a component and reduces
  it. 
  \item An \emph{Encoder} which maps the reduced component to a
  bitstring. It is reasonable to expect that in the binary case this
  module will be trivial.  The encoded components are sent to the
  network S-queue.
\end{itemize}

\subsection{Use of \ttt generic! packages}
\label{sub:0.1.2;dsp_book}

It is clear that the internal core of the processing module is made of
several similar modules and many actions (especially packets
homekeeping) are common to many packages.  This suggests of the \ttt
generic package! feature of \ttt Ada!

\chapter{The generic processor}

The analog and binary processors are implemented as specializations of
a generic processor.  More precisely, in directory  \ttt Processor/Generic!
one can find package \ttt Generic\_Processor! which provides a \ttt
Processor\_Type! type.  According to \fref{processor-structure}, in
order to have a fully working processor one must specify

\begin{itemize}
\item
The encoder/decoder blocks
\item
The recover block
\item
The reducer block
\end{itemize}
%
Each block is represented by a type with a collection of methods (for
example, the encoder must respond to an \ttt Encode!  method; the
recover object must accept new reduced components and returns complete
components.).  Specifying those blocks require also to specify their
types and the types they work with.  More precisely,

\begin{itemize}
  \item
    The type of the reduced component (output of the decoder block and
    input of the recover and encoder blocks)
  \item
    The type of the complete component (output of the recover block
    and input of the reducer block)
\end{itemize}
%
The actual blocks (that is, the actual objects which will be used to
carry out the decoding/recovering/reducing/encoding steps) 
specified at initialization time.  This will allows for having
several, say, analog processors using different enconding
procedures.\footnote{I believe that if such a flexibility is desired,
  the encoder/decoder types should be \ttt Ada! class types}.

\begin{description}
\item[\ttt Initialize!] Of course, this is the first function to be
  called.  This function will need 
  \begin{itemize}
    \item The destination S-queues
    \item The encoder/decoder objects
  \end{itemize}
\item[\ttt Receive!] This procedure is used to give a new \ttt
  network\_crumb! to the processor.
\item[\ttt Packet\_Needed!] This procedure is used to request a packet
  with a given timestamp when the packet deadline is going to expire.
  Note that the processor sends packets to the network and player
  S-queues as soon as they are ready.  However, it is possible to
  request to the processor a packet ``with urgency'' by using this
  procedure.
\item[\ttt Packet\_Lost!] This procedure is used to signal to the
  processor that a packet deadline expired and that the packet has
  been declared lost.  The processor can free any resource used for
  the lost packet.
\end{description}
%

\chapter{Data Types}
\label{chap:1;dsp_book}

Inside the processor module data mainly travel in ``packets.''  There
are several types of packets

\begin{description}
\item[\ttt network\_packet!]  This represents the packet received via
  the network and suitably parsed to an internal representation. (The
  parsing from a collection of byte to the internal representation
  will be done, most probably, inside the task \ttt reader!).  Among
  its fields it will have
  \begin{description}
    \item[\ttt Timestamp!] which coincides with the \ttt
    Multimedia\_Timestamp! associated with the corresponding
    multimedia packet (remember that there is a one-to-one
    correspondence between multimedia and network packets)
    \item[\ttt Stream\_ID!] which determines the stream the packet
    belongs to.
    \item[\ttt Class!] Inherited from the corresponding \ttt
    multimedia_packet!.  The purpose of this field is to allow to
    distinguish among different packet types in a single stream, e.g.,
    B, P and I packets in a video stream.
    \item[\ttt Priority!] Inherited from the corresponding \ttt
    multimedia_packet!.  The purpose of this field is to allow for
    selective dropping of packets in case of congestion.
    \item[\ttt Payload!] An array of \ttt network_crumb!s
  \end{description}
 \item[\ttt network\_crumb!] It represents the encoded version of a
 reduced component.  It is extracted from a \ttt network_packet! by
 the \ttt Splitter! module and sent to the corresponding processor (on
 the basis of its \ttt Stream_ID!).  It will decoded by the \ttt
 Decoder! module to obtain the corresponding \ttt
 reduced_component!. Among its fields a \ttt network_crumb! has
 \begin{description}
   \item[\ttt Timestamp!] Inherited from \ttt network_packet!
   \item[\ttt Stream\_ID!] Inherited from \ttt network_packet!
   \item[\ttt Reduction\_IDX!] A long integer which identifies
   the \emph{reduction vector} used to reduce the component.
   \item[\ttt Component\_IDX!] An integer which identified the
   component type.
   \item[\ttt Payload!] The bitstring which represents the encoded
   content. 
 \end{description}
 \item[\ttt reduced\_component!] It represents the reduced version of
 a component (analog or binary).\footnote{To be precise, there are at
 least two types of \ttt reduced\_components!: \ttt reduced\_analog! and
 \ttt reduced\_binary!.}  Among its fields it has
 \begin{description}
   \item[\ttt Timestamp!] Inherited from \ttt network_crumb!
   \item[\ttt Stream\_ID!] Inherited from \ttt network_crumb!
   \item[\ttt Reduction\_IDX!] Inherited from \ttt network_crumb!
   \item[\ttt Component\_IDX!] Inherited from \ttt network_crumb!
   \item[\ttt Payload!] The values associated with the reduced
 component.  It is an array of \ttt Float! or \ttt GF! elements.
 \end{description}
%
 \ttt Reduced_component!s are fed to the \ttt Recover! block and
 produced by the \ttt Reducer! block.
 \item[\ttt complete\_component!] It corresponds to a component of the
 original multimedia packet.  It share with the \ttt
 reduced_component! all the fields but \ttt Reduction_IDX!. \ttt
 Complete_component!s are produced by the \ttt Recover! block and fed
 to the  \ttt Reducer! block and to the player \ttt S-queue!.
 \item[\ttt multimedia\_packet!] This represents the data packet sent
 to the player.  Its format is clearly very dependent on the used
 multimedia format.  It is created by the \ttt S-queue! associated
 with a player.
\end{description}
%

\chapter{Transport layer}
\label{chap:2;dsp_book}

\section{Introduction}
\label{sect:2.0;dsp_book}

\medusa- is still in an experimental state. In order to allow for
flexibility in the network structure, we decided to use a thin
meta-transport layer which would make the rest of the software
independent on the chosen transport mechanism.  A the moment, this
chapter is  just a ``collection of notes.''  I still need to ``comb my
ideas.'' 

\section{Internal structure}
\label{sect:2.1;dsp_book}

The thin layer will present to the rest of \medusa- a transport
mechanism that grants that no packet will be received more than once
and with no packet size limit.  Delivery and packet order are not
granted.  The API of the thin layer will not require the allocation of
fixed-size buffers.  The underling protocol will initially be
UDP-based, but maybe we will transition to transport protocols with
rate control, such as DCCP or UDP+TFRC.  Possibly \medusa- will have
its own set of port numbers, mapped to the ports of the underling
transport protocol in some definite way.

\fref{network_layers} shows a possible decomposition of the structure
of the network layer in \medusa-.  The sublayers shown in
\fref{network_layers} will be quite ``thin.''

\begin{figure}
\centerline{\unafiguray[10cm]{network_layers}}
\caption{
\label{fig:network_layers}}
\end{figure}
%
The upper-level API (the API exposed to the rest of \medusa-) will
have the following procedures

\begin{description}
  \item[{\ttt Open\_Source!}]  Open a ``listening point'' from which we
  will receive the data packets
  \item[{\ttt Receive!}] Used to receive data/command packets from
  other peers.
  \item[{\ttt Open\_Destination!}] Open an ``output stream'' used to
  send data to other peers.  
  \item[{\ttt Send\_Data!}] Used to send a data packet to another peer.
  \item[{\ttt Send\_Commnad!}] Used to send a command to another peer.
\end{description}

\subsection{Duplicate elimination}
\label{sub:2.1.1;dsp_book}

This layer limits the size of the packets sent to the lower layer by
possibily fragmenting the packets received by the upper layer.  On the
receiver side it remounts the packets.  Moreover, it take care of
duplicated received packets.  To accomplish such a goal, this layer
will add to the packets an auxiliary 4-byte header

\begin{verbatim}
  typedef struct {
    unsigned reserved         :8;
    unsigned max_fragment_idx :4;  
    unsigned fragment_idx     :4;
    unsigned timestamp        :16;
  }
\end{verbatim}
%
The fragments are numerated with integers $0$, $1$, \ldots, \ttt
max_fragment_idx!.  Clearly, if \ttt max_fragment_idx! is equal to
zero, only one fragment is present.  Since we expect the fragment size
to be 1500 bytes, this allows us to use packets as large as 24000
bytes (this should be plenty).  The \ttt reserved! field is reserved
for future extensions (e.g., for allowing for different ways of
fragmenting the packets) and it should be zero.

\subsection{Bounded-size transport layer}
\label{sub:2.1.0;dsp_book}

Most probably the bounded-size transport layer will be a very thin
layer.  It should expose an API including

\begin{description}
  \item[{\ttt receivefrom!}] Similar to the corresponding function in
  the BSD socket API
  \item[{\ttt sendto!}] Similar to the corresponding function in
  the BSD socket API
\end{description}
%
Since the transport exposed by the bounded-size network layer is
UDP-like, there is no necessity for an \ttt open! function.  Actually,
with DCCP or UDP+TFRC there is a ``session open'' stage, but this
could be hidden inside the \ttt sendto! function (which could
automatically open the session and keep a cache of opened sessions).

When working with pure UDP as transport layer, this layer will be a
very thin one, just mirroring the UDP API.  When using other transport
means (e.g., DCCP or UDP+TFRC) this layer will also take care of all
the added management.  

In \ttt Ada! there are several solutions for making this layer
``flexible'' 

\begin{enumerate}
  \item
    Implement the upper layer in a generic package which accepts in
    its formal parameters the package which implements this layer.
  \item 
    Define an abstract root object and implement each bounded-size
    layer version as a derived type.
\end{enumerate}
%
The former solution could maybe more efficient, but the latter would
allow for having several different layers at the same time.

\end{document}
