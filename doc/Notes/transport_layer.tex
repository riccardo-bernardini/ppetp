%-*- mode: latex; ispell-local-dictionary: "english" -*-
\documentclass{article}

\usepackage{amsmath}
\usepackage{amsbsy}
\usepackage{amssymb}
\usepackage{amsthm,amsfonts}
\usepackage{color}
\usepackage{srcltx}
\usepackage{appendix}
%-*- mode: latex; iso-accents-mode: nil; -*-

%
% Macro definition for medusa_book.  The following definitions have
% been ``cut-and-paste''d from my LaTeX definition file, so every now
% and then there are some macros with italian names.  Maybe one day
% I'll do some renaming.  At least, name clashes become very
% improbable...
%

\newif\ifhaschapter
\makeatletter
\@ifundefined{c@chapter}{\haschapterfalse}{\haschaptertrue}
\makeatother

%
% Shorthand for \texttt, used for source code & simila.  Usually I do
% not love an excessive use of shorthand, but this is too frequent...
%
%\def\ttt#1!{\texttt{#1}}
%
% New version of the above comand which makes '_' category equal to
% ``letter.''  This allows me to write \ttt Multimedia_Timestamp!
% instead of \ttt Multimedia\_Timestamp!
%
\def\ttt{\catcode`\_=12 \tttii}
\def\tttii#1!{{\tt #1}\catcode`\_=8{}}
%
% -- Lettere bold
%
\newcommand{\bfa}{{\mathbf a}}
\newcommand{\bfb}{{\mathbf b}}
\newcommand{\bfc}{{\mathbf c}}
\newcommand{\bfd}{{\mathbf d}}
\newcommand{\bfe}{{\mathbf e}}
\newcommand{\bff}{{\mathbf f}}
\newcommand{\bfg}{{\mathbf g}}
\newcommand{\bfh}{{\mathbf h}}
\newcommand{\bfi}{{\mathbf i}}
\newcommand{\bfj}{{\mathbf j}}
\newcommand{\bfk}{{\mathbf k}}
\newcommand{\bfl}{{\mathbf l}}
\newcommand{\bfm}{{\mathbf m}}
\newcommand{\bfn}{{\mathbf n}}
\newcommand{\bfo}{{\mathbf o}}
\newcommand{\bfp}{{\mathbf p}}
\newcommand{\bfq}{{\mathbf q}}
\newcommand{\bfr}{{\mathbf r}}
\newcommand{\bfs}{{\mathbf s}}
\newcommand{\bft}{{\mathbf t}}
\newcommand{\bfu}{{\mathbf u}}
\newcommand{\bfv}{{\mathbf v}}
\newcommand{\bfw}{{\mathbf w}}
\newcommand{\bfx}{{\mathbf x}}
\newcommand{\bfy}{{\mathbf y}}
\newcommand{\bfz}{{\mathbf z}}
%
\newcommand{\bfA}{{\mathbf A}}
\newcommand{\bfB}{{\mathbf B}}
\newcommand{\bfC}{{\mathbf C}}
\newcommand{\bfD}{{\mathbf D}}
\newcommand{\bfE}{{\mathbf E}}
\newcommand{\bfF}{{\mathbf F}}
\newcommand{\bfG}{{\mathbf G}}
\newcommand{\bfH}{{\mathbf H}}
\newcommand{\bfI}{{\mathbf I}}
\newcommand{\bfJ}{{\mathbf J}}
\newcommand{\bfK}{{\mathbf K}}
\newcommand{\bfL}{{\mathbf L}}
\newcommand{\bfM}{{\mathbf M}}
\newcommand{\bfN}{{\mathbf N}}
\newcommand{\bfO}{{\mathbf O}}
\newcommand{\bfP}{{\mathbf P}}
\newcommand{\bfQ}{{\mathbf Q}}
\newcommand{\bfR}{{\mathbf R}}
\newcommand{\bfS}{{\mathbf S}}
\newcommand{\bfT}{{\mathbf T}}
\newcommand{\bfU}{{\mathbf U}}
\newcommand{\bfV}{{\mathbf V}}
\newcommand{\bfW}{{\mathbf W}}
\newcommand{\bfX}{{\mathbf X}}
\newcommand{\bfY}{{\mathbf Y}}
\newcommand{\bfZ}{{\mathbf Z}}
%
\newcommand{\bfomega}{{\boldsymbol\omega}}
\newcommand{\bflambda}{{\boldsymbol\lambda}}
\newcommand{\bfepsilon}{{\boldsymbol\epsilon}}
\newcommand{\bfdelta}{{\boldsymbol\delta}}
\newcommand{\bfDelta}{{\boldsymbol\Delta}}
\newcommand{\bfalpha}{{\boldsymbol\alpha}}
\newcommand{\bfeta}{{\boldsymbol\eta}}
\newcommand{\bfzero}{{\boldsymbol 0}}
\newcommand{\bfuno}{{\boldsymbol 1}}
\newcommand{\bfell}{{\boldsymbol \ell}}


\newcommand{\fref}[2][.]
{Fig.~\ref{fig:#2}\senondot{#1}{({#1})}}%\fref@ccoda{#2}}
%
% \sedotelse {cond} {then} {else}
%
%     Espande a {then} se cond=., a {else} altrimenti
%
\newcommand{\sedotelse}[3]
    {\def\sedottmpa{#1}\def\sedottmpb{.}%
     \ifx\sedottmpa\sedottmpb{#2}\else{#3}\fi}
%
\newcommand{\senondot}[3][\relax]{\sedotelse{#2}{#1}{#3}}
%
\newcommand{\notainterna}[1]{$<<<$ \emph{#1} $>>>$}
\def\medusa-{{\scshape Medusa}}
\def\medusa-{{\scshape Medu$\chi$a}}
\def\medusa-{{\scshape Medu$\Sigma$a}}
%
% -- Comando \unafigura[dim]{filename}
%   default dim = 10cm
%   se dim = .<n> (<n> intero) dim. standard per tabular con righe
%   di <n> figure
%
\newlength{\unafigstdx}
\setlength{\unafigstdx}{10cm}
\newlength{\unafigstdy}
\setlength{\unafigstdy}{10cm}
\newcommand{\sceglidim}[1]{\ifcase #1 10cm\or 10cm\or 7.5 cm\else 4cm\fi}
\def\stddim#1#2;{\ifx .#1\sceglidim{#2}\else #1#2\fi}
\newcommand{\unafigura}[2][\unafigstdx]
   {\resizebox{\stddim #1;}{!}{\includegraphics{#2}}}
\newcommand{\unafiguray}[2][\unafigstdy]
   {\resizebox{!}{\stddim #1;}{\includegraphics{#2}}}
%\newcommand{\unafigurax}[2][\unafigstdx]{\fbox{file=`{#2}', hsize=\stddim
%#1; (to be included)}}
\newcommand{\unafigurax}[2][10cm]{\framebox[\stddim #1;][c]{file=`{#2}', hsize=\stddim #1; (to be included)}}
\newlength{\pippolen}

%
%
%

%"esempio" = italian word for "example". 
\newtheoremstyle{esempio} 
   {\topsep}{\topsep}{\small\listparindent 1.5em
   \advance \linewidth-2em
   \parshape 1 1em \linewidth}
   {}{\hspace{-1em}\itshape}{}{\newline}
   {\thmname{#1}\thmnumber{ #2}\thmnote{ (#3)}}

\theoremstyle{esempio}

\ifhaschapter
\newtheorem{example}{Example}[chapter]
\newtheorem{commento}{Remark}[chapter]
\else
\newtheorem{example}{Example}[section]
\newtheorem{commento}{Remark}[section]
\fi
\newcommand{\soprasotto}[2]{\mathrel{\mathop{#1}\limits_{#2}}}
\newcommand{\commenta}[2]{\soprasotto{\underbrace{#1}}{#2}}
\newcommand{\er}[1]{(\ref{#1})}

%
% List of remarks (\`a la "Ten lectures...")
%
\newenvironment{remarks}
{\par
  \normalfont
  \textbf{Remarks.}
  \begin{enumerate}}
{\end{enumerate}}
\newcommand{\remark}{\item}

\newcommand{\norma}[2][]{\|{#2}\|_{#1}}
\let\norm\norma
\newcommand{\normainf}[1]{\norma[\infty]{#1}}
\newcommand{\abs}[1]{\lvert{#1}\rvert}
\newcommand{\floor}[1]{\left\lfloor{#1}\right\rfloor}
\newcommand{\ceil}[1]{\lceil{#1}\rceil}
\newcommand{\round}[1]{\lceil{#1}\rfloor}
%%%%%%%%%%%%%%%%%%%%%%%%%%%%%%%%%%%%%%%%%%%%%%%%%%%%%%%%%%%%%%%%%%%%%%
\newcommand{\gf}[1]{\operatorname{GF}(#1)}

\newcommand{\redfac}{R}
\newcommand{\redroot}{b}
\newcommand{\redvec}{\bfr}
\newcommand{\reduced}{\bfu}
\newcommand{\redmtx}{\bfR}
\newcommand{\npeers}{N}
\newcommand{\peercap}{M}

\newcommand{\peer}[1]{P_{#1}}
\newcommand{\idvec}[1]{\operatorname{ID}_{#1}}
\newcommand{\locap}{C_{\text{lo}}}
\newcommand{\hicap}{C_{\text{hi}}}
\newcommand{\defect}{D}
\newcommand{\excess}{E}

\newenvironment{API}
               {\list{}{\labelwidth 0pt \itemindent-\leftmargin
                        \let\makelabel\functionlabel}}
               {\endlist}
\newcommand*\functionlabel[1]{\hspace\labelsep
                                \tt\bfseries {#1}}
\def\function{\catcode`\_=12 \funzioneii}
\newcommand\funzioneii[1]{\item[{#1}]\catcode`\_=8}

\def\ppmtp;{PPETP}
\def\ppetp-{PPETP}
\def\ppc-{ETCP}
\newcommand{\vriv}[5][cccc]
{\left[
  \begin{array}{#1}
  {#2} &{#3} &{#4}&{#5} \\
  \end{array} 
\right]}
\newcommand{\vciv}[5][c]
  {\left[
     \begin{array}{#1} 
      {#2} \\ 
      {#3} \\
      {#4} \\
      {#5}
     \end{array}
   \right]}
\newcommand{\perdef}{:=}
\def\marker-{\ttt Marker!}
\DeclareTextCommandDefault{\adarange}
   {{.\kern\fontdimen3\font .\kern\fontdimen3\font}}


%
\title{Notes for a P2P transport layer}
\author{R. Bernardini}
%
\setcounter{secnumdepth}{3} 
\setcounter{tocdepth}{3}
\begin{document}
  \maketitle
\section{Why this document?}
It came to my mind that \medusa- can be transformed in a
pseudo-transport layer which can be used to transmit any sequence of
packets.  The idea is that \medusa- could be used as an
almost-one-to-one replacement for UDP in order to exploit any
multimedia-format-over-UDP already available.  For example, by
transmitting RTP/AVP packets over \medusa- one could obtain
RTP/AVP/\medusa- protocol.

This document is just a random collection of notes about this idea.

\section{Overview}
\label{sect:0;transport_layer}

The procedure is something like this:

\begin{enumerate}
\item
A node wants to receive a stream of packets over \medusa-.  To such an
end, it contacts the packet source and ask a stream description.
\item
The node creates a new session over \medusa- and initializes it by
giving it the stream description received by the server.
\item
If the node wants to receive data from a peer, it calls a function
\ttt pull_from!
\item
If the node wants to transmit data to another perr, it calls \ttt
push_to! 
\item
The node will read received packets via a \ttt recv! function
\item
The node will transmit packets via a \ttt send! function
\item 
The node can create several output streams
\end{enumerate}
%
Let us try to make an API

\begin{API}
  \function{New_Session(Description : String)} create a new \medusa-
  session and init it with the \ttt Description! string (see
  Appendix~\ref{sect:1;transport_layer}).  Return an handler to the
  new session.

  \function{Push_To(Session, IP_Address, Port, Output := 0)} Send a
  the reduced stream number \ttt Output! to the peer with \ttt
  IP_Address! and \ttt Port!

  \function{Pull_From(Session, IP_Address, Port, Auth)} Contact on
  \ttt Port! the peer wiht \ttt IP_Address! and ask it a new reduced
  stream. 

  \function{New_Reduced_Stream(Session)} Create a new reduced
  stream. Return the stream number.

  \function{Close_Reduced_Stream(Session, Stream)} Close a reduced
  output stream.

  \function{Register_Event_Callback(Session, Callback)} Register a
  function to be called in case of exceptional events (i.e., a
  non-responding node)

  \function{Send(Session, Packet)} Send a new packet over the network

  \function{Recv(Session, Packet)} Wait for a new packet to arrive

  \function{Shutdown(Session)} Close the session.

  \function{Command_Port(Session)} Return the port the Session is
  listening for commands (used by \ttt Push_To! and \ttt Pull_From!)
\end{API}

Slightly more detail procedure:

\newsavebox\dataexchangebox
\newcommand{\dataexchange}[2]{
\savebox{\dataexchangebox}{\parbox[t]{50em}{\tt #2}}
\begin{trivlist}
\item
  \begin{tabular}{p{4em}p{50em}}
    \tt #1 & \usebox{\dataexchangebox}
  \end{tabular}
\end{trivlist}
}

\dataexchange{C->W}{GET /concert.sdp HTTP/1.1}
\dataexchange{W->C}{HTTP/1.1 200 OK\\
\ldots\\
m=video \ldots\\
a=control:rtsp://www.example.com/video\\
m=audio \ldots\\
a=control:rtsp://www.example.com/audio}

Here the client open a new \medusa- session.  \medusa- open a new
command port (3456).  The client retrieves the command port and pass
it to the server.  When the session was opened, the server IP was
given to the opening function.

\dataexchange{C->S}{SETUP rtsp://www.example.com/video RTSP/1.0\\
Transport: RTP/AVP/MTP 3456\\
}

Now the server sends to the command port the stream description.

\dataexchange{S->C (3456)}{CONFIGURE\\
\\
<stream>\\
<packet-class>\\
\ldots\\
</packet-class>\\
</stream>}
\dataexchange{C->S}{200 OK}

The client configured the \medusa- layer successfully.  Now the server
can reply to the \ttt SETUP! request.

\dataexchange{S->C}{RTSP/1.0 200 OK\\
Transport: RTP/AVP/MTP\\
Seed: ABC444\\
MTP-Info: peer-list="123.11.12.10:3567:AEF468,156.23.34.1:6234:AEF468"}

In this case the server replies with a list of peers.  The node will
give the addresses in the peer-list to the \ttt Pull_From! procedure
which will send something as

\dataexchange{C->P}{SEND\\
Destination: 123.234.1.2/3333 3567\\
Auth: AEF468}
\dataexchange{P->C}{200 OK}

The value of field \ttt Auth! is created by hashing the \ttt Seed!
value that the server gave to the peer and the destination IP/port. 

\section{Other random ideas}
\label{sect:2;transport_layer}

Maybe the best approach is to split the \medusa- transport protocol
into two levels: a basic level which, more or less, coincides with the
DSP module and an upper level which controls the lower one.  The upper
level is not strictly necessary, but it could prove convenient.

\subsection{Lower level model}
\label{sub:2.0;transport_layer}

\medusa-, as TCP, has \emph{connections}.  From the point of view of a
node, a \medusa- connection has 

\begin{itemize}
  \item
    One or more \emph{input ports} (which correspond to UDP ports)
  \item
    One or more \emph{output channels} (each channel correspond to a
    reduction choice)
  \item
    Each channel has one or more \emph{output ports} (each output port
    corresponds to a destination peer)
\end{itemize}
%
Every \medusa- connection must be configured by assigning it a
\emph{profile} which describes the composition of the stream
transmitted over \medusa-.  This makes the \medusa- transport protocol
quite peculiar since it must ``know'' something about the type of data
which are transmitted.

A profile must specify

\begin{itemize}
  \item 
    The classes of packets in the stream
  \item
    For every class of packet, 
    \begin{itemize}
      \item
        How to recover the packet from its components (for the source
        it is also necessary to specify how a packet is split into its
        components) 
      \item
        The components contained in a packet
        and, more precisely,
        \begin{itemize}
          \item
            If the component is analog or binary
          \item
            Which encoder/decoder is to be used to encode the component
        \end{itemize}
    \end{itemize}
\end{itemize}
%
Profiles will be described by means of XML syntax.  A \medusa- API
could have a configuration function which accepts XML profile
description as strings.  One could object that XML parsing could
require time, but since it is done only at configuration time, we can
afford it.  It is reasonable to assume that in future we could have
also profiles for packet classes and components, although this is not
strictly necessary.

At least one profile is defined

\begin{verbatim}
  <stream-profile name="basic">
    <packet>
      <component>
        <analog/>
        <encoder>trivial</encoder>
      <component/>
      <join>trivial</join>
    </packet>
  </stream-profile>
\end{verbatim}

Profile \ttt basic! corresponds to a stream of binary packets
considered as single binary components.  Profile \ttt basic! is a
catch-all profile and it can be used with any type of data.

Note that the profile does not specify how the components are reduced.
This because the profile gives the \emph{structure} of the transmetted
stream, while the reduction details are a ``private'' matter of the
node. Of course, some parameters, such as the reduction factor, can be
enforced or suggested by some entity external to the node (e.g., the
server in a live streaming application).  Nevertheless, the reduction
details are not part of the stream structure.  The reduction details
(reduction factor and reduction vector) will be typically included in
the transmitted packet or determined by other means.

The lower level layer will expose the following API

\begin{API}
  \function{New_Connection} Create a new connection
  \function{New_Input(Connection)} Open a new input port
  \function{Set_Timeout(Input)} Set a timeout over an input port.  A
  new event will be raised if the time between packets is greater than
  the timeout.
  \function{New_Channel(Connection)} Open a new channel (with its own
  reduction details)
  \function{New_Destination(Channel, Address)} Associate a new
  destination to channel \ttt Channel!
  \function{Close_Destination(Destination)}
  \function{Close_Channel(Channel)}
  \function{Close_Input(Port)}
  \function{Close_Connection(Connection)}
  \function{Receive(Connection, Multimedia_Packet)} Wait for a new multimedia
  packet to arrive
  \function{Send(Connection, Medusa_Packet)} Send a new packet over
  the connection.  Note that while \ttt Receive! returns a multimedia
  packet, \ttt Send! expects a packet in the \medusa- format, i.e., a
  packet already processed by the splitter function.  
  \function{Register_Event_Handler} Register a callback function to be
  called whenever an event happens.  If convenient, the API could have
  different \ttt Register_*! functions for different types of events.
\end{API}
%
Note that in the API above there is no way to \emph{ask} to another
peer for data (i.e., there is no \ttt Pull_From!) method.  This
because the \ttt Pull_From! procedure will be implemented by means of
procedures external to the lower layer.

\subsection{Higher layer}
\label{sub:2.1;transport_layer}

The higher layer is a control layer which is not strictly necessary
for the application.  Nevertheless, by specifying it we offer a
standard way to control the lower layer.

The higher layer protocol proposed in this document is based on a
RTSP-like (or HTTP-like) structure, that is, the typical message has
the format

\begin{verbatim}
  message      = start-line 
                 *(message-header CRLF)
                 CRLF
                 [ message-body ]

  start-line   = request-line | reply-line
  request-line = method  SP URI SP version
  reply-line   = version SP status-code SP Reason CRLF
  version      = <protocol name. To be defined>
\end{verbatim}

The methods available are

\begin{itemize}
  \item \ttt SEND!
  \item \ttt OPTIONS!
\end{itemize}

\subsubsection{\ttt SEND! method}
\label{subsub:2.1.0;transport_layer}

Used to ask a peer to send a stream to another peer.  The address of
the destination peer is transmitted in the Header section. The header
section could include an authorization field in order to be sure that
the request is legitimate.  

\begin{example}
  Consider, for example, the case of a live event streaming.  In such
  a case the user could connect to a central server which holds the
  connection informations about the streaming.  The connection can be
  made via https, in order to be sure of the server identity.  The
  server can require the user to authenticate itself.   

  The server generates a random seed that sends to the user with the
  other session informations.  When a user asks for a  possible
  peer, the server reply with two values: the peer address (typically
  IP address + port) and an ``authorization'' value obtained by
  hashing the user address, the peer address, the random seed
  which was sent to the peer and the number of streams that the user
  is authorized to receive from the peer.  The node which receives the
  SEND command does a similar hash and check that the two values are
  equal. 

  If this type of protection was not present, several threats would be
  possible 
  \begin{enumerate}
    \item A user could ask for a large number of streams, saturating
    the peer output bandwidth, causing a DoS.
    \item The number of stream a user is entitled to receive could
    depend on the subcription fee payed by the user.  A user could
    subscribe at the lowes fee, then ask for more streams.
    \item Two users could collaborate to pay only one subscription.
    To such an end, one user subscribes and connect to the central
    server.  After obtaining the peer infos, the user ask to each node
    to send data both to the user itself and to its friend (which
    receive the data without paying and without using output
    bandwidth).
   \item A user could ask to a peer to send many streams to, say, the
   DNS port of a DNS server, causing, possibly, a DoS.
  \end{enumerate}
\end{example}

\subsubsection{OPTIONS method}
\label{sub:2.2;transport_layer}

Used to configure the node.  It could be used to transmit, for
example, the stream profile and to fix the number of output channels,
the number of output ports and the reduction factor corresponding to
each pair (component, channel).

\appendix

\section{Stream description format}
\label{sect:1;transport_layer}

In this section we formally describe the XML format used to describe
the stream structure.  The description is given in \ttt RELAX
NG!\footnote{http://www.relaxng.org/} 
compact syntax \cite{Relax-compact}


\begin{verbatim}
# Starting node
start = Stream

#
# A stream is collection of optional profile definitions
# followed by one or more packet class definitions
#
Stream = element stream { 
   attribute profile { NMTOKEN } |
   ( attribute name { NMTOKEN }?,
     Definition *,
     Packet + )
}
#
# Example of declaration of a stream with a well-known
# profile
#
#  <stream profile="basic" />
#
#
# Example of a possible MPEG-like stream.  It is supposed
# that the packet profiles 'MPEG:I',  'MPEG:P'
# are well-known, while the definition for B packets is contained
# in a remote resource.
#
#  <stream name='mpeg'>
#    <packet profile='MPEG:I' />
#    <packet profile='MPEG:P' />
#    <packet url='http://www.example.com/profiles/MPEG_B.xml' />
#  </stream>
#

#
# A definition can define both a packet profile,  a component
# profile or a procedure.
#
Definition = element define {
    attribute name { NMTOKEN },
    (Packet | Component | Procedure)
}
#
# Example:
# <define name='MDX:basic'>
#   <component>
#     <binary />
#     <encoder name='MDX:trivial' />
#   </component>
# </define>
#
#

#
# A packet class is specified by giving its 
# profile name or by specifying its components
# and how to recover the packet from the 
# components
#
Packet = element packet {
   attribute profile { NMTOKEN } |
   attribute url     { anyURI }  |
   (Component +, Join, Split?)
}
#
# Example of definition of a packet with two components.
# The first component is of analog type and it must be
# encoded with a DCT-based encoder.  The second component
# is a binary one and it has the profile MDX:basic defined
# above.  In order to merge the two components in a single
# multimedia packet, the node must use the Ruby procedure
# stored in glue.rb.
#
# <packet>
#   <component>
#     <analog />
#     <encoder name='DCT' />
#   </component>
#   <component profile='MDX:basic' />
#   <join>
#     <procedure language='ruby'> 
#        <external>http://www.example.com/glue.rb</external>
#     </procedure>
#   </join>
# </packet>
# 

Join = element join {
   attribute name { NMTOKEN } |
   Procedure
}

Split = element split {
   attribute name { NMTOKEN } |
   Procedure
}

#
# A component is defined by specifying its profile or
# by declaring its type (Analog or Binary) and the
# corresponding encoder
#
Component = element component {
   attribute profile { NMTOKEN } |
   attribute url     { anyURI }  |
    (Component_Type, Encoder, Decoder ?)
}

Component_Type = element analog | binary { empty }

Encoder = element encoder {
   attribute name { NMTOKEN } |
   Procedure
}

Decoder = element decoder {
   attribute name { NMTOKEN } |
   Procedure
}

#
# Procedure definition.  In order to specify a procedure it is
# necessary to specify how to interpret it (with the attribute
# language) and its code.  The code can be specified by 
# inserting it "inline" (in BASE64 format) or by giving an 
# URL to access it.
#
Procedure = element procedure {
      attribute language { NMTOKEN },
      (Inline_Code | External_Code)
}

Inline_Code   = element inline   { base64Binary }
External_Code = element external { anyURI }
#
# Example:
# <procedure language='javascript'>
#    <external>http://www.example.com/src/encoder.js</external>
# </procedure>
#
\end{verbatim}


\bibliographystyle{alpha}
\bibliography{medusa_biblio}

\end{document}
