%-*- mode: latex; ispell-local-dictionary: "english" -*-
\documentclass{medusabook}

\usepackage{amsmath}
\usepackage{amsbsy}
\usepackage{amssymb}
\usepackage{amsthm,amsfonts}
\usepackage{color}
\usepackage{srcltx}
\usepackage{appendix}
%-*- mode: latex; iso-accents-mode: nil; -*-

%
% Macro definition for medusa_book.  The following definitions have
% been ``cut-and-paste''d from my LaTeX definition file, so every now
% and then there are some macros with italian names.  Maybe one day
% I'll do some renaming.  At least, name clashes become very
% improbable...
%

\newif\ifhaschapter
\makeatletter
\@ifundefined{c@chapter}{\haschapterfalse}{\haschaptertrue}
\makeatother

%
% Shorthand for \texttt, used for source code & simila.  Usually I do
% not love an excessive use of shorthand, but this is too frequent...
%
%\def\ttt#1!{\texttt{#1}}
%
% New version of the above comand which makes '_' category equal to
% ``letter.''  This allows me to write \ttt Multimedia_Timestamp!
% instead of \ttt Multimedia\_Timestamp!
%
\def\ttt{\catcode`\_=12 \tttii}
\def\tttii#1!{{\tt #1}\catcode`\_=8{}}
%
% -- Lettere bold
%
\newcommand{\bfa}{{\mathbf a}}
\newcommand{\bfb}{{\mathbf b}}
\newcommand{\bfc}{{\mathbf c}}
\newcommand{\bfd}{{\mathbf d}}
\newcommand{\bfe}{{\mathbf e}}
\newcommand{\bff}{{\mathbf f}}
\newcommand{\bfg}{{\mathbf g}}
\newcommand{\bfh}{{\mathbf h}}
\newcommand{\bfi}{{\mathbf i}}
\newcommand{\bfj}{{\mathbf j}}
\newcommand{\bfk}{{\mathbf k}}
\newcommand{\bfl}{{\mathbf l}}
\newcommand{\bfm}{{\mathbf m}}
\newcommand{\bfn}{{\mathbf n}}
\newcommand{\bfo}{{\mathbf o}}
\newcommand{\bfp}{{\mathbf p}}
\newcommand{\bfq}{{\mathbf q}}
\newcommand{\bfr}{{\mathbf r}}
\newcommand{\bfs}{{\mathbf s}}
\newcommand{\bft}{{\mathbf t}}
\newcommand{\bfu}{{\mathbf u}}
\newcommand{\bfv}{{\mathbf v}}
\newcommand{\bfw}{{\mathbf w}}
\newcommand{\bfx}{{\mathbf x}}
\newcommand{\bfy}{{\mathbf y}}
\newcommand{\bfz}{{\mathbf z}}
%
\newcommand{\bfA}{{\mathbf A}}
\newcommand{\bfB}{{\mathbf B}}
\newcommand{\bfC}{{\mathbf C}}
\newcommand{\bfD}{{\mathbf D}}
\newcommand{\bfE}{{\mathbf E}}
\newcommand{\bfF}{{\mathbf F}}
\newcommand{\bfG}{{\mathbf G}}
\newcommand{\bfH}{{\mathbf H}}
\newcommand{\bfI}{{\mathbf I}}
\newcommand{\bfJ}{{\mathbf J}}
\newcommand{\bfK}{{\mathbf K}}
\newcommand{\bfL}{{\mathbf L}}
\newcommand{\bfM}{{\mathbf M}}
\newcommand{\bfN}{{\mathbf N}}
\newcommand{\bfO}{{\mathbf O}}
\newcommand{\bfP}{{\mathbf P}}
\newcommand{\bfQ}{{\mathbf Q}}
\newcommand{\bfR}{{\mathbf R}}
\newcommand{\bfS}{{\mathbf S}}
\newcommand{\bfT}{{\mathbf T}}
\newcommand{\bfU}{{\mathbf U}}
\newcommand{\bfV}{{\mathbf V}}
\newcommand{\bfW}{{\mathbf W}}
\newcommand{\bfX}{{\mathbf X}}
\newcommand{\bfY}{{\mathbf Y}}
\newcommand{\bfZ}{{\mathbf Z}}
%
\newcommand{\bfomega}{{\boldsymbol\omega}}
\newcommand{\bflambda}{{\boldsymbol\lambda}}
\newcommand{\bfepsilon}{{\boldsymbol\epsilon}}
\newcommand{\bfdelta}{{\boldsymbol\delta}}
\newcommand{\bfDelta}{{\boldsymbol\Delta}}
\newcommand{\bfalpha}{{\boldsymbol\alpha}}
\newcommand{\bfeta}{{\boldsymbol\eta}}
\newcommand{\bfzero}{{\boldsymbol 0}}
\newcommand{\bfuno}{{\boldsymbol 1}}
\newcommand{\bfell}{{\boldsymbol \ell}}


\newcommand{\fref}[2][.]
{Fig.~\ref{fig:#2}\senondot{#1}{({#1})}}%\fref@ccoda{#2}}
%
% \sedotelse {cond} {then} {else}
%
%     Espande a {then} se cond=., a {else} altrimenti
%
\newcommand{\sedotelse}[3]
    {\def\sedottmpa{#1}\def\sedottmpb{.}%
     \ifx\sedottmpa\sedottmpb{#2}\else{#3}\fi}
%
\newcommand{\senondot}[3][\relax]{\sedotelse{#2}{#1}{#3}}
%
\newcommand{\notainterna}[1]{$<<<$ \emph{#1} $>>>$}
\def\medusa-{{\scshape Medusa}}
\def\medusa-{{\scshape Medu$\chi$a}}
\def\medusa-{{\scshape Medu$\Sigma$a}}
%
% -- Comando \unafigura[dim]{filename}
%   default dim = 10cm
%   se dim = .<n> (<n> intero) dim. standard per tabular con righe
%   di <n> figure
%
\newlength{\unafigstdx}
\setlength{\unafigstdx}{10cm}
\newlength{\unafigstdy}
\setlength{\unafigstdy}{10cm}
\newcommand{\sceglidim}[1]{\ifcase #1 10cm\or 10cm\or 7.5 cm\else 4cm\fi}
\def\stddim#1#2;{\ifx .#1\sceglidim{#2}\else #1#2\fi}
\newcommand{\unafigura}[2][\unafigstdx]
   {\resizebox{\stddim #1;}{!}{\includegraphics{#2}}}
\newcommand{\unafiguray}[2][\unafigstdy]
   {\resizebox{!}{\stddim #1;}{\includegraphics{#2}}}
%\newcommand{\unafigurax}[2][\unafigstdx]{\fbox{file=`{#2}', hsize=\stddim
%#1; (to be included)}}
\newcommand{\unafigurax}[2][10cm]{\framebox[\stddim #1;][c]{file=`{#2}', hsize=\stddim #1; (to be included)}}
\newlength{\pippolen}

%
%
%

%"esempio" = italian word for "example". 
\newtheoremstyle{esempio} 
   {\topsep}{\topsep}{\small\listparindent 1.5em
   \advance \linewidth-2em
   \parshape 1 1em \linewidth}
   {}{\hspace{-1em}\itshape}{}{\newline}
   {\thmname{#1}\thmnumber{ #2}\thmnote{ (#3)}}

\theoremstyle{esempio}

\ifhaschapter
\newtheorem{example}{Example}[chapter]
\newtheorem{commento}{Remark}[chapter]
\else
\newtheorem{example}{Example}[section]
\newtheorem{commento}{Remark}[section]
\fi
\newcommand{\soprasotto}[2]{\mathrel{\mathop{#1}\limits_{#2}}}
\newcommand{\commenta}[2]{\soprasotto{\underbrace{#1}}{#2}}
\newcommand{\er}[1]{(\ref{#1})}

%
% List of remarks (\`a la "Ten lectures...")
%
\newenvironment{remarks}
{\par
  \normalfont
  \textbf{Remarks.}
  \begin{enumerate}}
{\end{enumerate}}
\newcommand{\remark}{\item}

\newcommand{\norma}[2][]{\|{#2}\|_{#1}}
\let\norm\norma
\newcommand{\normainf}[1]{\norma[\infty]{#1}}
\newcommand{\abs}[1]{\lvert{#1}\rvert}
\newcommand{\floor}[1]{\left\lfloor{#1}\right\rfloor}
\newcommand{\ceil}[1]{\lceil{#1}\rceil}
\newcommand{\round}[1]{\lceil{#1}\rfloor}
%%%%%%%%%%%%%%%%%%%%%%%%%%%%%%%%%%%%%%%%%%%%%%%%%%%%%%%%%%%%%%%%%%%%%%
\newcommand{\gf}[1]{\operatorname{GF}(#1)}

\newcommand{\redfac}{R}
\newcommand{\redroot}{b}
\newcommand{\redvec}{\bfr}
\newcommand{\reduced}{\bfu}
\newcommand{\redmtx}{\bfR}
\newcommand{\npeers}{N}
\newcommand{\peercap}{M}

\newcommand{\peer}[1]{P_{#1}}
\newcommand{\idvec}[1]{\operatorname{ID}_{#1}}
\newcommand{\locap}{C_{\text{lo}}}
\newcommand{\hicap}{C_{\text{hi}}}
\newcommand{\defect}{D}
\newcommand{\excess}{E}

\newenvironment{API}
               {\list{}{\labelwidth 0pt \itemindent-\leftmargin
                        \let\makelabel\functionlabel}}
               {\endlist}
\newcommand*\functionlabel[1]{\hspace\labelsep
                                \tt\bfseries {#1}}
\def\function{\catcode`\_=12 \funzioneii}
\newcommand\funzioneii[1]{\item[{#1}]\catcode`\_=8}

\def\ppmtp;{PPETP}
\def\ppetp-{PPETP}
\def\ppc-{ETCP}
\newcommand{\vriv}[5][cccc]
{\left[
  \begin{array}{#1}
  {#2} &{#3} &{#4}&{#5} \\
  \end{array} 
\right]}
\newcommand{\vciv}[5][c]
  {\left[
     \begin{array}{#1} 
      {#2} \\ 
      {#3} \\
      {#4} \\
      {#5}
     \end{array}
   \right]}
\newcommand{\perdef}{:=}
\def\marker-{\ttt Marker!}
\DeclareTextCommandDefault{\adarange}
   {{.\kern\fontdimen3\font .\kern\fontdimen3\font}}


%
\titleimage{chestCN9697}% clarita1000@gmail.com
\title{Medusa internals}
\author{R. Bernardini}
%
\setcounter{secnumdepth}{3} 
\setcounter{tocdepth}{3}
\begin{document}
  \maketitle
\tableofcontents
\chapter{Introduction (read me, please!)}

\section{Why this document?}
\label{sect:0;medusa_book}

The goal of this book is to document the internal details of
\medusa-. Yes, I know, you already guessed that by reading the
title\ldots{} (if you didn't, you can stop reading here).  Maybe it is
worth spending a few words about \emph{why} I felt the need to write
this document.

First, \emph{what is} \medusa-? \medusa- is a project started at the
University of Udine (Italy).  The goal of \medusa- is to develop an
open structure for efficient peer-to-peer (P2P) \emph{live streaming}.
I am not going into details here why this is a cool idea and which
kind of problems must be solved.  The important point that I want to
emphasize here is that \medusa- is quite a complex piece of software
that (I hope) will be maintained in the future by other developers
(beside me).  Therefore, we need some documentation for those future
developers.

Let me say that I already modified open source software (sometimes for
necessity, sometimes for fun).  My experience is that when you are
trying to change a piece of software wrote by someone else, you do not
need a verbose description of every single variable and line of code,
but rather the ``big picture'' of the software.  Instead, often you
find code which looks like (no kidding)

\begin{verbatim}
    int i, j;  /* Loop variables */       (really?!?)
    count ++;  /* Increment count by 1 */ (Wow !)
\end{verbatim}

\noindent
but no description of the ``high level structure'' of the software and
you are left to yourself in finding the part you need to
change\footnote{That can also be fun, mind you, unless you are in a
rush\ldots}.  It is like entering a wood with a description of every
single leaf, but no map.  Well, this document is the ``wood map.''

\paragraph{A little warning}
Currently (14th April 2008) \medusa- is still in a development stage
and there is not a complete one-to-one match between this document and
the current state of the software.  However, this document is a
reliable information source about the \emph{future} state of the
software.  (Remember that this document is aimed to future developers.)

\section{Description levels}
\label{sect:0.0;medusa_book}

As I said, \medusa- is quite a complex piece of software.  The
\medusa- suite is made of a client and a server, both are made of
\emph{modules} that communicate among themselves, each module in turn
is obtained by composing several libraries.  As you see, there is a
lot to be documented\ldots

Our approach will be to describe the internal structure of \medusa-
going from the coarser level down to the most detailed one.  More
specifically, we will talk about

\begin{itemize}
\item
Motivation: why video streaming over P2P is such a cool idea and which
problems (\emph{evanescence} and \emph{asymmetric bandwidth}) must be
solved
\item
An overview of the algorithms used in \medusa- to 
\begin{itemize}
  \item
    Solve the evanescence and asymmetric bandwidth problem
  \item
    Manage the P2P structure
\end{itemize}
\item
A description of the communication protocol between peers and between
the peers and the server
\item
A description of the internal modules of client and server
\item
A description of the communication protocol between the internal
modules. 
\item
A high-level description of a possible internal structure of the most
complex modules.
\end{itemize}
%
\ldots and we stop here\ldots{} Further details would describe the
actual implementation of each library composing each module, but we
demand this level of detail to the single module documentation that
you will find in the \texttt{doc} directory of the module (Yes! every
module implementation has [will have] its own \texttt{doc} directory).
Actually, in the current state some module has more than one
implementation (typically one implementation is just a very primitive
version of the module used for debugging purposes) and every
implementation has its own internal details.


\chapter{Preliminary remarks}

%\section{Motivation and problem statement}
%\label{sect:1;medusa_book}

\section{Motivation}
\label{sub:1.0;medusa_book}

I could say that to know why P2P is an interesting solution to video
streaming you just need to give a look to the introduction of papers
in the literature \cite{deshpande01:_live_media_p2p} \cite{Octoshape}
\cite{hei08:_iptv_p2p} \cite{liu08:p2p}.  Very
briefly, the idea is that if you want to broadcast a video stream to a
group of users via many unicast connections, you need a bandwidth
equal to the video bandwidth multiplied by the number of users.  An
alternative solution is to use IP multicast, but such a solution is
not readily available for a number of reasons \cite{liu08:p2p}.  A
solution which could be scalable and does not rely on network
infrastructure (differently from the IP multicast solution) is a P2P
approach where every new user behaves also as a server, replying the
received signal.  In a sense, the P2P approach could be described as
an ``application level multicast.''


\section{Problems in P2P streaming}
\label{sub:1.1;medusa_book}

So, P2P is the ultimate solution to video streaming? Alas, no.  The
fact that intermediate nodes are home PC and not real servers
introduces some problems which must be solved in order to make P2P
video streaming practical.

\paragraph{Evanescence}
The \emph{evanescence} problem is due to the fact that we cannot
control how long a home user will remain connected.  If a node
suddenly leaves, the nodes which were receiving data from it will
receive no data anymore.  Although in a ``classical'' file-sharing P2P
context this would cause only a delay in the download, in a video
streaming application sudden departure of a node can cause an
unacceptable loss of quality.

\paragraph{Asymmetric Bandwidth}
Typically a home user, connected via DSL to the Internet, has enough
\emph{download} bandwidth to receive video data, but not enough
\emph{upload} bandwidth to act as a video server.  A typical solution
to the \emph{asymmetric bandwidth} problem is to split the file into
pieces (called \emph{chunks}) and let a node download different chunks
from different peers.\footnote{This description is maybe a little
oversimplified, but it suffices in our context.  Refer to the
literature for more details \cite{hei08:_iptv_p2p} \cite{liu08:p2p}.}
The \emph{chunky} approach works fine in a file-sharing context, but
its application is more delicate in the video streaming context
because video data has an ``expiration date'' (i.e., play-out time)
after which the data is useless \cite{liu08:p2p}.

To solve this problem \medusa- introduces a \emph{chunkless} approach
described in more detail in Section~\ref{sect:1.0;medusa_book} and
\cite{bernardini08:dcc08}. 

\paragraph{NAT and firewalls}
Most of home PC are assigned private IP address by the Internet
Service Provider (ISP). Such private address needs Network Address
Translation (NAT) when the home PC must interact with hosts outside
the ISP network.  The problem introduced by NAT in a P2P context is
that NAT makes home PC unreachable from nodes which do not belong to
the same subnetwork unless \emph{hole punching} techniques are
employed \cite{UDP_hole}.

Currently the NAT problem is not addressed in \medusa- (but it will
be).

\paragraph{Security}
Relying on home users for distributing video data opens some security
problems \cite{hei08:_iptv_p2p}.  Here we cite only the \emph{stream
poisoning} threat where a malicious user sends random packets instead
of video data.  Such packets will cause quality degradation and,
possibly, will be propagated by other nodes.  The chunkless technique
described in Section~\ref{sect:1.0;medusa_book} can take care of this
problem. 


\section{Differences between video streaming and file-sharing}
\label{sub:1.2;medusa_book}

It is worth to emphasize here the main differences between  video
streaming and file-sharing, with respect to the P2P structure.

\begin{description}
  \item[Existence of a central node] In a file-sharing application one
  would like to have a structure that is as distributed as possible,
  since the requirement of a set of central nodes introduce a single
  point of failure in the system.  In a video streaming application
  there is a ``natural'' central node, that is, the host(s) of the
  entity which wants to broadcast the video.

  The existence of a central node simplifies several aspect of P2P
  networks such as indexing and network management.
%
  \item[Data timeliness] In a file-sharing application the chunks of
  data have no expiration time and a lowering of the bandwidth and/or
  packet loss causes only longer download times.  In a video streaming
  application data have an expiration time and any data lost or
  arrived late can give rise to a (often unacceptable) quality
  degradation.
\end{description}

\section{The \medusa- multimedia model}
\label{sect:1.6;medusa_book}

Our aim is to produce a general ``P2P streaming building block'' which
can be used in a wide range of applications. The ``reduction''
algorithm described in Section~\ref{sect:1.0;medusa_book} and the
``Processing module'' included in the user program (see
\fref{peer-internal}) are designed to be independent on the specific
format.  It is worth to briefly describe the model I had in mind when
I wrote these notes.

\begin{description}
  \item[The content] The user is interested in some \textbf{multimedia
content} which is a collection of one or more \textbf{multimedia
constituents}.  A \emph{constituent} is roughly defined as the data
that one would feed as input to a player. Another possible definition
is that a constituent is something you can assign a MIME type to. For
example, a remote classroom application can have two constituents: one
for the video showing the teacher and one for a vectorial
representation of the slides used by the teacher.

Note that if the video is encoded with a multiple description coder,
the descriptions \emph{are not} constituents in the sense used here,
since it is quite probable that all the descriptions will be feeded to
a single (MD-enabled) player.  A similar remark applies also to the
different resolutions in a scalable streams.
\item[The stream] A constituent is a collection of
  \textbf{streams}. Each stream is identified by a small
  integer called \textbf{stream ID}. Examples of constituents
  containing more than one stream are the output of scalable or
  multiple description coders.  
\item[The stream] Each stream is a sequence of \textbf{packets}. Each
  packet in a stream is identified by a \textbf{timestamp}.  It is not
  strictly necessary that there is a linear relationship between
  packet timestamp and ``playing time'' (in seconds) although it is
  reasonable to think that this will often be true.  The pair made of
  the timestamp and the stream ID is the \textbf{packet ID}.  

  Every packet carries also an \textbf{importance} field.  This field
  can prove useful when the output bandwidth of a node is smaller than
  its ``nominal'' value.  In such a case the node can decide to drop
  the least important packets.  The importance field is wrote by the
  content source on the basis, for example, of the additional
  distorsion that would be experienced in case the packet was missed.
  For example, in a scalable stream one could assign a smaller
  importance to the descriptions carrying information about the finer
  details in order to give precedence to the base layer.
\item[The packet] A packet can be further divide into one or more
  \textbf{components} and each component can be an \emph{analog
  component} or  a \emph{binary component}.  Without attempting a
  formal definition, we can say that while it can be acceptable to
  reconstruct an analog component with some error, it is usually
  unacceptable to reconstruct a binary component with error.  
  Examples of  analog and binary components could be, respectively,
  the DCT coefficient  and the motion vectors in a video stream.

  Although it is not required, it reasonable to expect that data with
  the same ``meaning'' will belong to the same component.  It is also
  worth to emphasize that analog components can be ``degraded'' to
  binary ones by considering them as simple ``sequences of bits.''  As
  an extreme choice, one can consider each packet as a single binary
  component.  Actually, it is quite probable that the first version of
  \medusa- will allow only for single-component packets.
\end{description}
%
\begin{commento}
In the most common case (a
single-description, non-scalable, video program) the concept of
content, constituent and stream coincide (that is, the content has
only one constituent which in turn contains only one stream).
\end{commento}

\chapter{Algorithms}
\label{chap:0;medusa_book}

\section{Overview of server-peer and peer-peer interaction}
\label{sect:0.1;medusa_book}

It is worth giving a rough overview of the interactions between peers
and between the server and a peer according to the \medusa- protocol.

\begin{enumerate}
\item
Suppose a new node desires to receive the multimedia content.  The
first step is to contact the root server (\fref[a]{steps}).  The new
node open a \emph{data port} which will be used to receive the data
and a \emph{command port} used to receive commands from the server.
The port numbers are sent to the root server with the join request.
If the new node is behind a NAT it will also carry out a \emph{UDP
hole punching} procedure \cite{UDP_hole}.
\item
The root server, after an authentication phase, begins transmitting
the video data to the new node to allow for an immediate playback
(\fref[b]{steps}).  

Note that in \fref[b]{steps} the data flow is denoted with
\emph{three} dashed lines.  Each line represents a stream with a 1/3
of the bandwidth requested by the original multimedia stream,
therefore the new node needs three streams to recover the video
stream. The reason for splitting the original stream into substreams
is that a typical home user has not enough upload bandwidth to
transmit the whole data stream.  A substream is the \emph{quantum} of
information that we can ask to the node to upload.  (How the original
stream is partitioned in smaller substream is described in
Section~\ref{sect:1.0;medusa_book}). In the example of \fref{steps} we
suppose that each node is able to transmit at most two substreams.

Finally, note that there is nothing magic in the number ``3'' used in
\fref{steps} and that the original stream can be partitioned into any
number of substreams.  The only constraint is that each substream must
require a bandwidth not larger than the smallest upload capacity of a
node. 
\item
While sending the video data to the new node, the root server searchs
for some available peers which are asked so send data to the
\emph{data port} of the new node (\fref[c]{steps}).  When the connection
between the old peers and the new one is established, the server can
stop sending data to the new peer (\fref[d]{steps}).
\item
If necessary, the server can ask the new node to send data to some old
one (\fref[e]{steps} and (f)).
\end{enumerate}
%

\pippolen=.48\textwidth
\begin{figure}
\begin{center}
\begin{tabular}{cc}
\framebox{\unafigura[\pippolen]{step_1}} 
  & \framebox{\unafigura[\pippolen]{step_2}}\\
(a)&(b)\\
\framebox{\unafigura[\pippolen]{step_3}} 
  & \framebox{\unafigura[\pippolen]{step_4}}\\
(c)&(d)\\
\framebox{\unafigura[\pippolen]{step_5}} 
  & \framebox{\unafigura[\pippolen]{step_6}}\\
(e)&(f)
\end{tabular}
\end{center}
\caption{Steps for joining a \medusa- P2P network
\label{fig:steps}}
\end{figure}

It is worth observing that with the proposed approach the client nodes
are totally unaware of the P2P network structure.  This allows to
change the server node allocation algorithm without changing the
clients. 

\begin{commento}
  It is worth observing that many variations are possible on the
  structure of \fref{steps}.  For example, in
  \fref{struttura-provider} the video provider is supposed to have a
  \emph{network} of video servers.  Each new user contacts a
  \emph{dispatcher host} which allocates the new user to a
  specific video server, maybe to the ``closest'' one.  
\end{commento}

\begin{figure}
\centerline{\unafigura{struttura-provider}}
\caption{A possible multi-server scenario.
\label{fig:struttura-provider}}
\end{figure}
%

\section{The chunkless approach}
\label{sect:1.0;medusa_book}

Our solution to  the asymmetric bandwidth problem (see
Section~\ref{sub:1.1;medusa_book}) operates at the \emph{stream}
level, that is, every stream in the multimedia content is processed in
an independent way.  

Our goal is to construct from a stream a set of smaller streams
(called \emph{substreams} in the following), each one requiring a
$1/\redfac$ of the bandwidth of the original stream.  Value $\redfac$
will be called \emph{reduction factor}.

\begin{example}
What is a typical value for $\redfac$?  Suppose to transmit video at
2~Mbit/s and that one can ask a home user to upload a maximum of
256~Kbit/s.  This means that $\redfac$ must be at least $2\cdot
2^{20}/2^{18} = 8$.  We can assume that values of $\redfac$ in the
interval $10$--$20$ will be typical.
\end{example}

In order to reduce the bandwidth of a factor of $\redfac$ without
resorting to chunks, \medusa- employs the following algorithm

\begin{enumerate}
\item
Every multimedia stream is a sequence of packets and each packet is a
sequence of bytes.  Let $K$ be an integer number (typically it will be
$K=2$ or $K=4$) and consider every block of $k$ consecutive bytes in
the packet as an element of finite field\footnote{Yes, folks! We need
Galois fields! No way to avoid it, sorry\ldots} $\gf{2^{8K}}$.  Let
$c_n$ be the $n$-th value obtained in this way.  Organize the $c_n$
values in a matrix $\bfC$ with $\redfac$ rows and as many columns as
needed. 
%
\begin{equation}
\label{eq:0;medusa_book}
\bfC = 
 \left[
   \begin{array}{ccc}
     c_0           & c_{\redfac}   & \cdots \\
     c_1           & c_{\redfac+1} & \cdots \\
     \vdots        & \vdots \\
     c_{\redfac-1} & c_{2\redfac-1} & \cdots \\
   \end{array}
 \right]
\end{equation}
%
If needed, the last column is padded with zeros.
\item
At startup every node chooses a value $\redroot \in \gf{2^{8K}}$ at
random and constructs the row vector
%
\begin{equation}
\label{eq:1;medusa_book}
\redvec = 
\left[ 
  \begin{array}{ccccc}
    1 & \redroot & \redroot^2 & \cdots & \redroot^{\redfac-1}
  \end{array}
\right]
\end{equation}
%
called \emph{reduction vector}.
\item
In order to reduce the packet size of a factor $1/\redfac$  the node
left multiplies matrix $\bfC$ by the reduction vector $\redvec$ in
order to obtain the \emph{reduced packet}
%
\begin{equation}
\label{eq:2;medusa_book}
\bfu = \redvec \bfC
\end{equation}
%
Note that the number of entries in $\bfu$ is $1/R$ of the number of
entries of $\bfC$.
\item
Reduced packet $\bfu$ is transmitted to other peers.
\end{enumerate}
%
In order to recover the original packet a node uses the following
algorithm
\begin{enumerate}
\item
The node receives $\redfac$ reduced packets $\bfu_1$, \ldots,
$\bfu_\redfac$ from $\redfac$ different peers.  Let the $n$-th reduced
vector $\redvec_n$ be based on value $\redroot_n$, that is, 
$\redvec_n = [1, \redroot_n, \cdots, \redroot_n^{\redfac-1}]$.
\item
To recover $\bfC$ the node solves linear system
%
\begin{equation}
\label{eq:3;medusa_book}
\begin{cases}
  \bfu_1 &= \redvec_1 \bfC\\
  \bfu_2 &= \redvec_2 \bfC\\
  \vdots &\qquad \vdots\\
  \bfu_\redfac &= \redvec_\redfac \bfC
\end{cases}
\end{equation}
%
via
%
\begin{equation}
\label{eq:4;medusa_book}
\bfC 
=
\commenta{\left[ 
  \begin{array}{c}
    \qquad \redvec_1 \qquad \\
     \redvec_2  \\
    \vdots\\
     \redvec_\redfac  \\
  \end{array}
\right]^{-1}}{\redmtx^{-1}}
\left[ 
  \begin{array}{c}
    \qquad \bfu_1 \qquad \\
     \bfu_2  \\
    \vdots\\
     \bfu_\redfac  \\
  \end{array}
\right]
\end{equation}
\end{enumerate}

\begin{commento}
Equation \er{eq:4;medusa_book} requires the computation of the inverse
of matrix $\redmtx$.  By writing matrix $\redmtx$ explicitly 
%
\begin{equation}
\label{eq:5;medusa_book}
\redmtx  =
\left[ 
  \begin{array}{ccccc}
    1 & \redroot_1  & \redroot_1^2 & \cdots & \redroot_1^{\redfac-1} \\
    1 & \redroot_2  & \redroot_2^2 & \cdots & \redroot_2^{\redfac-1} \\
    \vdots & \vdots &    \vdots    &        & \vdots \\
    1 & \redroot_\redfac  & \redroot_\redfac^2 
          & \cdots & \redroot_\redfac^{\redfac-1} 
  \end{array}
\right]
\end{equation}
%
it is easy to see that it is a Wandermonde matrix which is granted to
be invertible as soon as all the $\redroot_n$ values are different.

It is also worth observing that matrix $\redmtx$ depends only on the
reduction vectors $\redvec_n$ which in turn depend only on the peers
which are sending us data.  Since the peer set is not likely to change
frequently, one can compute the inverse of $\redmtx$ once and cache
the result.
\end{commento}

\subsection{Robustness to node departure and poisoning attack}
\label{sub:1.0.0;medusa_book}

The chunkless approach give us (almost) for free a way to protect us
against the evanescence problem and the poisoning attack.

\begin{description}
\item[Evanescence]
To gain robustness against sudden node departures and/or packet
losses, it suffices to contact a number of peer $\npeers >
\redfac$. It is clear that \er{eq:4;medusa_book} can be applied as
soon as we receive data from at least $\redfac$ out of $\npeers$.  In
particular, if one node suddenly leaves, we are still able to recover
the video data.
\item[Poisoning]
To defend ourselves against the poisoning attack we\ldots{} still
contact a number of peer $\npeers > \redfac$.  To check if we are
victim of a poisoning attack it suffices
\begin{enumerate}
\item
Reconstruct $\bfC$ using \er{eq:4;medusa_book} and a subset of
$\redfac$ reduced vectors (for example, $\bfu_1$, \ldots, $\bfu_\redfac$).
\item
Multiply the reconstructed $\bfC$ by the reduction vectors relative to
the packets not used at previous step (for example,
$\redvec_{\redfac+1}$, \ldots, $\redvec_\npeers$) and check that the
resulting vectors coincide with received data.  If there is some
discrepancy, someone is trying to ``poison'' us. We can also easily
find who is the poisoner.
\end{enumerate}
%
\end{description}

\subsection{Extensions}
\label{sub:1.0.1;medusa_book}

The first version of \medusa- will employ the procedure described
above to create the substreams.  However, we plan in the future an
extended version of such an approach, here briefly discussed.

The data in a generic multimedia packet can be partitioned according
to their ``nature.'' Some data are related with the samples of the
multimedia signal and have a ``analog'' nature, while some other data
have an inherently ``binary'' nature.  An example of analog data could
be the DCT coefficient in a video stream, while an example of binary
data could be any header information.  The main difference between the
two types of data is that it could be acceptable to reconstruct analog
data with some error, but it is usually unacceptable to reconstruct
binary data with errors.  

In the more general approach packet reduction is carried out as
follows

\begin{enumerate}
\item
The packet is split into analog and binary components.  Note that a
single packet can have \emph{any} number of analog/binary components.
\item
Each binary component is reduced according to the algorithm described
in Section~\ref{sect:1.0;medusa_book}.  Note that different binary
component can be reduced by using different reduction factors.  For
example, a smaller reduction factor can be used with the most
important binary components, in order to be able to reconstruct them
even if fewer packets are received.
\item
Each analog component is reduced according to an algorithm similar to
the one described in Section~\ref{sect:1.0;medusa_book}, but using
real (or complex) numbers instead of finite field elements.  The
result of the reduction is successively encoded in a binary string.
\item
All the reduced components (binary and analog) are collected into a
single packet which is sent to other peers.
\end{enumerate}
%

\section{P2P management: the onion skin approach}
\label{sect:1.1;medusa_book}

As already said, the \medusa- client is totally unaware of the P2P
network topology and any network structure could be used without
changing the client.  Nevertheless, in order to make a complete suite,
it is necessary to decide also the P2P topology.

For the sake of description simplicity we will suppose, for the
moment, that every peer is capable to output at most $\peercap$
reduced streams.  Some important requirements on the P2P
topology are

\begin{description}
  \item[Efficiency] The node output bandwidth must be exploited as
  much as possible (that is, the percentage of nodes which emits less than
  $\peercap$ reduced streams must be small) since any bandwidth not
  covered by the P2P network must be provided by the root server.
  \item[Small diameter] The \emph{diameter} of the network (i.e., the
  largest number of intermediate nodes between any two nodes of the
  network) must be small, since every ``hop'' between adjacent nodes
  introduce a delay and the larger the diameter, the greater the
  maximum delay introduced by the network.
  \item[No loops] Obvious requirement.
  \item[Light-weight management] It must be easy to add a new node and
  to ``repair'' the network when a node leaves.  Light-weight is to be
  interpreted both in terms of server computational cost and number of
  messages sent over the network.
\end{description}

\medusa- server  constructs the P2P network by following a ``onion
skin'' approach.  More precisely,

\begin{enumerate}
\item
Denote with $e_i$ the $\peercap$-dimensional vector whose components
are equal to zero, with the exception of $i$-th
component which is equal to $1$.
\item
To every peer $\peer n$ is assigned an $\peercap$-dimensional 
vector  $\idvec n$ of positive integers.
\item
Peer $\peer n$ sends a substream to peer $\peer m$ if $\idvec m =
\idvec n+e_i$ for some $i$.
\item 
ID vectors $\idvec n$ are assigned in an ``onion skin'' fashion,
according to the infinite norm.  More precisely, a vector $\idvec n$
with
%
$$
\normainf{\idvec n}
  = \max_{k} \abs{(\idvec n)_k} = A
$$
%
can be allocated only after all the vectors with norm less than $A$
have been allocated.
\end{enumerate}
%
\begin{remarks}
\remark
\fref[a]{skin} shows an example of the onion skin allocation procedure
in the case $\peercap=2$.  A circle centered on point $(n,m)$
represents the node with vector ID equal to $[n, m]$. The arrows show
the data-flow direction, while the numbers inside the circles show the
allocation order.  Each skin is labeled with infinite norm of the
points belonging to the skin.  \fref[a]{skin} shows the network
structure after the joining of the 14-th node (dashed nodes denote
still-to-be-allocated positions).  Note that in \fref[a]{skin} the
nodes in a skin are allocated in a counter-clockwise order, but this is
not necessary.  
\remark
An example of data-flow in a
\medusa- network with $\peercap=3$ can be seen in \fref[b]{skin}.
\end{remarks}

\begin{figure}
\begin{center}
\begin{tabular}{c}
\framebox{\unafigura{rete-2d}} \\[1.5em] (a) \\[1.5em]
\framebox{\unafigura{rete-3d}} \\[1.5em] (b)
\end{tabular}
\end{center}
\caption{Example of \medusa- P2P network topology. (a) Case
  $\peercap=2$ (b) Case
  $\peercap=3$
\label{fig:skin}}
\end{figure}

\subsection{Repairing an onion skin network}
\label{sect:0.2;medusa_book}

Repairing an onion skin network is quite easy

\begin{enumerate}
\item
When a node leaves the network the root server is notified.  The
notification can arrive
\begin{itemize}
  \item
    From the leaving node itself, if the node is leaving because the
    user is stopping the client
  \item
    From the neighbor nodes, if the node left without notice
    (e.g. because of a crash)
\end{itemize}
\item
The missing node leaves a ``hole'' in the network.  The server repair
it by taking a node from the most external skin and ``moving'' it in
the place of the leaving node.  

The procedure for ``moving'' peer $\peer n$ in place of $\peer m$ is
the following

\begin{enumerate}
  \item
    Ask to  $\peer n$  to send data to the nodes which were
    served by $\peer m$
  \item
    Ask the nodes which were serving $\peer m$ to redirect their
    data to $\peer n$
  \item
    Ask the nodes which were serving $\peer n$ to close
    the transmission
\end{enumerate}
\end{enumerate}
%

\subsection{Peers with different upload capabilities}
\label{sub:1.1.0;medusa_book}

So far we supposed that every peer in the network has not enough
upload bandwidth to transmit a complete video stream.  Since the total
\emph{download} bandwidth must be equal to the total \emph{upload}
bandwidth, it is easy to see that within such an hypothesis the P2P
network cannot sustain itself and the missing bandwidth must be
provided by the root server.  If the P2P network is going to be
self-sufficient, some nodes must have an upload bandwidth
\emph{greater} than the bandwidth required by the multimedia stream.

It is worth to see how the onion-skin approach can be modified in
order to exploit the presence of nodes with a larger upload
bandwidth. Let $\redfac$ be the reduction factor used in the reduction
algorithm and suppose that to gain reliability we plan to have each
node receiving $\npeers > \redfac$ substreams.  Suppose also that the
nodes can be partitioned into two classes: a \emph{small-bandwidth}
(SBW) class that can transmit $\locap < \redfac$ substreams and a
\emph{large-bandwidth} (LBW) class that can transmit $\hicap >
\npeers$ substreams.  Summarizing,
%
\begin{equation}
\label{eq:6;medusa_book}
\hicap > \npeers > \redfac > \locap
\end{equation}
%
Define also quantities
%
\begin{subequations}
\label{eq:7;medusa_book}
\begin{eqnarray}
\label{eq:7a;medusa_book}
 \defect &=& \npeers-\locap  \\
\label{eq:7b;medusa_book}
 \excess &=& \hicap-\npeers  
\end{eqnarray}
\end{subequations}
%
called respectively \emph{defect} and \emph{excess}.

The idea is to organize SBW nodes and LBW nodes in two different
onion-skin networks.  Since each LBW node can transmit $\hicap >
\npeers$ substreams, the LBW can be organized in a
$\npeers$-dimensional self-sustaining networks and use the $\excess$
streams per node that remain available to feed the SBW nodes.  Note
that every SBW node needs to receive $\defect$ more substreams in
order to have a complete set of $\npeers$ substreams.  We want the
$\defect$ additional substreams to come from $\defect$ different LBW
nodes, in order to gain robustness with respect to node departure.

A possible approach to assign SBW nodes to LBW nodes is the
following.  Suppose $\peer L$ is the $n$-th node in the LBW network.
Node $\peer L$ will transmit to SBW nodes whose index is
%
\begin{equation}
\label{eq:8;medusa_book}
\excess \floor{\frac n \defect} + \alpha,
\qquad \alpha=0, 1, \ldots, \excess-1
\end{equation}
%
For example, suppose $\excess=3$ and $\defect=2$. The mapping between
LBW and SBW indexes is as follows
%
\begin{center}
  \begin{tabular}{l|*{6}c}
    LBW source & 0 & 1 & 2 & 3 & 4 & 5
\\ \hline
    SBW destinations &
0,1,2 & 0,1,2 &
3,4,5 & 3,4,5 & 
6,7,8 & 6,7,8 
  \end{tabular}
\end{center}
%
Note that each SBW node receives from $\defect=2$ LBW nodes and that
each LBW node transmits to $\excess=3$ different nodes.

\chapter{Protocols}
\label{chap:1;medusa_book}

\section{Root server-peer protocol}
\label{sect:1.3;medusa_book}

\subsection{General remarks}
\label{sub:1.3.2;medusa_book}

It is worth beginning with an informal description of the main
characteristics of the protocol.

\begin{itemize}
\item
The message exchanged between the peers and the server are text lines.
Messages share the following common format
%
\begin{verbatim}
     <command> <timestamp> <p1> ... <pN> <CRC32>
\end{verbatim}
%
where 
\begin{center}
\begin{tabular}{p{0.25\textwidth}p{0.5\textwidth}}
  \texttt{<command>}\dotfill&denotes the message type\\
  \texttt{<timestamp>}\dotfill&is an increasing hexadecimal number\\
  \texttt{<p1>\ldots<pN>}\dotfill&are the command parameters\\
  \texttt{<CRC32>}\dotfill&is an hexadecimal number representing the
  result of computing the CRC32 of the remainder of the line
\end{tabular}
\end{center}
%
\item
Fields in the command line are separated by spaces.  If a parameter
can accept any type of strings (e.g., the \texttt{URI} field in
command \texttt{STRT} in the following), spaces and percent signs are
encoded URL-like (i.e., the space is denoted with \texttt{\%20} while
the percent sign with \texttt{\%25}).
\item
A special type of message is the \texttt{RPLY} command, used to reply
to other commands.  The general structure the \texttt{RPLY} command is
%
 \begin{verbatim}
    RPLY <timestamp> <command> <timestamp2> <p1> ... <pN> <CRC32>
\end{verbatim}
%
where \texttt{<timestamp>} is the timestamp associated to the
\texttt{RPLY} message, \texttt{<command>} and \texttt{<timestamp2>}
are, respectively, the first and second fields of the
message we are replying to.
\end{itemize}
%
In the following we are going to describe all the messages currently
implemented.  In order to make the description precise, we are going
to use the Augmented Backus-Naur Form (ABNF) described in the RFC
5234 \cite{rfc5234}.\footnote{See
  \texttt{http://www.ietf.org/rfc/rfc5234.txt}.} 
Remanding to the original document for a formal description of the
ABNF, the main points of the RFC are summarized, for the convenience
of the reader, in Appendix~\ref{chap:3;medusa_book}.  

The basic rules given in B.1 (Core Rules) of RFC 5234 \cite{rfc5234}
will be used.  Among them we recall here

\begin{verbatim}
  ALPHA          =  %x41-5A / %x61-7A   ; A-Z / a-z
  DIGIT          =  %x30-39             ; 0-9
  HEXDIG         =  DIGIT / "A" / "B" / "C" / "D" / "E" / "F"
  SP             =  %x20                ; ASCII space
  LF             =  %x0A                ; ASCII line-feed
  CR             =  %x0D                ; ASCII carriage return
  EOT            =  %x04                ; ASCII end-of-text
  OCTET          =  %x00-FF             ; Generic byte
  VCHAR          =  %x21-7E             ; visible characters
\end{verbatim}
%
Finally, in order to simplify the description of some rule, we extend
the ABNF of \cite{rfc5234} with the following rule: if \ttt rule! is
the name of a rule which always evaluate to a string representing an
integer value, \verb-{rule}- allows to use such an integer in other
rules. For example, in the following we define an array of integers as
follows

\begin{verbatim}
  integer    = 1*DIGIT
  array-int  = length  {length}(SP element)
  length     = integer
  element    = integer
\end{verbatim}
%
According such rules, a \ttt array-int! is an integer value (\ttt
length!) representing the array length, followed by \ttt length!
integers (\ttt element!) separated by spaces (\ttt SP!). Without this
extension it would not be possible to express the constraint that the
number of \ttt element! must be equal to the value of \ttt length!.

\subsection{Common definitions}
\label{sub:1.3.3;medusa_book}

\begin{verbatim}
  integer    = 1*DIGIT                    ; decimal integer

  array-int  = length  {length}element    ; Array of integers
  length     = integer
  element    = integer

  ID-CHAR    = ALPHA / DIGIT / "-" / "_"  ; chars used in identifiers
  name       = 1*ID-CHAR                  ; generic identifier

  base64-data = 1*base64-char
  base64-char = ALPHA / DIGIT / "+" / "/" / "="

  timestamp  = 6HEXDIG
  crc32      = 8HEXDIG

  address    = (ip-address port-type port) / bogus-addr

  bogus-addr = "X"

  ip-address = 8HEXDIG
  port       = 4HEXDIG
  port-type  = "T" / "U" / "X"            ; T -> TCP port
                                          ; U -> UDP port
                                          ; X -> Invalid (used to 
                                          ;      specify "no port")

  node-id    = array-int
\end{verbatim}

\subsection{From server to peer}
\label{sub:1.3.1;medusa_book}

\subsubsection{Authentication request}
\label{subsub:1.3.1.2;medusa_book}

\paragraph{Syntax}
\begin{verbatim}
  auth        = "AUTH" SP timestamp SP challenge [SP signature] 
                  SP crc32 EOT
  challenge   = base64-data
  signature   = algorithm "-" encr
  algorithm   = name
  encr        = base64-data
\end{verbatim}

\paragraph{Semantic}

When a user contacts the server with a \ttt STRT! command the server
can ask the user to authenticate itself via a \ttt AUTH! request.  The
authentication algorithm is more or less a standard challenge-based
procedure.  More precisely,
  \begin{enumerate}
    \item
      Each user has a \ttt username! and a \ttt password!
    \item
      Upon reception of the \ttt AUTH! request the user concatenates
      its \ttt username!, its \ttt password! and the \ttt challenge!
      found in the \ttt AUTH! request.  The resulting string is
      processed via MD5 (maybe in a future we will use other hash
      functions). 
   \item
      The hashed value is sent back to the server together with the
      \ttt username! in a \ttt RPLY! command.
   \item
      The server computes the hash by using the \ttt username! found
      in the \ttt RPLY! command and the password associated with the
      user.  If the two hashes are equal, the user is accepted.
  \end{enumerate}
  Note that with this approach the user is authenticated to the
  server, but the server is not authenticated.  The optional field
  \ttt signature! is reserved for this.  To compute the \ttt
  signature! field the server
  \begin{enumerate}
    \item
       Pad (at one side or both) the challenge field with some random
       data and encrypt the result with some asymmetric algorithm.
       Encode the result in \ttt BASE64! \cite{rfc4648}.  The result
       is represented by rule \ttt encr!
    \item
       Prepend the result with the algorithm name followed by a
       ``minus'' sign \ttt-!.
    \item
      The client can verify the correctness of the \ttt signature!
      field by knowing the public key of the server.
  \end{enumerate}

\subsubsection{Node ID}
\label{subsub:1.3.1.3;medusa_book}

\begin{verbatim}
  id = "ID"   SP timestamp SP node-id SP crc32 EOT  
\end{verbatim}
If the server accepts the new node, it assigns it an ID
  vector (see description of the P2P structure in
  Section~\ref{sect:1.1;medusa_book}).  With this command the server
  communicates to the new node its ID vector.

\subsubsection{P2P management}
\label{subsub:1.3.1.4;medusa_book}

\begin{verbatim}
  peer-list   = "PEER" SP timestamp addr-list SP crc32 EOT
  addr-list   = 1*(SP address SP stream-id)
  
  send        = "SEND" SP timestamp SP stream-id SP address 
                   SP crc32 EOT
  stream-id   = integer
  
  receive     = "RECV" SP timestamp SP port SP source-id 
                   SP crc32 EOT
  source-id   = node-id / "X"
\end{verbatim}

\paragraph{Semantic}
Current protocol structure 

\begin{description}
  \item[PEER]
Used to transmit to the node a list of possible
  peers. \notainterna{To be written} 
\item[RECV] When the server asks (with a \ttt SEND! command) to source
  peer $\peer S$ to send data to destination peer $\peer D$, it sends
  also a \ttt RECV! command to $\peer D$ to announce it that it is
  going to receive data on port number \ttt port! from the peer whose
  ID is \ttt source-id!.  This command is necessary if the node is
  expected to warn the server (with a \ttt LEFT! command) when it
  stops receiving data from peer $\peer S$.

  If $\peer S$ announces its departure (with a \ttt BYE! command) or
  if the server asks (with a \ttt SEND!  command) to source peer
  $\peer S$ to stop data transmission toward $\peer D$, the server
  sends a \ttt RECV! command to $\peer D$ with \ttt source-id! equal
  to \ttt X! to warn node  $\peer D$ that it will receive no more data
  from $\peer S$.  This will avoid $\peer D$ to send useless
  \ttt LEFT! commands.
\item[SEND]  With this command the server requires to the node to send
  its stream number \ttt stream-id! (a positive integer not greater
  than the \ttt capacity! value declared by the node with the \ttt
  STRT! command) to the port described by field \ttt address!.  If
  \ttt address! is an invalid address (that is, it is the single char
  \ttt X! or its \ttt port-type! is \ttt X!) the stream number \ttt
  stream-id! is closed.
\end{description}

\subsubsection{Function definition}
\label{subsub:1.3.1.5;medusa_book}

\paragraph{Syntax}
\begin{verbatim}
  function-def  = "FUNCTION" SP timestamp SP function-name 
                    SP function-code [SP language] SP crc32 EOT
  function-name = name
  function-code = base64-data
  language      = name
\end{verbatim}

\paragraph{Semantic}

In a few places of \medusa- it is necessary to specify how to process
data.  For example, analog components must be encoded in order to be
transmitted to other peers.  In order to keep the \medusa- engine as
general as possible, this document does not fix the encoding
procedure, but allows the server to communicate it to the clients by
declaring the encoder ``name'' via the \ttt ENCODER! attribute of the
\ttt MMEDIA! command.  Another example of server-specified processing
is how to reconstruct a multimedia packet from the set of the
corresponding components (attribute \ttt COMPOSER! of the \ttt MMEDIA!
command). 

Although it is reasonable to assume that one or more ``standard''
processing procedures, in order to keep the client as general as
possible, the protocol includes the command \ttt FUNCTION!  which
allows the server to send to the user a new function to be used in
packet processing.

Field \ttt function-name! is the name of the function.  \ttt
function-code! is the ``binary code'' of the function described in
some language which is still to be defined, most probably some kind of
``virtual machine code'' (e.g. the bytecode of \ttt Lua!\footnote{\ttt
http://www.lua.org/!} or GNU CLISP\footnote{\ttt
http://clisp.cons.org/!}).  One could also imagine a less general
but more efficient language, specifically tailored to the data
processing problem.  In order to allow for different languages, we
introduced the option \ttt language! field.

\subsubsection{Content description}

\paragraph{Syntax}
\begin{verbatim}
  mmedia      = "MMEDIA" SP timestamp SP *(mmedia-attr SP) crc32 EOT
  mmedia-attr = mmedia-nconst / stream-comp / mmedia-mime /
                   mmedia-composer / const-mime-type / mmedia-type /
                   mmedia-struct / const-struct / stream-struct

  constituent-id  = integer
  stream-id       = integer
  class-id        = integer

  mmedia-nconst   = "N-CONSTITUENTS" SP integer
  const-nstreams  = "N-STREAMS" SP integer [SP constituent-id]
  stream-classes  = "N-CLASSES" SP integer [SP stream-id [SP constituent-id]]

  mmedia-struct   = "MMEDIA-STRUCT" SP mime-name
  const-struct    = "CONSTITUENT-STRUCT" SP mime-name [SP constituent-id]
  stream-struct   = "STREAM-STRUCT" SP mime-name
                       [SP stream-id [SP constituent-id]]

  const-mime-type = "CONTENT-TYPE" SP mime-type [SP constituent-id]
  constituent-id  = integer

  pkt-components  = "COMPONENTS" SP *(component-decl ":") component-decl
                      [SP class-id [SP stream-id [SP constituent-id]]]
  component-decl  =  component-type R-factor
  component-type  = "A" / "B"
  R-factor        = integer

  comp-encoder    = "ENCODER" SP encoder [SP decoder] SP component-id 
                      [SP class-id [SP stream-id [SP constituent-id]]]
  encoder  = name
  decoder  = name

  mmedia-composer = "COMPOSER" SP composer-name [SP stream-id]
  composer-name   = name
\end{verbatim}

\paragraph{Semantic}
\label{subsub:1.3.1.1;medusa_book}
 Used to describe to the node the structure of the
  multimedia content
  \begin{description}
    \item[N-CONSTITUENTS] Number of constituents of the multimedia
    content.  By default the content has only one constituents.  Each
    constituent is uniquely determined by its ID, an integer between 0
    and the number of constituents minus one.
    \item[N-STREAMS] Number of streams in a constituent.  If only one
    constituent is present, the \ttt constituent-id! can be omitted.
    Each stream in a constituent is determined by its ID. By default
    each constituent has only one stream. 
    \item[N-CLASSES] Number of packet classes in a stream.  By default
    each stream has only one packet class.  \ttt stream-id! \ttt
    content-id! can be omitted if only one stream/constituent is present.
    \item[CONTENT-TYPE] Declare the \ttt mime-type! of a constituent.
    \item[COMPONENTS] Describe the components in a given class of
    packets.  The string \ttt COMPONENTS! is followed by a list of
    strings of form \ttt component-type R-factor! separated by colons
    (with no spaces in between).  \ttt component-type! can assume
    values \ttt A! (for analog components) or \ttt B! (for binary
    components).  \ttt R-factor! is the reduction factor applyied to
    the component.  For example, the string

    \begin{center}
      \ttt COMPONENTS B4,A5,B3 2 4!
    \end{center}
    declares that the packets of class 2 of the stream 4 of content 0
    (by default) are partitioned into three components: two binary
    components (with reduction factors equal to 4 and 3) and one
    analog component (with reduction factor equal to 5).
    \item[ENCODER] Define the encoder/decoder pair to be used for a
    given component.   Typically we can expect that each encoder will
    have a corresponding default decoder.  Alternatively, the decoder
    can be esplicitely declared.  The default encoder for binary
    components is the trivial encoder.  No default encoder is defined
    for analog components.
    \item[COMPOSER] Define the composer to be used for reconstructing
    the multimedia packets from their components.  If stream packets
    have only one binary component, the default composer is the
    trivial one.  No default composer is defined for the other cases.
  \end{description}


\subsection{From peer to server}
\label{sub:1.3.0;medusa_book}

\paragraph{Syntax}
\label{subsub:1.3.0.0;medusa_book}

\begin{verbatim}
  peer-msg   = strt / rply-auth / bye / left

  strt       = "STRT" SP timestamp SP URL SP data-port SP
                  ctl-port SP capacity SP crc32 EOT
  data-ports = array-int
  ctl-port   = 4HEXDIG
  capacity   = integer
             ; URL is not defined here. See RFC 1738

  rply-auth  = "RPLY" SP timestamp SP "AUTH" SP timestamp SP
                  username SP hash SP crc32 EOT
  username   = 1*VCHAR
  hash       = 1*HEXDIG

  bye        = "BYE"  SP timestamp SP node-id SP crc32 EOT

  left       = "LEFT" SP timestamp SP node-id SP crc32 EOT
\end{verbatim}

\paragraph{Semantic}
\label{subsub:1.3.0.1;medusa_book}

\begin{description}
  \item[STRT]  This is the command sent to the server when a node
  wants to join the P2P network.  The meaning of the fields is as
  follows
  \begin{description}
    \item[\ttt URL!] Identifies the ``program'' that the user wants
    to watch.
    \item[\ttt data-ports!] The number (in hexadecimal format) of the
    UDP ports used by the peer to receive data from other peers.
    \item[\ttt ctl-port!]
    The number (in hexadecimal format) of the
    UDP port used by the peer to receive commands from the server and
    other peers. 
    \item[\ttt capacity!] The number of substreams that the peer is
    able to send.
  \end{description}
  \item[RPLY to AUTH] Used to reply to the \ttt AUTH! request. \ttt
  username! is \ldots{} well\ldots{} the name of the user, while the
  field \ttt hash! is obtained by hashing the username, the password
  and the challenge.  See explanation of command \ttt AUTH!  in
  Section~\ref{subsub:1.3.1.1;medusa_book} for more details.
  \item[BYE] Sent to the server when the node is going to leave. Field
  \ttt node-id! is the ID that the node received from the server with
  the command \ttt ID! 
  \item[LEFT] Sent to the server when a node suddenly stop receiving
  data from a peer. \ttt node-id! is the ID of the disappeared node.
\end{description}

\section{Inter-peer protocol}
\label{sect:1.4;medusa_book}

\subsection{Commands}
\label{sub:1.4.1;medusa_book}

\begin{verbatim}
  rv-idx = "RVIDX" SP vect-index [SP function-name] 
              [SP component-id [SP class-id [SP stream-id]]]
  function-name = name
  vect-index    = integer
\end{verbatim}

\begin{description}
  \item[RVIDX]  With this command a source peer communicates to a
  destination peer the index of the reduction vector used for the \ttt
  component-id!-th component of packets of class \ttt class-id! in the
  \ttt stream-id!-th stream.  As usual, all the ``\ttt-id!'' values
  default to 0.

  Index \ttt vect-index! is mapped to the actual reduction vector by a
  function \ttt function-name!. Field \ttt function-name! can be a
  ``builtin'' function or it can be a function defined by the root
  server with the command \ttt FUNCTION!. If \ttt function-name! is
  missing the following default is used
  \begin{itemize}
  \item
  If the component is a binary one and the chosen field is $\gf{2^d}$,
  the base $b$ for the reduction vector is polynomial
  $\sum_{n=0}^{d-1} b_k x^k$, where $b_k \in \{0,1\}$ and $b_{d-1}
  b_{d-2} \cdots b_0$ is the binary expansion of \ttt integer!
  ($b_{d-1}$ is the most significant bit).  
  \item No default function is (currently) defined for the analog case
  \end{itemize}
\end{description}

\subsection{Data}
\label{sub:1.4.2;medusa_book}

Currently the only data exchanged between different peers is the data
relative to reduced packets.  In order to describe the inter-peer data
format, we will use an \ttt Ada!-like format.  We decided for this
solution, rather then using other standars such XDR
\cite{rfc4506}\footnote{See \ttt
  ftp://ftp.rfc-editor.org/in-notes/std/std67.txt!} since in the \ttt
Ada! version is more explicit the way fields are mapped into bits.
Multibyte data (e.g. \ttt Timestamp! in \ttt Network_Packet!) are
stored in network order (i.e., big endian order).

%%- use the External Data Representation Standard (XDR)
%%- described in RFC 4506 \cite{rfc4506}.\footnote{See \ttt
%%- ftp://ftp.rfc-editor.org/in-notes/std/std67.txt!} Since XDR is very
%%- C-like, we do not include here a summary of it. 
%%- 
%%- A drawback of XDR is that the minimum data size (32 bits) is too
%%- coarse-grained for us.  Because of this, we extend XDR by adding three
%%- data types: \ttt bit!, \ttt byte!  and \ttt short! (16 bit).  The byte
%%- order in a \ttt short!  is similar to the byte order in an \ttt int!
%%- in XDR: first the most significant byte, then the least one.

\paragraph{Syntax}

\begin{verbatim}
  -- Structure of a packet exchanged between peers
  for Network_Packet'Bit_Order use High_Order_First;
  for Network_Packet use
    record
      Data_Pkt     at 0 range 0..0; -- Always 1
      Reserved     at 0 range 1..3; -- Always 0
      N_Components at 0 range 4..7;

      -- Constituent  at 1 range 0..3; -- Maybe not present
      -- Stream       at 1 range 4..7; -- 

      Class        at 1 range 0..3;
      Priority     at 1 range 4..7;

      Timestamp    at 2 range 0..31;

      Payload      at 7 range 0..end;
    end record;



  -- Structure of a single component 
  for Network_Packet'Bit_Order use High_Order_First;
  for Network_Crumb use
    record
       Explicit_RV   at 0 range 0..0;  -- 1 if RV explicitely given
       Payload_Size  at 0 range 1..15;

       case (Explicit_RV)
         when 0 => 
            Payload       at 1 range 0..Payload_Size*8;
         when 1 => 
            RV_Index      at 2 range 0..31;
            Payload       at 6 range 0..Payload_Size*8;
       end case;
    end record;
\end{verbatim}

\paragraph{Semantic}
\label{subsub:1.4.0;medusa_book}

%\begin{figure}
%\centerline{\unafigura{info-byte}}
%\caption{Outline of the \ttt info! field in \ttt network\_packet!
%\label{fig:info-byte}}
%\end{figure}
%%
\begin{description}
  \item[\ttt Network\_Packet!]  This \ttt struct! shows the structure
  of the packets which are sent from one peer to the other.  There is
  one-to-one correspondence between  \ttt
  network\_packet!'s and the packets in the original multimedia
  stream.  Each \ttt network\_packet! is obtained by first decomposing
  the multimedia packet into components, reducing each component and
  finally collecting all the reduced components in a single \ttt
  network\_packet!.  The meaning of the fields is the following
  \begin{description}
    \item[\ttt Data\_Pkt!] The most significant bit of the first byte
    of a data packet is always equal to 1.  This allows to
    distinguish in an easy way data packet from command packet: since
    command packet contains 7-bit ASCII text, the bit corresponding to
    \ttt Data\_Pkt! will always be 0.
    \item[\ttt Reserved!] Three bits reserved for future extensions.
    They MUST be set to zero.
    \item[{\ttt N_Components!}] The number of \emph{crumbs}
    (components reduced and encoded) contained in this packet.
    \item[\ttt Constituent!] The ID of the constituent this packet
    belongs to.  We are not sure if the final version will contain
    this information or if this information will be implicitly bound
    to the port we receive the packets from.  At the moment, we are
    more oriented for the latter solution.
    \item[\ttt Stream!] The ID of the stream this packet
    belongs to.  See remark about the \ttt Constituent! field.
    \item[\ttt Class!] This field allows one to distinguish
    between different packet types in the same stream.  For example,
    in a video stream one could distinguish between packets relative
    to I, P or B frames.
    \item[\ttt Priority!]  This field can be used by rate control
    algorithms to decide which packets can be ``sacrificed'' in order
    to meet the output rate constraints.  The actual meaning of this
    field is still to be decided, but one can expect that it will be
    related to the distorsion one would expect if the packet is lost.
    \item[\ttt Timestamp!] Since there is a one-to-one and onto
    correspondence between multimedia and network packets, each \ttt
    network_packet! conceptually ``inherits'' the timestamp of the
    multimedia packet.  It should be said that, depending on the
    multimedia format, it is not granted that the multimedia timestamp
    is represented by a 32-bit unsigned integer (e.g. a video format
    could have a timestamp expressed in hours, minutes, seconds and
    frame number).  Nevertheless, it is reasonable to assume that for
    every multimedia format one can find a function which bijectively
    maps multimedia timestamps into network timestamps.
  \item[\ttt Payload!]  This is just the sequence of the reduced
  components contained in the packet.
  \end{description}
  \item[\ttt Network_Crumb!]  This struct represents a \emph{crumb},
   i.e., a component (analog or binary) reduced and encoded.
    \begin{description}
	\item[\ttt Explicit\_RV!] This is a 1-bit flag.  If equal to 0
	the reduction vector used for this component is the one
	declared with the \ttt RVIDX! command.  If this field is equal
	to 1 the reduction vector index is explicitly given in the
	\ttt RV_Index! field.

        The reason for introducing this field is that although most
        of the times a node will use the reduction vector declared
        with the \ttt RVIDX! command, it could happen that sometimes
        it is necessary to use a different reduction vector just
        once.  For example, if the node is not able to reconstruct the
        component since it did not receive enough data, it could
        decide to insert in the \ttt Network_Packet! the reduced
        components received by another nodes.  In this case the node
        would set the \ttt Explicit_RV! bit and write the reduction
        vector index in \ttt RV_Index!.
      \item[\ttt Payload\_Length!] This is the number of bytes
      contained in the \ttt payload! field.
      \item[\ttt RV\_Index!] This field is present only if \ttt
      Explicit\_RV! is set and it is the index associated to the
      reduction vector used to reduce this component.
      \item[\ttt Payload!] This is just a byte array with the
      component payload.  The way the payload is interpreted is not
      described here since it depends on the internal details of the
      reduction/reconstruction algorithm.  More precisely, the
      content of  \ttt Payload! will be given as-it-is to a suitable
      \ttt Decoder! object inside the DSP module (see doc on the DSP
      module for details).
    \end{description}
\end{description}

% \begin{commento}
% \label{rem:packet-class}
%   It is worth describing some applicative context where the packet
%   classification word in \ttt network\_packet.info! can prove useful.
% 
%   \begin{itemize}
%     \item
%       A multimedia could be made of several independent streams, for
%       example
%       \begin{itemize}
% 	\item
%           Audio, video and vector graphic in a remote classroom
%           application 
%         \item
%           Base and enhancement layers in a scalable stream
%         \item
%           Different descriptions in multiple description context
%         \item
%            \ldots and maybe a combination of everything\ldots
%       \end{itemize}
%      The packet classification word can be used to differentiate among
%      the different streams.
%    \item
%     Another application of packet classification   is related with congestion control.  If the path toward a peer
%   is congested, a possible solution could be to lower the
%   data rate sent over that path.  Data rate reduction could be
%   achieved, for example, by using a larger reduction factor or,
%   more simply, by dropping some packets, relying on the fact
%   that the destination peer would still be able to reconstruct
%   the original stream by exploiting the redundancy in the
%   received data.  Note that dropping packets at the source peer
%   does not necessarily lower the quality of the reconstructed
%   signal, although it makes the system less robust.
%   
%   A possible refinement of the packet dropping solution could be
%   to store in the packet classification field the ``importance'' of
%   the packet (e.g., the increase in distorsion that would be
%   experienced if the packet was dropped) and drop, if necessary, 
%   the least important packets.
%   \end{itemize}
% \end{commento}

\begin{subappendices}
\section{TCP, UDP, DCCP, \ldots ?}
\label{sect:1.5;medusa_book}

In the discussion above we did not specify which transport protocol is
to be used.  Currently we can distinguish three cases

\begin{description}
  \item[Initial Handshaking] Initiated by the client with the \ttt STRT!
  command.  Currently is done over UDP (for development reasons that
  it is not worth explaining), but it will change to TCP in the
  immediate future.  The Handshaking begins with the \ttt STRT!
  command and ends with the \ttt ID! sent from the server to the
  client and it can include the \ttt AUTH! and the corresponding \ttt
  RPLY!.  The handshaking also includes the \ttt MMEDIA! command.
  \item[Commands to client] such as \ttt SEND!. Currently transmitted
  over UDP, but maybe we will switch to TCP, if we can be granted that
  we can reliably punch TCP holes through NATs.
  \item[Data transmission from peer to peer] Currently this is done
  over UDP without any rate control.  Since we are still in an
  experimental phase, this is not a problem.  In the future some rate
  control will have to be added.  See Section~\ref{sec:rate-control}
  for some ideas about it.
\end{description}

\subsection{Rate control}
\label{sec:rate-control}

This section is very ``raw'' since it is just a ``collection of
ideas'' about the introduction of rate control in \medusa-.

Currently we send the reduced packets over ``bare bones UDP'', without
any rate control.  Since, as said above, we are still in an
experimental phase, we can accept that.  However, some form of rate
control will have to be introduced before the first public release of
\medusa-.

An interesting solution could be the use of the Datagram Congestion
Control Protocol (DCCP) \cite{rfc4340}. DCCP is a transport protocol
(therefore, it is at the same level of TCP and UDP) which does rate
control, but it does not do any retransmission and does not grant for
packet reception.  The DCCP rate control algorithm is not fixed, but
it can be changed, depending on the application needs. The algorithm
to be used is controlled by a parameter called CCID (Congestion
Control ID).  Currently two control algorithms are defined, a TCP-like
control algorithm (CCID 2) \cite{rfc4341} and a control algorithm with
smoother rate variations (CCID 3) \cite{rfc4342} based on TFRC (TCP
Friendly Rate Control) \cite{rfc3448}.  CCID~3 was designed with
multimedia applications in mind.

The drawback of using DCCP is that (it seems that) it is not widely
adopted.  So far, I found only references that there DCCP is included
in Linux kernel 2.6.x, but it is not clear to me if there is a DCCP
stack in Windows, Mac OS, whatever\ldots

An alternative approach could be to write a ``TFRC over UDP'' library
to be used until DCCP gets adopted.  I gave a first look to TFRC
specifications \cite{rfc3448} and the problem does not look too
difficult.  The server should encapsulate the UDP data in a larger
packet with the information required by TRFC, that is,
\begin{itemize}
  \item A sequence number 
  \item A timestamp (in ms)
  \item The estimate round trip time
\end{itemize}
%
The client should send feedback packets with
\begin{itemize}
  \item The timestamp of last packet
  \item The time elapsed between the processing of last packet and the
  feedback  transmission
  \item The rate experienced by the receiver
  \item An estimate of the packet loss probability
\end{itemize}

\subsubsection{API draft for TRFC over UDP}
\label{subsub:sec:rate-control.0;medusa_book}

Actually, it could be a nice idea to write a ``TFRC over UDP'' library
which is not specifically tailored to \medusa-, but that it could be
prove useful in other contexts.  This pseudo-transport protocol would
share the same port space of UDP and maybe will have some handshaking
for connection creation and shutdown.  It would be an ``asymmetric''
protocol, with a node acting as ``server'' (sending only data) and
the other as ``client'' (sending only feedbacks).  A possible API
for an \ttt Ada! package could be

\begin{verbatim}
package TFRC_over_UDP is
   type TFRC_Socket_Type is private;

   type Discarding_Type is new Integer range -1..100;
   type Deadline_Type   is new Float;

   Non_Discardable : constant Discarding_Type := -1;
   No_Deadline     : constant Deadline_Type   := -1.0;


   procedure Connect_Socket
     (Socket : TFRC_Socket_Type;
      Server : in out Sock_Addr_Type);

   procedure Send_Socket
     (Socket      : TFRC_Socket_Type;
      Item        : Ada.Streams.Stream_Element_Array;
      Last        : out Ada.Streams.Stream_Element_Offset;
      Discardable : Discarding_Type := Non_Discardable;
      Deadline    : Deadline_Type   := No_Deadline);

   procedure Receive_Socket
     (Socket : TFRC_Socket_Type;
      Item   : out Ada.Streams.Stream_Element_Array;
      Last   : out Ada.Streams.Stream_Element_Offset);
end TRFC_over_UDP;
\end{verbatim}
%
The interface of the procedures above is the same of the corresponding
procedures in the \ttt GNAT.Socket! package of GNAT.\footnote{See~\ttt 
http://www.iuma.ulpgc.es/users/jmiranda/gnat-rts/rts/g-socket\_\_ads.htm!
(Note: there are \emph{two} underscores before ``\ttt ads!'')}
The only difference between the two packages is the procedure \ttt
Send\_Socket! that in the proposed interface has the two optional parameters
\ttt Discardable! and \ttt Deadline!.

In order to understand the role of the \ttt Discardable! parameter, it is
worth remembering that TFRC controls the output rate by modifying the
number of packets transmitted in a unit of time.  By setting \ttt
Discardable! to a value different from \ttt Non\_Discardable!, we are
saying to the transport level that the packet can be discarded in case
of necessity.  Large  values of \ttt Discardable! correspond to large
probability that the packet is discarded.

\ttt Deadline! is the maximum allowable amount of time (maybe in ms)
before data is transmitted.

\end{subappendices}


\chapter{Internal structure}
\label{chap:2;medusa_book}

In this chapter we are going to move toward a more ``implementative''
point of view.  

\section{Peer structure}
\label{sect:2.1;medusa_book}

\fref{peer-internal} shows the internal structure of a peer.  As it
can be seen from the figure, the peer is implemented as a collection
of \emph{modules}.  Each module can be implemented with a stand-alone
program. Module communication is carried out via local TCP/IP sockets.

\begin{commento}
The reason for using local TCP/IP sockets for inter-module
communication is that we consider this type of Inter Process
Communication (IPC) device to be currently the most portable one.
\end{commento}
%
The modules which compose the peer are

\begin{description}
  \item[Main] This is the program ``run'' by the user.  Its duty is to
  start the other modules and interact with the user (usually via a
  graphical interface)
  \item[DB] 
  In the current peer implementation there are few ``variables'' which
  are ``super-global,'' in the sense that although are written by a
  single module, they must be readable by all the modules.   
  
  \begin{commento}
      An example of super-global value is the number of the data port
      used to receive data from other peers.  As it will be seen in
      the following, that port is opened by the Processing Module, but
      the port number must be known also to the P2P Module.
  \end{commento}

  Rather to have every module to inquiry the others to request the
  value of such super-global variables, we decided to introduce this
  module which plays the role of a ``shared blackboard.''  Each module
  write its super-global variables to the DB module where they can be
  read by the other modules.
  \item[P2P] This module takes care of the P2P interaction with the
  server and (eventually, in a future) with the other peers.  Commands
  such as \ttt STRT!, \ttt AUTH!, \ttt RECV!, 
  \ldots{} are handled by this module (in some case by forwarding them
  to other modules)
  \item[Processing] This is the most complex module.  Its duty is to
  receive the reduced packets from other peers, carry out the
  reconstruction, reduce the reconstructed packets and send the new
  reduced packets to other peers.
  \item[Player] The duty of this module is to receive the packets
  reconstructed by the Processing module and ``show'' them to the
  user.  The role of this module can also be played by a totally
  independent program (such as \texttt{mplayer} or \texttt{VLC}) as
  long as it can read data from a TCP/IP socket.
\end{description}


\begin{figure}
\centerline{\unafigura[0.9\textwidth]{peer-internal}}
\caption{Internal details of a peer and interaction with the ``outer
  world.'' 
\label{fig:peer-internal}}
\end{figure}
%

\begin{commento}
It is worth to describe  the interaction between the peer modules and
the outer world  shown in \fref{peer-internal}
\begin{enumerate}
\item
After launching the program, the user requires through the User
Interface (UI) a programme.  As a consequence, the Main module
forward the user request to the P2P module.
\item
The P2P module contacts the root server and send it a \ttt STRT!
command.  
\item
If the root server issues a \ttt AUTH! command, the P2P contacts the
Main module to ask for the user's credentials.
\item
If the user is correctly authenticated the root server begins
transmitting multimedia data to the node data ports.  The data ports
are listened by the DSP module which begins the reconstruction of the
multimedia packets which are in turn sent to the Player.
\item
If the root server issues a \ttt SEND! command, the P2P module
receives it and forwards it to the DSP module.
\end{enumerate}
%
\end{commento}

\subsection{Inter-module protocol}
\label{sect:2.2;medusa_book}

Each module opens an ``internal'' TCP port for receiving requests from
the other modules.  The general form of the protocol is as follows

\begin{enumerate}
\item
Module $C$ contact module $S$ on its command port, issues a
request and shutdowns its writing side of the connection.
\item
Module $S$ carries out the request, replies to module $C$ and closes
the connection.
\end{enumerate}
%
A description of the currently implemented inter-module commands
follows. 

\subsubsection{General commands}
\label{subsub:2.2.1;medusa_book}

All the modules recognize the following commands

\begin{verbatim}
  general = quit
  quit = "QUIT" SP timestamp SP crc32 EOT
\end{verbatim}
%
Upon reception of the \ttt QUIT! command the module will exit.

\subsubsection{P2P and Main}
\label{subsub:2.2.0;medusa_book}

\begin{verbatim}
  main-to-p2p = play
  play = "PLAY" SP timestamp SP URL SP crc32 EOT
\end{verbatim}
%
This command is sent by the Main module to the P2P module to request a
programme identified by \ttt URL!.

\begin{verbatim}
  p2p-to-main = p2p-auth
  p2p-auth    = "AUTH" SP timestamp SP crc32 EOT
  auth-reply  = "AUTH" SP username SP password EOT
\end{verbatim}
%
The command \ttt p2p-auth! is used by the P2P to ask to the Main
module for the user's credentials.  The Main module replies with \ttt
auth-reply!. 

\subsubsection{P2P and DSP}
\label{subsub:2.2.2;medusa_book}

\paragraph{Syntax}
\begin{verbatim}
  p2p-to-dsp  = send / receive / mmedia / open

  ; send      = Same syntax of the server-to-peer SEND command
  ; receive   = Same syntax of the server-to-peer RECV command
  ; mmedia    = Same syntax of the server-to-peer MMEDIA command

  open        = "OPEN" SP num-ports SP crc32 EOT
  num-ports   = integer

  dsp-to-p2p  = ports 
  ports       = "PORTS" SP port-list SP crc32 EOT
  port-list   = port-number *("," port-number)
  port-number = integer
\end{verbatim}

\paragraph{Semantic}
\begin{description}
   \item[\ttt send!] This command has the same syntax and meaning of
   the corresponding command sent by the server to the peer.  One can
   think that this command  is forwarded verbatim to the DSP module by
   the P2P module.
   \item[\ttt receive!] This command has the same syntax and meaning of
   the corresponding command sent by the server to the peer.  One can
   think that this command  is forwarded verbatim to the DSP module by
   the P2P module.
   \item[\ttt mmedia!] This command has the same syntax and meaning of
   the corresponding command sent by the server to the peer.  One can
   think that this command  is forwarded verbatim to the DSP module by
   the P2P module.  
   \item[\ttt open!] With this command the P2P module asks to the DSP
   module to open \ttt num-ports! ports to be used for receiving data
   from other peers.  If successfull, the DSP module will reply with 
   \ttt ports!.
   \item[\ttt ports!] Used as a reply to the \ttt open! command to
   communicate to the P2P module the list of the ports the DPS module
   is listening to.
\end{description}
%
\subsubsection{DB module protocol}
\label{sub:2.1.0;medusa_book}

\paragraph{Syntax}


\begin{verbatim}
  dot        = %d46 ; "."
  dash       = %d45 ; "-"
  underscore = %d95 ; "_"

  varname-char = ALPHA / DIGIT / dot / dash / underscore
  varname      = ALPHA *varname-char

  length-del-str = length ":" *OCTET
  length         = integer

  db-command = db-get / db-set

  db-get  = "GET" 1*(SP varname) *(SP timeout) EOT
  timeout = integer *1(dot integer)

  db-get  = "SET" SP varname SP length-del-str EOT

  db-replies = db-ok / db-malformed / db-unknown

  db-ok        = "000 OK" LF *length-del-str
  db-malformed = "001 MALFORMED" LF
  db-unknown   = "001 UNKNOWN" LF
\end{verbatim}

\paragraph{Semantic}

\begin{description}
  \item[GET] If all the given \ttt varname! correspond to known
  variables, the DB module replies with a \ttt db-ok! command, writing
  each value as a \emph{length delimited string}, that is, by writing
  the string length (as a number in base 10), followed by a colon
  ``:'', followed by the bytes which make the string.

  If some variable is not known, the DB module suspends the request
  until all the variables are known.  If the last parameter is a
  number, it is interpreted as a timeout value in seconds.  If the
  timeout expires before all the variable become known, the DB reply
  with a \ttt db-unknown!

  If the command is malformed, the DB replies \ttt db-malformed!.
  \item[SET] Assigns to \ttt varname! the value of the second
  parameter express as a length delimited string.  The reply to this
  command is always \ttt db-ok!, unless the command is malformed, in
  which case the DB replies \ttt db-malformed!.
\end{description}


\section{Server structure}
\label{sect:2.0;medusa_book}

To be written\ldots{} Anyway, the server is quite similar to a peer
with a larger upload bandwidth.

\appendix

\chapter{Summary of ABNF (RFC 5234)}
\label{chap:3;medusa_book}

The Augmented Backus-Naur Form (ABNF) described in the RFC
5234 \cite{rfc5234}\footnote{See \texttt{http://www.ietf.org/rfc/rfc5234.txt}.} is
used in this document to describe the communication protocols used in
\medusa-.  Remanding to the original document for a formal description of the
ABNF, here we report, for the convenience of the reader, the main
points of the RFC.

\begin{itemize}
  \item
    ABNF rules have the form
    \begin{verbatim}
      name = elements crlf
    \end{verbatim}
where \texttt{name} is the rule name, \texttt{elements} is one or more
   rule names or terminal specifications (characters or literal
   strings) and \texttt{crlf} is the ``end-of-line.'' The use of
   angular brackets \texttt{<$\cdots$>} around rule names is optional.
\item
Rule names are case insensitive.  Therefore, \texttt{foo},
\texttt{FOO} and \texttt{<FoO>}  denote the same rule name.
\item
Rules resolve in sequence of characters.  In ABNF a character is just
an integer value. The format for integers in ABNF is a '\texttt{\%}'
followed by a base indicator

\begin{center}
  \begin{tabular}{rcl}
    \texttt{b} & $\rightarrow$ & binary \\
    \texttt{d} & $\rightarrow$ & decimal \\
    \texttt{x} & $\rightarrow$ & hexadecimal 
  \end{tabular}
\end{center}
%
followed by the value.  For example, both rules

\begin{verbatim}
  CR = %x0D
  CR = %d13
\end{verbatim}
%
define \texttt{CR} as the ASCII ``carriage return.''
\item
If a rule requires a concatenation of characters, a shortened notation
where successive values are separated by ``.'' can be used.  For
example, the following rule

\begin{verbatim}
  CRLF = %x0D.0A
\end{verbatim}
%
define \texttt{CRLF} as the concatenation of ASCII ``carriage return''
and ``line feed.''  If convenient,  string literals between double
quotes \texttt{"$\cdots$"} can be used, for example
%
\begin{verbatim}
  command = "print"
\end{verbatim}
%
A string literal is always to be interpreted according to the US-ASCII
encoding and it is \emph{case insensitive} (i.e., \texttt{command}
above will match \texttt{print}, \texttt{PRINT}, \texttt{Print},
\texttt{PriNT} and others).
\item
Alternatives are separated by "\texttt{/}", for example
\begin{verbatim}
  DIGIT       =  "0"/"1"/"2"/"3"/"4"/"5"/"6"/"7"/"8"/"9"
\end{verbatim}
%
A ``range notation'' is also possible, for example, the rule above can
also be written as
\begin{verbatim}
  DIGIT       =  %x30-39
\end{verbatim}
\item
Repetition is denoted by means of the "\texttt{*}" prefix.  More
precisely, 
\begin{verbatim}
  <a>*<b>elements
\end{verbatim}
%
indicates at least \texttt{<a>} and at most \texttt{<b>} occurrences
of \texttt{element}.   Shortened forms are
\begin{center}
  \begin{tabular}{rp{0.6\textwidth}}
    \texttt{<a>*element} & at least \texttt{<a>} occurrences of
    \texttt{element} \\
    \texttt{*<b>element} & at most \texttt{<b>} occurrences of
    \texttt{element} \\
    \texttt{*element} & any number (even zero) of occurrences of
    \texttt{element} \\
    \texttt{<n>element} & exactly \texttt{<n>} occurrences of
    \texttt{element} (equivalent to \texttt{<n>*<n>element})\\
  \end{tabular}
\end{center}
\end{itemize}


\bibliographystyle{alpha}
\bibliography{medusa_biblio}

\end{document}

% LocalWords:  english Udine unicast IP multicast DSL chunkless ISP UDP Mbit eq
% LocalWords:  substreams substream Kbit ccc ccccc medusa Wandermonde DCT rete
% LocalWords:  SBW LBW timestamp pN CRC STRT RPLY Naur ABNF AUTH username encr
% LocalWords:  RECV ctl XDR struct bitmask idx IPC mplayer VLC UI DSP auth ok
% LocalWords:  varname crlf FoO rcl PriNT rp biblio DCCP NATs
